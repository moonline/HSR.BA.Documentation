\section*{Rückblick Laurin Murer}
	Das Projekt hat mir aus verschiedenen Gründen sehr grossen Spass bereitet.
	Der erste Grund ist, dass ich mit meinem Projektpartner grosses Glück hatte.
	Obwohl ich bei diesem Projekt das erste Mal mit ihm zusammengearbeitet habe,
	hat die Zusammenarbeit äusserst gut geklappt und wir haben uns oft optimal ergänzt.
	
	Zudem war das Thema inhaltlich sehr spannend.
	Ich kann mir gut vorstellen, \eeppi\ oder ein ähnliches Tool in der Zukunft einmal beruflich einzusetzen.
	Und auch toll war es, dass mehrheitlich alles sehr gut geklappt hat und wir es in etwa so umsetzen konnten, wie wir es auch geplant hatten.
	
	Weniger gefallen hat mir der Umgang mit \LaTeX\ zur Dokumentation des Projekts,
	denn die Formatierung richtig umzusetzen, war teilweise recht aufwändig
	und nicht mehr mit heutigen technischen Standards vergleichbar.
	Aber Microsoft Word als Alternative hätte ich auch nicht gewollt, da wären einfach andere Probleme aufgetreten.
	
	Spannend fand ich TypeScript kennenzulernen. Diese Sprache habe ich bis jetzt noch nicht gekannt.
	In der Studienarbeit haben wir neue Frameworks und Technologien eingesetzt, was sehr viel Neues zu lernen gab.
	Jetzt in der Bachelorarbeit haben wir je genau etwas Neues gelernt: ich eine neue Technologie und Tobias Blaser ein neues Framework.
	Aus meiner Sicht hat sich dies gelohnt. Es ist immer spannend, etwas Neues zu lernen
	und trotzdem bleibt man nicht andauernd stecken, weil man sich mit der Technologie/dem Framework noch nicht auskennt.
	
	Besonders stolz bin ich auf die API-Dokumentation und wie die darin dargestellten Daten erstellt werden.
	Es gefällt mir sehr, dass ich als Entwickler die Dokumentation gerade im Code erstellen kann
	und dass die angezeigte Rückgabe der API aus echten Aufrufen der API besteht.
	
	Bei einem zukünftigen Projekt würde ich mich für einen leistungsfähigeren Entwicklungsserver einsetzen.
	Bei \eeppi\ hatten wir das Problem, dass der Server zu wenig Leistung besitzt um die für Tests benötigten Vagrant-Umgebungen auszuführen.
	Während dem Projekt wurde ich ein grosser Fan von Vagrant und möchte diese Technologie unbedingt auch in zukünftigen Projekten einsetzen.
