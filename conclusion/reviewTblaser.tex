	\section*{Rückblick Tobias Blaser}
		% Persönliche Berichte einschliesslich (selbst-)kritische Reflexion der Studierenden zu ihren Erfahrungen bei der Arbeit
		
		
		% Was hab ich gelernt		
		Viele der von uns eingesetzten Technologien kannte ich bereits oder hatte sogar schon mehrere Projekt damit umgesetzt.
		Trotzdem habe ich auch im Bereich dieser Technologien noch einiges dazugelernt.
		So habe ich bei AngularJS, TypeScript und Vagrant immer wieder Neues entdeckt, obwohl ich schon länger damit arbeite.
		Komplett neu für mich waren das Play-Framework, der Build-Server Jenkins, das Testframework Jasmine und das Projektplanungstool Jira.
		
		
		% Was war gut
		Sehr angenehm war die Zusammenarbeit mit meinem Projektpartner Laurin Murer. Unsere Stärken haben sich optimal ergänzt und wir hatten eine Menge Spass zusammen.
		
		Unsere Planung hat sich gut bewährt und uns viel Stress erspart, da wir gegen das Ende der Arbeit noch Zeit für ein Refactoring zur Verfügung hatten.
		
		
		% Was war nicht gut
		Trotz der guten Planung war es für mich nicht immer einfach, alles unter einen Hut zu kriegen, da ich nebenbei privat und geschäftlich stark engagiert war.
		Zudem hatten wir vor dem ersten Entwicklungsmilestone zu viel Zeit in unsere Tools investiert und mussten darum im anschliessenden Milestone etwas nachlegen beim Entwickeln.
		
		Ebenfalls etwas zeitintensiv erwiesen sich die Vorgaben der HSR, da sie viele Inkonsistenzen und Unklarheiten beinhalteten und wir einige Male nachfragen mussten, welche Vorgaben wir konkret erfüllen müssen.
		
		
		% Was würde ich anders machen?
		Bei einem erneuten ähnlichen Projekt würde ich auf einige generische Konzepte verzichten und eine pragmatischere Umsetzung suchen. So hatten wir auf dem Client ein ausgeklügeltes Factorysystem gebaut, um aus den Daten der Schnittstelle wieder Objekte zu bauen.
		Wir dachten sogar über automatisiertes Lazy-Loading nach. 
		Rückblickend betrachtet wäre eine einfache und pragmatische Lösung un diesem Falle vermutlich weniger Zeitintensiv gewesen.
		

		% Was würde ich wieder so machen?
		Trotz des Aufwandes, den die vielen Tools bei der Konfiguration verursacht hatten, würde ich wieder so viele Tools einsetzen, da sie uns eine Menge Vorteile eingebracht haben.
				
		
		% Fazit
		Zum Schluss ziehe ich eine sehr positive Bilanz.
		Die Arbeit hat Spass gemacht und das Resultat lässt sich sehen.