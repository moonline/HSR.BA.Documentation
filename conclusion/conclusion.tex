\chapter{Schlussfolgerungen}
	% Zielerreichung, offene Punkte, Bugs/technische Probleme, Zusätzliche Features, Ausblick/Erweiterbarkeit
	
	
	
	
	\section{Ausblick}
		\eeppi\ deckt die notwendige Kernfunktionalität ab damit Benutzer angenehm arbeiten können.
		Im Rahmen einer Weiterentwicklung sind viele Möglichkeiten denkbar. 
		Einige wie ein Recht-Rollen System sind eher praktischer Natur, 
		andere wie beispielsweise Vererbungsmöglichkeiten für Tasktemplates würden die Wiederverwendbarkeit verbessern und dem Benutzer einen echten Mehrwert bieten.
		Nachfolgend sind einige denkbare Erweiterungen von \eeppi\ erklärt:
		
		\subsection{Rechte und Rollen}
			
			Ein Recht-Rollen Konzept bietet die Möglichkeit der Mandantenfähigkeit.
			\eeppi\ könnte als Cloud-Service angeboten werden und eine einzige Instanz für mehrere Kunden verwendet werden.
		
		
		\subsection{Vererbende Tasktemplate}
		
			Tasktemplates, die Eigenschaften von andern Templates erben können, 
			minimieren die Anzahl notwendiger Templates und erhöhen die Wiederverwendbarkeit.
			
		
		\subsection{Reporting}
		
			Die Möglichkeit, übertragene Tasks in \eeppi\ einzusehen erspart dem Benutzer den Umweg über das Projektplanungstool.
		
		
		\subsection{Undo von übertragenenen Tasks}
		
			Unter Umständen möchte ein Benutzer erstellte Tasks wieder zurückziehen.
			Sofern Projektplanungstools dies unterstützen, würde dies dem Benutzer eine Undo-Möglichkeit für fehlerhafte erstellte Tasks anbieten.
		
		
		\subsection{Bearbeitungsmöglichkeiten für zu übertragenede Tasks}

			
			
			
		\subsection{Kaskadierende Processors}
		
			Die aktuelle Ausgabe von \eeppi\ unterstützt keine Processors innerhalb von Processors.
			Dem Benutzer würde das verwenden von Processorwerten innerhalb eines Processors die Möglichkeit bieten, Processors wiederzuverwenden und generischer zu gestalten.
			
			
		\subsection{System Status Notification}
		
		
		
	
		\subsubsection{Verwendung multipler Datenquellen (mehrere DKS)}