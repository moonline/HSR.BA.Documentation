\chapter{Ergebnisse}
	
	\section{Zielerreichung}
		Die Zielsetzung von \eeppi\ lautet in der Aufgabenstellung wie folgt:
		\begin{quote}
			Ziel für den Kunden ist es, aus noch offenen und aus bereits getroffenen Architekturentscheidungen  Aufgaben (Tasks) abzuleiten und in eine Taskmanagementsoftware zu überführen, um diese anschliessend in dieser Software verwalten zu können. In dieser Arbeit sollen das Mapping-Konzept und die Tool-Architektur entworfen sowie eine Implementierung in Form eines Tools erstellt werden.
			
			Das Mapping-Konzept beinhaltet die Art der Abbildung von Entscheidungen aus dem CDAR-Tool auf Tasks eines Projektmanagementtools. Die zu entwerfende Toolarchitektur zeigt den konkreten Aufbau einer solchen Applikation.
		\end{quote}
	Aus dieser Zielsetzung leiteten sich drei Fragen ab:
	\begin{enumerate}
		\item{Wie lässt sich Entscheidungsbedarf in Form von agilen Planungsitems darstellen?}
		\item{Welche Umsetzungstasks ergeben sich aus getroffenen Entscheidungen?}
		\item{Wie können die Metamodelle und Werkzeugschnittstellen der beiden Domänen Architekturentscheidungen und Projektplanung aufeinander abgebildet und miteinander integriert werden?}
	\end{enumerate}
	
	\subsection{Hauptziel}
		Das Hauptziel der Arbeit war es, für das Ableiten von Tasks basierend auf Architekturentscheidungen:
		\begin{enumerate}
			\item{ein Mapping-Konzept zu erstellen}
			\item{eine Tool-Architektur zu entwerfen}
			\item{eine Implementierung in Form eines Tools zu erstellen}
		\end{enumerate}
		
		Das Mapping Konzept wurde im Rahmen der Konzeptphase zusammen mit der Tool-Architektur erarbeitet. 
		Diese beiden erfolgreich abgeschlossenen Ziele waren Voraussetzung für das erfolgreiche Abschliessen des dritten Ziels, 
		der Implementierung von \eeppi.
				
		
	
	\subsection{Rückblick auf die Fragestellungen der Aufgabenstellung}
		\subsubsection{Wie lässt sich Entscheidungsbedarf in Form von agilen Planungsitems darstellen?}
			Im Allgemeinen kann gesagt werden,
			dass auch das Treffen einer Entscheidung eine Aufgabe ist
			und deshalb durch einen Task in einem \ppt\ repräsentiert werden kann.
			Ein Benutzer kann den Entscheidungsbedarf in Form einer oder mehrer Taskvorlagen in \eeppi\ modellieren,  
			anschliessend mit der Entscheidung verknüpfen und ins Projektplanungstool übertragen.
			
			Je nach Grösse, Komplexität und Wichtigkeit der Entscheidung kann der Benutzer dafür einen kleinen Tasks (wie "'Entscheid treffen"') oder auch mehrere grosse Tasks erstellen (wie "'Entscheidungssitzung abhalten"' oder "'Alternativen evaluieren"') und diese gegebenfalls hierarchisieren (Subtasks).			
			
			Zusammengefasst stellt der Benutzer Entscheidungsbedarf in Form von agilen Planungsitems dar, indem er diese in \eeppi\ modelliert und in ein \ppt\ übertragen lässt, wo sie verwaltet und weiterverarbeitet werden können.


		\subsubsection{Welche Umsetzungstasks ergeben sich aus getroffenen Entscheidungen?}
			Dies kommt sehr stark auf die entsprechende Entscheidung an.
			Je nach Projekt kann es durchaus generische Tasks wie "'Entscheid dokumentieren"' oder "'Entscheid kommunizieren"' geben,
			doch die meisten Umsetzungstasks sind sehr spezifisch für einzelne Entscheidungen.
			
			Für ähnliche Projekte werden sich ähnliche Tasks Muster ergeben, die wiederverwendbarkeit wird hoch sein.
			Für neue und von bisherigen stark unterschiedliche Projekte wird der Benutzer viele neue Taskvorlagen benötigen, 
			entsprechend präsentieren sich auch die erzeugten Tasks.
			
		
		\subsubsection{Wie können die Metamodelle und Werkzeugschnittstellen der beiden Domänen Architekturentscheidungen und Projektplanung aufeinander abgebildet und miteinander integriert werden?}
			Der kleinste gemeinsame Nenner der drei Domänen - Entscheidungswissensverwaltung, Projektplanung und die \eeppi\-Domäne - stellen die Schnittstellen dar.
			Sie bilden auf jeder Seite einen Teil der Domäne ab und ermöglichen das integrieren eines kleinen Teils einer anderen Domäne. 
			Über diesen Korridor werden Daten ausgetauscht und für die Werkzeuge bereitgestellt.
		
			Konkret bindet \eeppi\ sowohl Entscheidungswissenssysteme als auch \ppt s als externe Systeme an
			und ermöglicht dem Benutzer \eeppi\ sehr flexibel an die beiden Domänen anzupassen.
			
			Modellmässig stellt \eeppi\ sehr wenige Anforderungen an beiden Domänen.
			Konkret muss ein Architekturentscheidungstool folgende Eigenschaften aufweisen:
			\begin{itemize}
				\item{Unterscheidung zwischen Vorlagen und konkreten Elementen}
				\item{Unterscheidung zwischen Eltern- (z.B. Problemen) und Kinder-Elementen (z.B. Optionen)}
				\item{Werkzeuge zur Auflistung aller Elemente und zur Filterung}
			\end{itemize}
			
			Ein \ppt\ muss im Prinzip lediglich einzelne Elemente (z.B. Tasks) erstellen können.
			
			Konkrete Anforderungen an die Elemente der beiden Domänen gibt es nicht,
			es ist aber für eine effiziente Verwendung förderlich,
			wenn sie eine Vielzahl von gleichen oder zumindest ähnlichen Attribute aufweisen.
			Sind die Schnittstellen stark unterschiedlich, sind viele Processors notwendig, um die Daten zu konvertieren.
			

	\subsection{Erfolgsfaktoren}
		In der Aufgabenstellung wurden kritische Erfolgsfaktoren festgelegt.
		Diese werden im folgenden mit Rückblick auf das Projekt beleuchtet.
		
		
		\subsubsection{Niedrige Einstiegshürden für User}
			Dieser Erfolgsfaktor setzt sich aus drei Teilaspekten zusammen.
			
			Der erste Aspekt ist ein geringer Installationsaufwand.
			Die Installation von \eeppi\ ist so einfach wie  die Installation eines Jira und in wenigen Minuten erledigt. 
			Für Benutzer wird das empfohlene Vorgehen im Abschnitt\ \ref{sec:installation} genau beschrieben.
			
			Ein weiterer Aspekt sind Lizenzfragen.
			Eine zu einschränkende Lizenz würde viele potentielle Interessenten abschrecken.
			Aus diesem Grund steht \eeppi, wie in Abschnitt\ \ref{sec:licensing} beschrieben, unter einer sehr offenen Lizenz.
			
			Letzter Aspekt ist die Robustheit im Betrieb.
			\eeppi\ war als Forschungsprojekt nicht längere Zeit im Produktivbetrieb
			und konnte sich deshalb damit nicht beweisen.
			Zur Verbesserung der Robustheit hat das Projektteam darum als letzte Iteration während der Entwicklung eine Phase zur Stabilisierung und Fehlerkorrektur durchgeführt.
		
		
		\subsubsection{Modularität und Erweiterbarkeit}
			Dieser Punkt beleuchtet die Schnittstellen und deren Dokumentation von \eeppi.
			Schnittstellen bietet wie in der Architektur beschrieben lediglich der Server
			und auch nur der Server verwendet externe Schnittstellen.
			Der Client verwendet nur die Schnittstelle des Servers
			und zeigt damit gleich, dass die Schnittstelle des Servers die notwendige Funktionalität anbietet.
			Wie in Abschnitt \ref{sec:apiDocumentation} erläutert, gibt es auch eine ausführliche Dokumentation zur Schnittstelle.
			Der grundlegende Aufbau von \eeppi\ ist mit diesem Dokument hier dokumentiert
			und kann als Basis für Folgearbeiten verwendet werden.
			
			
		\subsubsection{Reife der Konzepte}
			Dieser Erfolgsfaktor widmet sich der Konfigurierbarkeit, der Flexibilität und der Eleganz der Mapping Konzepte von \eeppi.
			Bei der Entwicklung von \eeppi\ hat das Projektteam äusserst Wert auf genau diese Punkte gelegt.
			Sehr viele Parameter lassen sich konfigurieren und auch die grundlegende Funktion,
			das Übertragen von Tasks an ein \ppt, ist direkt durch den Benutzer im Frontend auf hohem Level anpassbar.
			Das Mapping wurde so ausgelegt, dass der Benutzer nicht direkt Tasks erstellen muss,
			sondern dass er in Form eines Metamappings Task-Vorlagen erstellt,
			aus denen bei einer Übertragung in ein \ppt\ dann Tasks erstellt werden.


	
	\section{offene Punkte, Bugs/technische Probleme}
		%TODO: maybe write this! :-)
	
	
	\section{Zusätzliche Features}
		\subsection{API-Dokumentation}
			Zur Dokumentation der API Schnittstelle hat das Projektteam
			einen Mechanismus implementiert, der die Informationen über die Schnittstelle mittels Reflection aus dem Code und aus der Ressourcenkonfiguration des Play-Frameworks ausliest.
			Zusätzlichen werden während der Erzeugung der Dokumentation konkrete Entitäten erstellt und reale API aufrufe gefahren.
			Die entsprechenden Live-Resultate werden als Aufrufresultate in der Dokumentation angezeigt.
			Dadurch erhält ein Leser realistische Benutzungsbeispiele für die API mit passenden Antwortdaten.
			
	
		\subsection{Strukturierung von Tasktemplates}
			Eine Strukturierung von Tasktemplates wurde zugunsten höher prioritärer Features als optionales Feature deklariert.
			Dem Team ist es jedoch in der letzten Woche vor dem Featurefreeze noch gelungen, eine schlanke Umsetzung einer Hierarchisierung von Tasktemplates zu implementieren.
			Dies ermöglicht das erstellen von Subtasktemplates und das Exportieren von Subtasks.
			
		
		\subsection{Im- und Export der Daten von \eeppi}
			Über die API von \eeppi\ lassen sich alle Daten des Systems,
			ausgenommen Benutzerpasswörtern, exportieren und importieren.
			Dies ermöglicht das Zügel von Tasktemplates und Mappings zu einer andern \eeppi -Instanz.
			
			Zusätzlich lassen sich alle Daten auch durch ein Backup der Datenbank sichern und transferieren, da \eeppi\ keine Daten auf dem Dateisystem ablegt.
			
			
		\subsection{Beispielrequest für Redmine}
			Anfang des Projektes wurde von Projektteam und Betreuer beschlossen, 
			das \eeppi\ grundsätzlich an jedes Projektplanungstool angebunden werden können soll, das die technischen Voraussetzungen erfüllt.
			Für die Entwicklung aber der Fokus vor Allem auf einem System liegen soll, um sich nicht auf zu viele verschiedene API's konzentrieren zu müssen.
			Das Projektteam hat zusätzlich zur Beispielkonfiguration für Jira und einer entsprechenden Demoinstallation auch eine Konfiguration für Redmine erstellt, eine Beispielinstallation aufgebaut und dessen Funktion demonstriert.
	
	
	\section{Erweiterungs-Möglichkeiten}
		\eeppi\ zeigt Möglichkeiten und Wege, wie Entscheidungsmanagement und Projektplanung zusammengebracht werden kann.
		Diese neuen Möglichkeiten wecken wiederum Begehrlichkeiten und Ideen.
		Entsteht das Bedürfnis, \eeppi\ zu erweitern, bieten sich verschiedene Möglichkeiten, die nachfolgend kurz angerissen werden.
		

		\subsection{Erweiterung der bestehenden Applikation}
			Die naheliegendste Erweiterungsmöglichkeit ist die direkt Anpassung der Implementierung von \eeppi\.
			Beschränken sich die Anpassungen nur auf de Funktionalität und nicht auf die Menge der persistierten Daten, so muss lediglich die Clientapplikation angepasst werden.
	
		\subsection{Ersatz der Serverapplikation}
			Die Serverapplikation verwaltet primär die Persistenz und stellt diese Funktionalität in Form der API zur Verfügung.
			Soll diese grundlegend verändert werden oder durch eine neue Technologie ersetzt werden,
			kann die Serverapplikation wie ein Modul ausgetauscht werden, ohne den Client anpassen zu müssen.
			
			Die neue Serverapplikation muss lediglich den gleichen Funktionsumfang anbieten und die Daten in einem ähnlichen Aufbau liefern.
			Parameter wie z.B. Resourceadressen lassen sich auf dem Client konfigurieren. 
	
		\subsection{Ersatz der Clientapplikation}
			Die Clientapplikation besitzt den grössten Teil der Logik, so auch die Processors.
			Zudem ist sie verantwortlich für die Darstellung der Benutzeroberfläche.
			Die Benutzeroberläche selbst kann einerseits durch Custom Styles sehr einfach umgestaltet werden, durch anpassen der zu Grunde liegenden Templates sogar in sehr weitem Masse.
			
			Soll die Clientapplikation jedoch komplett ersetzt werden, beispielsweise durch Ablösung einer neuen Technologie, so ist dies problemlos möglich.
			Die neue Applikation kann wie die bestehende Clientapplikation die Serverschnittstelle verwenden.
			

		\subsection{Erstellung eines Proxies}
			Da \eeppi\ eine geringe Kopplung zu den externen Systemen aufweist,
			 ist es mit wenig Aufwand möglich,
			gewisse Funktionen in Form eines Proxies\footnote{Proxy-Pattern: \url{http://c2.com/cgi/wiki?ProxyPattern}} zu implementieren.
			Dabei würde der bestehende Teil von \eeppi\ nicht verändert werden,
			sonder lediglich neue Referenzen zu den externen Systemen konfiguriert werden.
			Dieser Ansatz würde sich beispielsweise gut für die Implementierung von weiteren \ppt-Schnittstellen eignen (siehe \ref{subsec:morePPTInterfaces}).
			
		\subsection{Verwendung der API}
			Bestimmte mögliche neuen Funktionen basieren auf Kernfunktionen von \eeppi,
			bieten aber einen weitergehenden Nutzen für den Benutzer
			(wie beispielsweise der Rückfluss von Informationen des Tasks zurück in das Task-Template, siehe \ref{subsec:informationFlowbackFeature}).
			Solche Funktionen könnten durch eine eigenständige Applikation implementiert werden,
			welche das API von \eeppi\ verwendet.


	\section{Mögliche Erweiterungen}
		\eeppi\ deckt die notwendige Kernfunktionalität ab, damit Benutzer angenehm arbeiten können.
		Im Rahmen einer Weiterentwicklung sind viele Möglichkeiten denkbar. 
		Einige wie ein Rechte-System sind eher praktischer Natur, 
		andere wie beispielsweise Vererbungsmöglichkeiten für Task-Templates würden die Wiederverwendbarkeit verbessern und dem Benutzer einen echten Mehrwert bieten.
		Nachfolgend sind einige denkbare Erweiterungen von \eeppi\ erklärt:
		
		\subsection{Rechte und Rollen}
			Mit einem Rechte- und Rollen-Konzept könnte der Zugriff auf die Daten in \eeppi\ eingeschränkt werden,
			und unterschiedlichen Zielgruppen unterschiedliche Datenstämme angeboten werden.
			
			Zudem würde ein Rechte- und Rollen-Konzept die Möglichkeit der Mandantenfähigkeit bieten.
			\eeppi\ könnte als Cloud-Service angeboten werden und eine einzige Instanz könnte für mehrere Kunden verwendet werden.
			
		
		\subsection{Vererbende Tasktemplate}
			Task-Templates, die Eigenschaften von andern Templates erben können, 
			minimieren die Anzahl notwendiger Templates und erhöhen die Wiederverwendbarkeit.
			
			Beispielsweise ist ein Task-Template für Sitzungen denkbar, mit
			davon abgeleiteten Templates für verschiedene Sitzungsarten.
			Dem Parenttemplate untergeordnete Subtasks wie z.B. "'Zur Sitzung einladen"' oder "'Sitzungs-Protokoll"' versenden, würden automatisch auch auf Subtemplates angewandt, bzw. mit diesen ans \ppt\ übertragen.
			
		
		\subsection{Reporting}
			Eine weitere Erweiterungsmöglichkeit für \eeppi\ stellt eine Übersicht über die in das \ppt\ exportierten Tasks dar.
			Dies würde dem Benutzer den Umweg über das \ppt\ ersparen.
		
		
		\subsection{Undo von übertragenen Tasks}
			Unter Umständen möchte ein Benutzer erstellte Tasks wieder zurückziehen,
			beziehungsweise wieder aus dem \ppt\ löschen.
			Sofern \ppt s dies unterstützen, würde dies dem Benutzer eine Undo-Möglichkeit für fehlerhafte erstellte Tasks anbieten.
		
		
		\subsection{Bearbeitungsmöglichkeiten für zu übertragende Tasks}
			Unter Umständen möchte der Benutzer die Tasks nicht in der erstellten Rohform in das \ppt\ exportieren.
			Eine Bearbeitungsmglichkeit vor dem Übertragen würde dem Benutzer eine nachträgliche Bearbeitung im \ppt\ ersparen.
			Aktuell muss er sie zuerst exportieren und dann direkt im \ppt\ anpassen.
			
			
		\subsection{Kaskadierende Processors}
			Die aktuelle Ausgabe von \eeppi\ unterstützt keine Processors innerhalb von Processors.
			Dem Benutzer würde das verwenden von Processorwerten innerhalb eines Processors die Möglichkeit bieten, Processors wiederzuverwenden und generischer zu gestalten.
			
			
		\subsection{System Status Notification}
			Ein weiteres kleineres Feature stellt eine Anzeige über den Status der angebundenen Systeme dar.
			Denkbar in der Form eines einfachen Icons, das anzeigt, 
			ob die Systeme (korrekt) konfiguriert sind, ob sie erreichbar sind
			und je nach System unter Umständen auch noch ob sie korrekt arbeiten.		
		
		
	
		\subsection{Verwendung mehrerer DKS}
			Aktuell unterstützt \eeppi\ nur ein \dks, welches aber konfiguriert werden kann.
			Eine mögliche Erweiterung wäre die Integration mehrerer konfigurierbarer \dks e als Datenquelle.
			Der Benutzer wäre anschliessend in der Lage, jeweils das gewünschte System für das Mapping oder als Quelle für die Erzeugung der Tasks auszuwählen.
		
		
		\subsection{Verwendung mehrerer Projekte}
			Analog zum vorhergehenden Punkt unterstützt \eeppi\ auch nur ein einziges Projekt.
			Die Möglichkeit mehrerer Projekte ermöglicht einerseits Mandantenfähigkeit in einer weiteren Dimension und andererseits auch Schutz des geistige Eigentum, 
			da zusammen mit dem Rechte-Rollen Konzept einzelnen Benutzern nur bestimmten Projekte zugeteilt werden könnten.


		\subsection{Unterscheidung von verschiedenen \ppt s}
			Benutzer können aktuell in \eeppi\ Request-Templates sowie \ppt-Accounts einem \ppt-Typ zuordnen.
			Allerdings unterstützt \eeppi\ in der jetzigen Version nur einen einzigen \ppt-Typ(-Identifikator).
			Diesem sind auch alle Request-Templates und Accounts zugeordnet.
			
			Eine Unterstützung für mehreren \ppt-Typ(-Identifikatoren) würde Benutzern Verwirrung ersparen,
			welchen Account sie jetzt für welches Request-Template verwenden können.
			
			
		\subsection{Rückfluss von Informationen des Tasks zurück in Task-Template}
		\label{subsec:informationFlowbackFeature}
			Wenn Benutzer einen Task aus einem Task-Template erstellen, wird dieser ins \ppt\ übertragen.
			Der Benutzer bearbeitet diesen dort und aktualisiert ihn während dem Projektverlauf.
			Die Erfahrungen, die dabei gemacht werden, bleiben im Task im \ppt\
			und bringen weiteren Projekten keinen Gewinn.
			
			Eine denkbare Erweiterungsmöglichkeit von \eeppi\ ist eine Rückführungsmöglichkeit von
			Erkenntnissen aus Projekten in die Task-Templates zurück.
			Unter Umständen gäbe es gar die Möglichkeit, dies teilautomatisiert zu tun.
			Beispielsweise könnte man Änderungen der Tasks erkennen und dem Benutzer vorschlagen, diese ins \eeppi\ zu übertragen.
			
			
		\subsection{Rückfluss von Informationen ins Entscheidungswissenssystem}
		\label{subsec:informationFlowbackFeatureDKS}
			Einerseits ist der Rückfluss von statistischen Werten über die Verwendung von Entscheidungen denkbar 
			andererseits ein Rückfluss von Zuständen.
			Wird beispielsweise im \ppt\ der Task "'Session State Evaluieren"' geschlossen und "'DB Session"' als Option ausgewählt, 
			so könnte diese Information im \dks\ dazu dienen, 
			die entsprechende Entscheidung als geschlossen zu markieren und mit der gewählten Option zu verknüpfen.
			Dem Benutzer würde diese Automatisierung Mehrspurigkeiten in verschiedenen Tools ersparen.


		\subsection{Weitere \ppt-Schnittstellen}
		\label{subsec:morePPTInterfaces}
			Aktuell unterstützt \eeppi\ \ppt s, die über eine Json-Schnittstelle verfügen
			und eine Authentifizierung über HTTP-Basic-Authentication ermöglichen.
			Für diese Art der Authentifizierung muss Benutzername und Passwort im Klartext vorliegen.
			
			Für erhöhte Sicherheit und Kompatibilität wäre eine Erweiterungsmöglichkeit von \eeppi\ denkbar,
			die \ppt s auch über weitere Schnittstellen (beispielsweise XML)
			und weitere Authentifizierungs-Methoden (beispielsweise OAuth\footnote{Ein offenes Authentifizierungs-Protokoll. OAuth beispielsweise würde Authentifizierung auch ohne das effektive Passwort ermöglichen: \url{http://oauth.net/2/}}) anbindet.
			
