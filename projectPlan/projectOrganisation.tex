\chapter{Projektorganisation}

	\section{Projektteam}
	\begin{figure}[H]
		\begin{minipage}[b]{0.5\linewidth}
			\includegraphics[width=0.5\textwidth]{projectPlan/media/img/lmurer.jpg}
			\centering
			\caption{Laurin Murer}
			\label{fig:laurinmurer}
		\end{minipage}
		\begin{minipage}[b]{0.5\linewidth}
			\includegraphics[width=0.5\textwidth]{projectPlan/media/img/tblaser.jpg}
			\centering
			\caption{Tobias Blaser}
			\label{fig:tobiasblaser}
		\end{minipage}
	\end{figure}

	\section{Projektbegleitung}
	\begin{figure}[H]
		\begin{minipage}[b]{0.5\linewidth}
			\includegraphics[width=0.5\textwidth]{projectPlan/media/img/ozimmermann.jpg}
			\centering
			\caption{\teacher}
			\label{fig:olafzimmermann}
		\end{minipage}
	\end{figure}
	\teacher\ betreut die Semesterarbeit und begleitet das Team durch regelmässige Meetings.


	\section{Projektmanagement}
		Zur Verwaltung des Projektes hat das Projektteam ein Jira eingesetzt.
		Zur Grobplanung und zur Planung der Milestones wurden Jira-Versions eingesetzt.
		
		\begin{figure}[H]
			\includegraphics[width=\textwidth]{projectPlan/media/img/jiraVersions.jpg}
			\centering
			\caption{Jira Versions/Milestones}
			\label{fig:jiraVersions}
		\end{figure}
		
		Issues wurden entsprechend von uns diesen Versionen zugeordnet.
		Zusätzlich haben wir labels zur Strukturierung eingesetzt.
		
		\begin{figure}[H]
			\includegraphics[width=\textwidth]{projectPlan/media/img/jiraIssuesOpenOrTodo.jpg}
			\centering
			\caption{Jira Issues, sortiert nach Version}
			\label{fig:jiraIssuesOpenOrTodo}
		\end{figure}
		
		Mit der Dauer einer Version sowie den geschätzten Aufwänden
		pro Issue haben wir jeweils eine Milestoneplanung durchgeführt.
		Dies wird von Jira jedoch nicht von Haus aus unterstützt,
		sodas wir dafür die Issue Daten in eine Tabellenkalkulation exportiert haben.
		
		Wir haben jeweils maximal 2/3 der zur Verfügung stehenden Zeit für
		Issues eingeplant und den Rest für Unvorhergesehenes, 
		Meetings und Planung vorgesehen.
		
		\begin{figure}[H]
			\includegraphics[width=\textwidth]{projectPlan/media/img/jiraDashBoard.jpg}
			\centering
			\caption{Jira Dashboard}
			\label{fig:jiraDashBoard}
		\end{figure}
		
		Jira bietet anpassbare Dashboards, 
		die einen Überblick über das laufende Projekt bieten.
		
		Der ActivityStream ermöglicht es uns, zusammen mit der Git History,
		auf einfache Weise nachzuvollziehen, an was der Teampartner in
		den letzten Stunden gearbeitet hat. 
		Dies senkt den Kommunikationsaufwand und die Notwendigkeit,
		jederzeit gemeinsam zu arbeiten.
		
		\begin{figure}[H]
			\includegraphics[width=\textwidth]{projectPlan/media/img/gitHistory.jpg}
			\centering
			\caption{Git History}
			\label{fig:gitHistory}
		\end{figure}
		
		Für grössere Features haben wir Git Flow Featurebranches eingesetzt.
		Für Releases entsprechend Releasebranches.
		Zusätzliche haben wir die Funktion ``Releases'' von Github
		zum Hinzufügen von fertigen Builds zu Releases genutzt.
		
		\begin{figure}[H]
			\includegraphics[width=0.5\textwidth]{projectPlan/media/img/githubReleases.jpg}
			\centering
			\caption{Github Releases mit Build-Archives}
			\label{fig:githubReleases}
		\end{figure}

		
	\section{Qualitätssicherung}
		Um sicherzustellen, das kein Teammitglied Issues schliesst,
		ohne das die Arbeit einem Review unterzogen wurde,
		haben wir den Issue Workflow im Jira entsprechend angepasst.		
		
		\begin{figure}[H]
			\includegraphics[width=0.5\textwidth]{projectPlan/media/img/jiraIssueWorkflow.jpg}
			\centering
			\caption{Jira Issue Workflow}
			\label{fig:jiraIssueWorkflow}
		\end{figure}
		
		Fertig gestellte Issues müssen immer dem andern Teammitglied
		zum Review gesandt werden und tauchen entsprechend auf dessen
		Dashboard als "Ready to review" auf.
		
		Es geht dabei nicht darum, 
		für jeden erledigten Issue den kompletten Code des Andern anzusehen, 
		sondern das Resultat grob anzuschauen, ev. 
		Edge-Cases\footnote{Spezialfälle, bezogen auf Input Daten oder 
		Workflows der grafischen Oberfläche} zu überprüfen. 
		Ein komplettes Code Review jedes Issues wäre zeitlich nicht machbar.