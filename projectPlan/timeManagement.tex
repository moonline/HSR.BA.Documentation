\chapter*{Zeitmanagement}
	\documentPartEntry{Zeitmanagement}
	%TODO: Kapitel und Bild unten mit den letzten Erkenntnissen und erfassten Zeiten nochmals überarbeiten!
	Zur zeitlichen Planung und für die Auswertung der Arbeitszeit haben wir das \ppt\ Jira\footnote{\url{https://de.atlassian.com/software/jira}} verwendet,
	wie wir es auch für das ganze Projektmanagement verwendet haben.
	Wann immer wir einen Task erstellten, haben wir auch die dafür benötigte Dauer geschätzt
	und diese im Jira eingetragen.
	Gleichzeitig haben wir auch die effektive benötigte Zeit gebucht.
	Ein Beispiel für eine Erfassung von Arbeitszeit ist in Abbildung\ \ref{fig:logWork} abgebildet.
	
	\begin{figure}[H]
		\includegraphics[width=0.5\textwidth]{projectPlan/media/img/logWork.png}
		\centering
		\caption{Erfassen von Arbeitszeit im Jira}
		\label{fig:logWork}
	\end{figure}
	
	Die erfassten Arbeitszeiten haben wir mit einem eigenen Ruby-Skript über die REST API von Jira\footnote{\url{https://docs.atlassian.com/jira/REST/latest/}} exportiert
	und in einer Excel-Tabelle jeweils laufend während dem Projekt ausgewertet.
	Grund dafür ist die dafür Fehlende Funktion im Basispaket von Jira\footnote{Es gibt eine kostenpflichtige Erweiterung, die Reporting anbietet}.
	Entstanden ist folgender Graph über die geleistete Arbeit in Abbildung\ \ref{fig:workGraph}.
	Darin sieht man den kontinuierlichen Verlauf während dem Projekt
	und die ausgeglichene Arbeitslast zwischen Tobias Blaser und Laurin Murer.
	Bereits von Anfang der Projektplanung an war geplant, die Arbeit eine Woche vor offiziellem Abgabetermin fertig zu stellen, um einen Puffer zu besitzen.
	
	Aus diesem Grund endet die Linie des Soll-Max in Abbildung\ \ref{fig:workGraph} auch zu dieser Zeit.
	Die beiden Soll-Linien geben die von der HSR vorgegebene minimal und maximal erwarteten Aufwände an.
	
	\begin{figure}[H]
		\includegraphics[width=\textwidth]{projectPlan/media/img/workGraph.pdf}
		\centering
		\caption{Graph über die geleistete Arbeit}
		\label{fig:workGraph}
	\end{figure}
	
	Wir haben von Begin des Projektes geplant, viel in das Projekt zu investieren, um ein sehr gutes Ergebnis zu erreichen.
	
	Die gleichmässig steigenden Geraden zeigen, dass sich unsere Projektplanung bewährt hat.
	Die Steigung dieses Graphs ist in Abbildung\ \ref{fig:weekhours} auch nochmals separat abgebildet.
	In diesem zweiten Graph ist zu sehen, wie viel Zeit wir pro Woche investiert haben.
	Es ist da nochmals deutlich zu erkennen,
	dass die geplanten Zeiten pro Woche über den Gesamtverlauf des Projekts mehr oder weniger im Rahmen liegen.
	Es gibt immer wieder Schwankungen, doch diese werden auch wieder zeitnah ausgeglichen.
	Die grösseren kurzfristigen Schwankungen sind auf Aktivitäten ausserhalb der Bachelorarbeit zurückzuführen,
	wie beispielsweise einem beruflichen Messebesuch in Berlin.

	\begin{figure}[H]
		\includegraphics[width=\textwidth]{projectPlan/media/img/weekhours.pdf}
		\centering
		\caption{Graph über die geleistete Arbeit pro Woche}
		\label{fig:weekhours}
	\end{figure}

	\subsection{Schätzgenauigkeit der Aufwände}
	Wenig verwunderlich liegt die Schätzgenauigkeit bei kleinen Paketen nahe bei der effektiv benötigten Zeit.
	Bei grösseren Paketen ist die Differenz grösser und schwankt zwischen 50\% weniger und 100\% mehr benötigte Zeit als Extremwerte.
	Wir haben dabei häufiger den Aufwand überschätzt als unterschätzt.
	Nach Hochrechnungen liegt die durchschnittlich geschätzte Dauer zwischen 80\% und 100\% der effektiv benötigten Zeit.
	
	In dieser Rechnung nicht miteingeschlossen sind neu erstellte Issues aufgrund der Notwendigkeit der Umsetzung weiterer Teilaspekte eines grösseren Issueblocks.
	Würde man dies mitberücksichtigen, so liegt die effektiv benötigte Zeit schätzungsweise zwischen 20\% und 40\% über der geschätzten Zeit.
	Aufgrund dieser Erkenntnis in den ersten Wochen haben wir für die restlichen Meilensteine jeweils maximal 2/3 der Zeit für Features eingeplant
	und die Restliche Zeit einerseits für Administrative Tätigkeiten und andererseits als Puffer vorgesehen, was sich bewährt hat. 
