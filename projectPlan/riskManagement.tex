\chapter{Risikomanagement}
	\label{risiken} 
	\section{Risiken}
		\newcounter{riskidcounter}
		
		\newcommand{\riskTable}[7]{
			%Die Spalten werden aufgeteilt auf 0.3 resp. 0.7-fache Textbreite
			\noindent
			\refstepcounter{riskidcounter}
			\begin{tabular}{|p{0.3\textwidth} | p{0.7\textwidth} |}
				\hline	
				Risiko-ID 		& \theriskidcounter \\
				\hline
				Titel 			& #1 \\
				Beschreibung 		& #2 \\
				max. Schaden		& #3  \\
				Eintrittswahrscheinlichkeit & #4  \\
				Gewichteter Schaden	& #5  \\
				Vorbeugung		& #6 \\
				Massnahmen		& #7 \\
				\hline
			\end{tabular}
			\hspace{0.5cm}
			\newline	
		}
		
		\riskTable{Qualität der CDAR Schnittstelle}{Reicht die CDAR Schnittstelle nicht, bzw. stellt sie nicht genügend Informationen bereit, so muss die CDAR Server Applikation angepasst werden.}{50h}{0.5}{25h}{-}{Eignung der Schnittstelle durch den Prototyp abklären.}
		
		\riskTable{Einarbeitung Play Framework}{Die Einarbeitung des Teammitglieds, das das Playframework noch nicht kennt, dauert länder als angenommen.}{16h}{0.25}{4h}{Rechtzeitiges Einarbeiten ins Framework}{Funktionen zurückstellen zugunsten Einarbeitung ins Framework.}
		
		\riskTable{Mapping Complexity}{Die Abbildung des Metamappings ist wesentlich komplexer als angenommen.}{16h}{0.25}{4h}{Minimales Mapping bereits im Prototyp umsetzen um Komplexität abschätzen zu können.}{Mapping Features priorisieren, die Kernfunktionalität abdecken, Nebenfunktionalität weglassen}

		\riskTable{Schnittstelleneinheitlichkeit}{Die Schnittstellen der gängigen Projektmanagementsysteme sind zu unterschiedlich, als das sie über einen Adapter mit einer Konfiguration abgedeckt werden können.}{16h}{0.25}{4h}{Bereits während der Prototypenphase verschiedene Schnittstellen berücksichtigen.}{Mehrere Adapter für verschiedene Schnittstellentypen erstellen.}

		
	\section{Umgang mit Risiken}
		Um die Risiken möglichst bald verringen zu können, sollen durch einen Prototyp die grössten Risikopunkte geklärt werden.