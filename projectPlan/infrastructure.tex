\chapter{Infrastruktur}
	\section{Hardware}
		\begin{itemize}
			\setlength{\itemsep}{-\parsep}
			\item Persönliche Entwicklungsgeräte für jedes Teammitglied, bevorzugt Laptop (eigene Geräte)
			\item Zugewiesene Arbeitsplätze im Zimmer 1.206
			\item Virtual Server für Projektmanagement und als Entwicklungsserver
		\end{itemize}

		
	\section{Tools}
		\subsection{Projektmanagement}	
			Jira und Redmine bieten identische Funktionalität und es wurden sowohl mit Jira wie mit Redmine gute Erfahrungen gemacht. 
			Redmine ist in der Basiskonfiguration eher auf RUP ausgerichtet, für Agile Entwicklung werden Plugins benötigt. 
			Jira ist in der Basiskonfiguration auf Agile Entwicklung ausgerichtet. 
			Die Bentzeroberfläche von Jira ist etwas moderner und benutzerfreundlicher gestaltet, ansosten sind sich beide Oberflächen jedoch ähnlich. 
			Jira ist kostenpflichitg, kostet allerdings sehr wenig und wird für nicht kommerzielle Projekte sogar gratis angeboten.
		
			\begin{description}
				\item[Evaluierte Produkte] Jira, Redmine
				\item[Ausgewähltes Produkt] Jira
				\item[Begründung] Jira bietet eine benutzerfreundliche Oberfläche, die erforderliche Funktionalität und ist grundsätzlich auf Agile Entwicklung ausgerichtet
			\end{description}


		\subsection{Versionsverwaltung}
			\subsubsection{Git}
				Git ist ein bewährtes Versionsverwaltungstool, bietet den Vorteil von lokalen Repositories, ist sehr schlank und bringt eine gute Merge-Automatik mit.

			\subsubsection{GitHub}
				Mit GitHub besitzen die Studenten durch andere Projekte bereits Erfahrung. Als
				Studenten haben sie Zugriff auf kostenlose "`Private-Repositories"'. Zudem
				bietet GitHub noch zusätzliche Funktionen wie Wiki, RST- und MD-Viewer sowie
				Repository-Zugriff und Dateibearbeitung über ein Webinterface.
				
			\subsubsection{Git Flow}
				Git Flow automatisiert häufige Git Operationen für einen Entwicklungsorkflow mit Master-, Develop-, und Featurebranches.

			\subsubsection{Backup}
				Ein zusätzliches Backup ist nicht notwendig, da durch die Versionierung mit Git die komplette Versionshistorie bei jedem Teilnehmer vorhanden ist. Somit ist das gesamte Projekt dreifach abgelegt (bei den Entwicklern sowie bei GitHub).
				
				Projekt- und Entwicklungsserverbackups werden regelmässig von Hand durchgeführt.


		\subsection{Dokumentation}
			\subsubsection{Für grosse Dokumentationen und Abgabedokumente: \LaTeX}
				\LaTeX\ ist perfekt geeignet für grosse, gemeinsam zu erarbeitende Dokumente,
				weil die Source-Dateien über Git versioniert und gemergt werden können und wenig
				Platz verbrauchen. 
				Zudem besteht ein sehr kleines Risiko auf Dokumentenverlust
				bzw. Dokumentenfehler durch die Software, weil \LaTeX\ die Source-Dateien gar
				nicht verändert, im Unterschied zu einer Office-Applikation.
				
			\subsubsection{Für Literatur und temporäre Dokumente: Dropbox}
				PDF Dokumente sowie Word Dokumente können nur schlecht versioniert werden mit git, da sie das Repository mit jeder Version aufblähen.
				Daher werden solche Dokumente über Dropbox ausgetauscht und erst eingecheckt, wenn sie keinen oder kaum mehr Änderungen unterliegen.

			\subsubsection{Für Notizen \& Meetingprotokolle: Restructured Text (rst), txt, Markdown (md)}
				Für Notizen und kleine Dokumente reichen RST, TXT oder MD vollständig aus. 
				Sie sind schlank, bieten nur das notwendigste, können versioniert und gemergt werden,
				weil es nur Textfiles sind, und werden von "`GitHub Document aPreview"'	unterstützt.

			\subsubsection{Für Diagramme, Skizzen: LibreOffice Draw (OpenDocument)}
				Wo es nicht anders geht, wird OpenDocument eingesetzt. Dabei wird
				berücksichtigt, dass es über Git nicht inkrementell versioniert und nicht
				gemerged werden kann.


		\subsection{Modeling}
			Als Modeling-Tool wird Astah gewählt, weil es ein gutes den den Studenten
			bekanntes Tool ist.
			Es deckt den geforderten Funktionsumfang grosszügig ab und bietet Image- sowie
			PDF-Export.

			
		%\subsection{UI Drafting}



		\subsection{Building}
			Ein Build-Server soll zum Automatisierten Ausführen der Tests eingesetzt werden.
			Einige JS Testing Frameworks wie JsUnit bietet eine Anbindungsmöglichkeit an Build Server.
			
			Jenkins ist ein etabliertes und weit verbreitetes Produkt, 
			das sich gut konfigurieren lässt und aus diesem Grund eingesetzt wird.


		\subsection{Entwicklungsumgebung}
			Jeder Entwickler verwendet seine eigene bevorzugte Entwicklungsumgebung. 
			
			Lizenzen für Webstorm werden von der Schule zur Verfügung gestellt.


		\subsection{RunTime Environment}
			Virtuellem Linux Server (Ubuntu Server 14.04) von der Schule zur Verfügung gestellt.
