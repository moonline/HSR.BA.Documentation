\chapter{Infrastruktur}	

	\section{Arbeitsplätze, Geräte und Server}
		Um Entwickeln, Builden und Testen zu können, benötigt das Team die folgende Infrastruktur:
		\begin{itemize}
			\setlength{\itemsep}{-\parsep}
			\item Zugewiesene Arbeitsplätze im Zimmer 1.206
			\item Persönliche Entwicklungsgeräte für jedes Teammitglied, bevorzugt Laptop (eigene Geräte)
			\item Virtuelle Server für Projektmanagement und als Entwicklungsserver
		\end{itemize}

		
	\section{Tools}
		\subsection{Projektmanagement}	
			Jira und Redmine bieten ähnliche Funktionalität und Teammitglieder haben mit Jira wie mit Redmine gute Erfahrungen gemacht. 
			Redmine ist in der Basiskonfiguration eher auf RUP ausgerichtet, für Agile Entwicklung werden Plugins benötigt. 
			Jira ist in der Basiskonfiguration auf Agile Entwicklung ausgerichtet. 
			Die Bentzeroberfläche von Jira ist etwas moderner und benutzerfreundlicher gestaltet, ansosten sind sich beide Oberflächen ähnlich. 
			
			Jira ist kostenpflichitg, kostet allerdings nur 10\$ \cite{atlassan_jira_2014} und wird für nicht kommerzielle Projekte sogar gratis angeboten.
			Vom Hersteller selbst angebotene Jira Plugins kosten häufig auch nur 10\$, 
			sodass das Erweitern von Jira mit Plugins auch für kleine Projekte finanziell tragbar ist.
			
			Redmine ist OpenSource und es existieren auch viele kostenlose Plugins.
		
			\begin{description}
				\item[Evaluierte Produkte] Jira, Redmine
				\item[Ausgewähltes Produkt] Jira
				\item[Begründung] Jira bietet eine benutzerfreundliche Oberfläche, die erforderliche Funktionalität und ist grundsätzlich auf Agile Entwicklung ausgerichtet
			\end{description}


		\subsection{Versionsverwaltung}
			\subsubsection{Git}
				\begin{description}
					\item[Ausgewähltes Produkt] Git
					\item[Mögliche Alternativen] SVN
					\item[Begründung] Git ist ein weit verbreitetes Versionsverwaltungstool, das sich bei den Teammitgliedern bei privaten, schulischen und geschäftlichen Projekten bewährt hat. Git bietet den Vorteil von lokalen Repositories, ist wesentlich schneller als SVN, benötigt für vergleichbare Repositories spürbar weniger Platz und bringt eine gute Mergeautomatik mit.
				\end{description}

			\subsubsection{GitHub}
				\begin{description}
					\item[Evaluierte Produkte] HSR Git, GitHub
					\item[Ausgewähltes Produkt] GitHub
					\item[Mögliche Alternativen] GitLab (Self hosted)
					\item[Begründung] Mit GitHub besitzen die Studenten durch andere Projekte bereits Erfahrung. 
						Als Studenten haben sie Zugriff auf kostenlose "`Private-Repositories"'. 
						Zudem bietet GitHub noch zusätzliche Funktionen wie Wiki, RST- und MD-Viewer sowie
						Repository-Zugriff und Dateibearbeitung über ein Webinterface.
				\end{description}
				
			\subsubsection{Git Flow}
				Git Flow\footnote{"`Git Extensions zur Automatisierung von Repository Workflows"': \url{https://github.com/nvie/gitflow}} automatisiert häufige genutzte Git Operationen für einen Entwicklungsorkflow \cite{driessen_successful_2010} mit Master-, Develop-, und Featurebranches.
				Beispiel: Das Releasen einer neuen Version inklusive Branching, Merging und Tagging.
				
				
		\subsection{Literaturverwaltung}
			Zur Verwaltung von Literatur hat sich das Team für "`Zotero"' \footnote{Freie Quellenverwaltungssoftware vom "`Roy Rosenzweig Center for History and New Media"': \url{https://www.zotero.org/}} entschieden.
			Zotero ist den Studenten seit der Einführung in Literaturrecherchen Anfang Studium bekannt und
			bietet viele prakitsche Funktionen, wie das Erzeugen von Literaturnachweisen aus aufgerufenen Webseiten.
			Es ist als Browser Plugin verfügbar, erlaubt das Teilen der Literaturliste unter den Teammitgliedern und bietet einen Export nach BibTex.
			

		\subsection{Dokumentation}
			\subsubsection{Für grosse Dokumentationen und Abgabedokumente: \LaTeX}
				\LaTeX\ ist sehr gut geeignet für grosse, gemeinsam zu erarbeitende Dokumente,
				weil die Source-Dateien über Git versioniert und gemergt werden können und wenig
				Platz verbrauchen. 
				Zudem besteht ein sehr kleines Risiko auf Dokumentenverlust
				bzw. Dokumentenfehler durch die Software, weil \LaTeX\ die Source-Dateien gar
				nicht verändert, im Unterschied zu einer Office-Applikation.
				
			\subsubsection{Für Literatur und temporäre Dokumente: Dropbox}
			PDF Dokumente sowie Office Dokumente können nur schlecht versioniert werden mit git, 
			da sie für Git ein Blob darstellen, und das Repository mit jeder Version unnötig aufblähen.
			Daher werden solche Dokumente über einen Cloudshare ausgetauscht und erst eingecheckt, 
			wenn sie keinen oder kaum mehr Änderungen unterliegen.
				
				\begin{description}
					\item[Evaluierte Produkte] Owncloud HSR, Owncloud L.M., Dropbox
					\item[Ausgewähltes Produkt] Dropbox
					\item[Begründung] OwnCloud wäre die Präferenz des Dozenten wie der Studenten. 
						OwnCloud bietet den Vorteil, dass die Hoheit über die Daten bei den Studenten selbst liegt, bzw. beim Hoster (Die HSR im Falle der HSR Cloud).
						
						Dropbox legt die Daten in der Amazon Cloud ab. 
						Die Datenschutzbestimmungen erlauben den Betreibern nicht nur die Analyse und Verwendung der Daten, 
						sondern auch die Weitergabe an Drittanbieter.
						
						Der Owncloud Client bietet leider keine Möglichkeit, 
						zwischen mehreren Clouds umzuschalten. 
						Daher müsste für die Verwendung einer BA Cloud der persönliche CloudSync deaktiviert werden, 
						was sehr unpraktisch ist.						
						Aus diesem Grund hat sich das Team trotz der Bedenken seitens Datenschutzes für Dropbox anstelle von OwnCloud entschieden.
				\end{description}

			\subsubsection{Für Notizen \& Meetingprotokolle: Restructured Text (rst), Markdown (md)}
				Für Notizen und kleine Dokumente haben sich die Studenten für RST und MD entschieden, 
				da deren Funktionalität für den Zweck vollständig ausreicht und im Zusammenhang mit den eingesetzen Tools einige Vorteile bieten. 
				Sie sind schlank, bieten nur das notwendigste an Markup, können versioniert und gemergt werden,
				weil es nur Textfiles sind, und werden von "`GitHub Document Preview"' unterstützt.

			\subsubsection{Für Diagramme, Skizzen: OpenDocument}
				Für Anwendungsfälle, in denen die bereits genannten Dokumentformate nicht ausreichen, 
				wird OpenDocument eingesetzt. Dabei wird berücksichtigt, 
				dass es über Git nicht inkrementell versioniert und nicht
				gemerged werden kann.
				
			\subsubsection{Dokumentation des API}
				Das \eeppi\ API ist zentraler Teil der Bachelorarbeit und aus diesem Grund ist deren Dokumentation sehr wichtig. Insbesondere die Korrektheit und Aktualität der Schnittstellendokumentation sind entscheidend die Benutzung der Schnittstelle.
				
				Die Dokumentation wird durch den Server generiert und in Form eines HTML Dokuments zur Verfügung gestellt. %TODO: API-Dokumentation in dieser Dokumentation integrieren und dann hier und im Abschnitt Schnittstellen referenzieren.
				Der Server sammelt die dazu notwendigen Informationen aus verschiedenen Quellen:
				\begin{enumerate}
					\item "'routes"'-Konfiguration des Play Framework (Konfiguration des HTTP-Verbs und der URL).
					\item Annotation der Kontrollermethoden (Informationen über notwendige Authentisierung).
					\item Dokumentations-Annotationen der Controllermethoden (Informationen zu Parameter, Aufruf, Responses sowie Beispielaufrufe).
					\item Ergebnisse von simulierten Aufrufen.
						Bei der Auslieferung der Dokumentation ruft der Server wenn möglich selbst die dokumentierten Methoden mit Beispieldaten auf und integriert die Resultate in die Dokumentation.
				\end{enumerate}
				
				Soweit möglich wird hierzu direkt die effektive Konfiguration verwendet. Dies garantiert, dass die Dokumentation stets auf dem neusten Stand ist.
				
				Ausserdem haben wir wo immer sinnvoll jeweils mindestens zwei Beispielaufrufe in die API-Dokumentation integriert.
				
				\subsection{Code Dokumentation}
					\subsubsection{Client}
						Für zentrale und komplexe Klassen wie z.B. das Repository wird Typedoc\url{http://typedoc.io}
						zur Erzeugung einer Dokumentation aus dem TypeScript Code verwendet.


		\subsection{Modelling}
			\begin{description}
				\item[Ausgewähltes Produkt] Astah
				\item[Mögliche Alternativen] Umlet
				\item[Begründung] Astah ist ein gutes, den Studenten bekanntes Tool.
					Es deckt den geforderten Funktionsumfang grosszügig ab und bietet Image- sowie PDF-Export.
			\end{description}
			
		%\subsection{UI Drafting}



		\subsection{Building \& Testing}
			Ein Build-Server soll zum automatisierten Ausführen der Tests eingesetzt werden.
			Einige JS Testing Frameworks wie JsUnit bietet eine Anbindungsmöglichkeit an Build Server.
			
			\begin{description}
				\item[Ausgewähltes Produkt] Jenkins
				\item[Begründung] Jenkins ist ein etabliertes und weit verbreitetes Produkt, 
					das sich gut konfigurieren lässt und aus diesem Grund eingesetzt wird.
					Zudem gibt es für viele Frameworks Plugins, so auch für JSUnit.
			\end{description}
							
				

		\subsection{Individuelle Entwicklungsumgebungen}
			Jeder Entwickler verwendet seine eigene bevorzugte Entwicklungsumgebung.
			Um auch Interaktionen mit externen Systemen zu testen,
			haben wir sogenannte Vagrant\footnote{Virtualisierungsautomatisierungslösung von HashiCorp für verschiedene Virtualisierungsumgebungen: \url{http://www.vagrantup.com/}}-Boxen erstellt.
			Dies sind Konfigurationen für virtuelle Maschinen, die wenige Bytes gross sind
			und mit einem einfachen Befehl komplette Systeme simulieren können.
			Für \eeppi\ haben wir je Vagrant-Umgebungen für folgende Systeme erstellt:
			\begin{itemize}
				\item{CDAR}
				\item{ADRepo}
				\item{\ppt s (Jenkins und Jira)}
				\item{komplettes \eeppi\ (inkl. Umsysteme)}
			\end{itemize}


	\section{Backup}
		\subsection{Persönliche Entwicklungsgeräte}
			\subsubsection{MacBook von Laurin Murer}\label{backupLaurin}
				Die Daten auf dem Rechner von Laurin Murer werden mehrstufig archiviert.
				Einerseits mit TimeMachine\footnote{Backuplösung von Apple: \url{https://www.apple.com/chde/support/timemachine/}} auf eine externe Festplatte und zusätzlich noch mit Wuala\footnote{Schweizer Cloud Speicher Dienst: \url{https://www.wuala.com/}} in die Cloud.

			\subsubsection{Linux Laptop von Tobias Blaser}
				Die Daten auf dem Rechner von Tobias Blaser werden mit BackInTime\footnote{Rsync basierte Backuplösung für Linux: \url{http://backintime.le-web.org/}} auf eine externe Festplatte archiviert.
				
		\subsection{Arbeitsplätze im Zimmer 1.206}
			Auf diesen Rechnern werden keine Daten abgelegt, dadurch müssen sie auch nicht in einem Backup eingeschlossen werden.

		\subsection{Virtual Server}
			Vom virtuellen Server werden lediglich einige Daten archiviert, und zwar jene, welche von uns erstellt wurden.
			Diese Daten werden alle unter /root/to\_backup referenziert.
			Dieser Ordner wird dann regelmässig auf das MacBook von Laurin Murer kopiert, wo es dann in seinem Backup eingeschlossen wird (siehe \ref{backupLaurin}).
		
		\subsection{Repository und Entwicklungsserver}
			Ein zusätzliches Backup des Projektrepositories ist nicht notwendig, da durch die Versionierung mit Git die komplette Versionshistorie bei jedem Teilnehmer vorhanden ist.
			Somit ist das gesamte Projekt dreifach abgelegt (bei den Entwicklern sowie bei GitHub) sowie auch in den entsprechenden Backups integriert.

