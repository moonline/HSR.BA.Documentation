\chapter{Architektur}
	% @var #1 id
	% @var #2 name
	% @var #3 thema
	% @var #4 subject area
	% @var #5 decision made
	% @var #6 issue/problem
	% @var #7 assumptions
	% @var #8 motivation
	% @var #9 justification
	% @var #10 implications
	% @var #11 derived requirements
	% @var #12 related decisions
	\newcommand{\decisionHeader}[4]{
			\textbf{Themengebiet}	& #4	& \textbf{Thema}	& #3 \\
			\hline
			\textbf{Name}		& #2	& \textbf{ID}		& #1 \\
	}
	
	\newcommand{\decisionContent}[8]{
			\textbf{Getroffene Entscheidung} & \multicolumn{3}{l}{\begin{minipage}[b]{0.75\linewidth}#1\end{minipage}} \\
			\hline
			\textbf{Problemstellung}	& \multicolumn{3}{l}{\begin{minipage}[b]{0.75\linewidth}#2\end{minipage}} \\
			\hline
			\textbf{Voraussetzung}		& \multicolumn{3}{l}{\begin{minipage}[b]{0.75\linewidth}#3\end{minipage}} \\
			\hline
			\textbf{Motivation}		& \multicolumn{3}{l}{\begin{minipage}[b]{0.75\linewidth}#4\end{minipage}} \\
			\hline
			\textbf{Begründung}		& \multicolumn{3}{l}{\begin{minipage}[b]{0.75\linewidth}#5\end{minipage}} \\
			\hline
			\textbf{Annahmen}		& \multicolumn{3}{l}{\begin{minipage}[b]{0.75\linewidth}#6\end{minipage}} \\
			\hline
			\textbf{Abgeleitete Anforderungen} & \multicolumn{3}{l}{\begin{minipage}[b]{0.75\linewidth}#7\end{minipage}} \\
			\hline
			\textbf{Verknüpfte Entscheidungen} & \multicolumn{3}{l}{\begin{minipage}[b]{0.75\linewidth}#8\end{minipage}} \\
	}
	
	\newcommand{\decision}[2]{
		\vspace{0.5cm}
		\noindent
		\begin{tabularx}{\textwidth}{|p{0.2\textwidth}|X|l|l|}
			\hline
			#1
			\hline
			#2
			\hline
		\end{tabularx}
		\vspace{0.5cm}
		\newline	
	}

	\section{Erweiterung CDAR}
		Im folgenden werden die Möglichkeiten beschrieben, wie \eeppi\ an das CDAR angebunden werden soll.
		
		\decision{
			\decisionHeader
			{1}
			{Erweiterung CDAR}
			{Architektur}
			{Architektur design}
		}{
			\decisionContent
			{}
			{In welcher Art soll \eeppi\ mit dem CDAR integriert werden}
			{}
			{Möglichst grosse Unabhängigkeit vom CDAR, möglichst einfaches Handling für den Benutzer}{
				Varianten:\newline
				\textbf{Lose Kopplung, keine Integration}\newline
				\eeppi\ bezieht seine Daten über die Schnittstelle des CDAR und benutzt den Login-Mechanismus. 
			Ansonsten ist \eeppi\ unabhängig und könnte auch mit einer andern Applikation als das CDAR gekoppelt werden. 
			Die Schnittstelle zum CDAR muss entsprechend konfigurierbar ausgelegt sein. \newline
				\newline
				\textbf{Lose Kopplung, keine Integration}\newline
				\eeppi\ bildet eine neue serverseitige Komponente sowohl für sich selbst wie für das CDAR.
			CDAR Persistence sowie API werden komplett ersetzt.
			Nur die Clientapplikation wird weiterverwendet.\newline
				\newline
				\textbf{CDAR UI erweitern}\newline
				\eeppi\ benutzt die API des CDAR, besitzt eine eigene Serverkomponente und erweitert die clientseitige Applikation des CDAR.
			}
			{}
			{}
			{"`Server Technologie"'}
		}		
		

	\section{Architekturübersicht}
	
	
	\section{Tier-Architektur}
		Aufgrund der Einschränkung von Technologie und den existierenden Komponenten des CDAR's stehen drei mögliche Tier-Architekturen zur Auswahl.
		Die hier verwendeten Begriffe stützen sich auf von Prof. Dr. Zimmermann verwendete Begriffe in der HSR Vorlesung "`Application Architecture"'\cite{_layers_2014}.
	
		Die Entscheidung ist für "`Distributed Application"', also eine Single Page Application gefallen, 
		weil diese Architektur es ermöglicht, die bestehende Serverlogik des CDAR zu nutzen, 
		eine Serverseitige (zentralisierte) Persistenz anzubieten sowie dem Benutzer eine schnelle und unabhängige Applikation zur Verfügung zu stellen, die die Rechenleistung des Clients beansprucht, sodass der Server und dessen Kosten schlank gehalten werden können.
		Da die App vom Server ausgeliefert wird, kann sie zentral von dort aus verwaltet, gewartet und kontrolliert werden.
		
		\subsection{1-Tier Structure: Centralized Computing (Client-only Application)}
			% TODO: Replace by image
			\begin{verbatim}
			        ------------------
			           Presentation
			Client        Logic
			             Resources
			        ------------------
			                |
			        ------------------
			           Webserver/
			Server       Hoster	
			        ------------------		
			\end{verbatim}

			Die Serverkomponente übernimmt lediglich das Ausliefern einer WebApp. 
			Die WebApp bezieht die Daten direkt aus externen Schnittstellen. 
			Persistenz findet dezentral auf dem Client statt in Form von File Persistence oder 
			Persistence durch das Framework (z.B. HTML5 Storage).
	
		\subsection{2-Tier Structure: Distributed Application (Single Page App)}
			% TODO: Replace by image
			\begin{verbatim}
			        ------------------
			           Presentation
			Client        Logic
			               API
			        ------------------
			                |
			        ------------------
			               API
			           Small Logic
			Server      Resources
			        ------------------		
			\end{verbatim}
			Die Serverkomponente übernimmt Persistenz sowie minimale Logik (z.B. Login).
			Presentation und Logik werden von der Client Komponente übernommen.
			
		\subsection{2-Tier Structure: Distributed Presentation}
			\begin{verbatim}
			        ------------------
			Client  ------------------   
			                ^
			                |
			        ------------------
			           Presentation
			Server        Logic
			            Resources
			        ------------------		
			\end{verbatim}
			Persistenz, Logik und Presentation werden vom Server übernommen.
			Die Presentation wird fertig aufbereitet an den Client gesendet (z.B. HTML Page).
			Es gibt keine aktiven Komponenten auf dem Client ausser asyncron nachladenden Skripten.
			
			
		
