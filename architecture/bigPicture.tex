\chapter{Architektur}
	
	Beim hier verwendeten Entscheidungstemplate handelt es sich um das "`IBM UMF Template for Decision Log"' \cite{hand_ibm_2008}.
	
	\section{Übersicht}
		\eeppi\ soll Entscheidungs-Wissens-Systeme (DKS) mit 
		Projekt-Planungs-Tools (PPT) verbinden.
		
		\begin{figure}[H]
			\includegraphics[width=\textwidth]{architecture/media/img/eeppiSchema.jpg}
			\centering
			\caption{Architektur Übersicht}
			\label{fig:architectureSchema}
		\end{figure}		
	
	
	\section{Anbindung des bestehenden DKS CDAR}
		Für das Anbinden ans \cdar\ gibt es verschiedene Möglichkeiten.
		Wir haben uns für eine eigene Serverkomponente, ein eigenes Userinterface und eine eigene Persistenz entschieden.
		Die Alternativen und die Details zu diesem Entscheid sind nachfolgend aufgeführt.
		
		\decision{
			\decisionHeader{I1}{Erweiterung \cdar\ / Integration}{Integration}{Architektur design}
		}{
			\decisionContent{Eigene Serverkomponente, keine UI Integration, eigene Persistenz}
			{In welcher Art soll \eeppi\ mit dem \cdar\ integriert werden?}
			{Die \cdar\ API Stellt alle benötigten Daten zur Verfügung.}
			{Diese Entscheidung beeinflusst die Möglichkeiten der Technologiewahl, der zu nutzenden Schnittstellen und Komponenten und ist daher Grundlegend für weitere Entscheidungen.}
			{
				Jede der Entscheidungen (Serverkomponente, Clientapplikation, Persistenz) kann unabhängig der andern zwei getroffen werden in diesem Fall. Darum sind hier nur die Alternativen der jeweiligen einzelnen Entscheidungen und nicht alle Kombinationen aufgelistet:
				\begin{description}					
					\item[\cdar\ UI erweitern]
					Integration der \eeppi-Funktionalität ins UI des \cdar.			
					\begin{description}
						\item[Vorteile] Nur eine Applikation für Benutzer
						\item[Nachteile] \cdar\ UI muss angepasst werden, \eeppi\ ist vom \cdar\ abhängig
					\end{description}
					
					\item[Serverkomponente ersetzen]
					\eeppi\ bildet eine gemeinsame neue Serverkomponente, die diejenige des \cdar\ ersetzt.
					\begin{description}
						\item[Vorteile] Einheitlichen Unterbau für \cdar\ und \eeppi, nur eine Schnittstelle, nur eine Serverkomponente, einfachere Installation
						\item[Nachteile] Sehr aufwändig, da die \cdar\ Server Komponente viel zu ersetzende Logik beinhaltet, \eeppi\ ist mit \cdar\ gekoppelt.
					\end{description}
					
					\item[\cdar\ Persistenz erweitern]
					\eeppi\ nutzt die Persistenz des \cdar\ und erweitert diese.
					\begin{description}
						\item[Vorteile] Einfachere Wartung, nur eine Persistenz für Backup
						\item[Nachteile] Kopplung von \eeppi\ an \cdar
					\end{description}
				\end{description}
			}
			{
				\eeppi\ soll die \cdar-API zum Laden der Daten benutzen, jedoch eine eigene Server- sowie UI-Komponente und eine eigene Persistenz besitzen. 
				\begin{description}
					\item[Vorteile] \
						\begin{itemize}
							\item Die Persistenz kann im gleichen System wie \cdar\ untergebracht sein, kann aber auch auf einem komplett andern Host laufen.
							\item Es sind keine Anpassungen an \cdar\ notwendig, weder an der Persistenz, der Serverkomponente noch am UI.
							\item \eeppi\ ist Unabhängig vom \cdar\ und könnte auch mit einer andern Applikation als das \cdar\ gekoppelt werden.
						\end{itemize}
					\item[Nachteile] Benutzer müssen zwei Applikationen nutzen (andere URL als \cdar), die Installation ist komplizierter
				\end{description}
			}
			{}
			{Die Schnittstelle für die Datenquelle (Anbindung \cdar) muss generisch und konfigurierbar gestaltet sein.}
			{"`Server Technologie"', "`Tier Architektur"'}
		}
		
		
		Im Laufe der Prototypenphase wurde seitens der Vertretung der Kundengruppe entschieden, 
		\cdar\ durch eine schlanke Schnittstellenapplikation namens "`ADRepo"' zu ersetzen, 
		die ihre Daten von über die Enterprise Architect Erweiterung "`ADMentor"' bezieht. 
		Diese Veränderung bestätigte die Entscheidung für die gewählte Variante.
		Vom Team war angedacht worden die Authentisierungsschnittstelle des \cdar zu nutzen um Benutzern einen Single Sign On zu ermöglichen.
		Die Ablösung des \cdar bedingte allerdings, das \eeppi\ selbst eine Benutzerverwaltung aufbauen muss.
		

	\section{Architekturübersicht}
	
	
	\section{Tier-Architektur}
		Aufgrund der Technologischen Möglichkeiten und den anzubindenden Schnittstellen stehen drei mögliche Tier-Architekturen zur Auswahl.
		Die hier verwendeten Begriffe stützen sich auf von Prof. Dr. Zimmermann verwendete Begriffe in der HSR Vorlesung "`Application Architecture"'\cite{prof._dr._zimmerman_layers_2014}.		
		
		\begin{figure}[H]
			\includegraphics[width=\textwidth]{architecture/media/img/tierArchitecture.png}
			\centering
			\caption{Architektur Varianten}
			\label{fig:tierArchitecture}
		\end{figure}
		\begin{description}
			\item[1-Tier Structure: Centralized Computing (Client-only Application)]
				Die Serverkomponente übernimmt lediglich das Ausliefern einer WebApp. 
				Die WebApp bezieht die Daten direkt aus externen Schnittstellen. 
				Persistenz findet dezentral auf dem Client statt in Form von File Persistence oder 
				Persistence durch das Framework (z.B. HTML5 Storage).
				
			\item[2-Tier Structure: Distributed Application (Single Page App)]
				Die Serverkomponente übernimmt Persistenz sowie minimale Logik (z.B. Login).
				Presentation und Logik werden von der Client Komponente übernommen.
				
			\item[2-Tier Structure: Distributed Presentation]
				Persistenz, Logik und Presentation werden vom Server übernommen.
				Die Presentation wird fertig aufbereitet an den Client gesendet (z.B. HTML Page).
				Es gibt keine aktiven Komponenten auf dem Client ausgenommen asyncron nachladende Skripte.				
		\end{description}
	
		\decision{
			\decisionHeader{1}{Tier Architektur}{Architektur}{Architektur design}
		}{
			\decisionContent{Distributed Application (Single Page Application)}
			{Welche Tier Architektur soll für \eeppi\ gewählt werden?}
			{}
			{Diese Entscheidung ist wichtig, damit eine möglichst lose Kopplung \& hohe Flexibilität auch für zukünftige, auf \eeppi\ und \cdar\ aufbauende Applikationen, erreicht wird und der Grundstein für Technologieentscheidungen gelegt wird.}
			{"`Centralized Computing"', "`Distributed Presentation"'}
			{"`Distributed Application"' ermöglicht, die bestehende Serverlogik des \cdar\ zu nutzen, 
				eine Serverseitige (zentralisierte) Persistenz anzubieten sowie dem Benutzer eine schnelle und unabhängige Applikation zur Verfügung zu stellen, 
				die die Rechenleistung des Clients beansprucht, sodass der Server und dessen Kosten schlank gehalten werden können.
				Da die App vom Server ausgeliefert wird, kann sie zentral von dort aus verwaltet, gewartet und kontrolliert werden.}
			{}
			{Die Serverkomponente muss Persistenz sowie eine Datenschnittstelle für den Client bereitstellen.}
			{Server Technologie, Client Technologie}
		}
		
	\section{Session State}
		Eine Session (Session State Pattern\footnote{\url{http://www.bettersoftwaredesign.org/Design-Patterns/Enterprise-Application-Architecture-Patterns/Session-State-Patterns}}) kann auf dem Client abgelegt werden, auf dem Server oder persistiert werden.
		
		\decision{
			\decisionHeader{2}{Session State}{Architektur}{Architektur design}
		}{
			\decisionContent{Client Session State}
			{Auf welchem Tier sollen User Sessions gespeichert werden?}
			{}
			{Diese Entscheidung beeinflusst Client wie Server und bestimmt, ob der der Server zustandsbehaftet oder nicht ausgelegt wird.}
			{"`Server Session State"', "`Database Session State"'}
			{"`Client Session State"' ermöglicht es, 
				den Server zustandslos zu implementieren. 
				Dadurch wird eine reine Resourcen-basierte Serverschnittstelle möglich.			
			}
			{}
			{Der Client muss eine Möglichkeit bieten, eine Session zu speichern. Bevorzugt soll dafür ein Session Cookie zum Einsatz kommen.}
			{}
		}
		
	\section{Datenfluss}
		\eeppi\ bezieht Wissens-Daten aus dem Entscheidungs-Wissens-System 
		und liefert Tasks an das Projekt-Planungs-Tool.
		
		Die Übertragung dieser Daten läuft jeweils über den \eeppi-Server, welcher dafür die Rolle eines CORS\footnote{Cross-Origin Resource Sharing: \url{http://de.wikipedia.org/wiki/Cross-Origin_Resource_Sharing}} Proxy einnimmt. 
		
		Dies ist nötig, da Clients nicht ohne zusätzliche Erlaubnis eines Remoteservers
		auf diesen zugreifen dürfen (Cross Origin Restriction).
		Dies ist ein allgemeines Problem und tritt bei vielen Webanwendungen auf.
		
		Anstatt einen eigenen CORS Proxy zu verwenden hätte man auch einen von Drittparteien verwenden können.
		Als Beispiels sei hier \hyperlink{http://www.corsproxy.com/}{corsproxy.com} genannt.
		Dies ist ein CORS Proxy in Form eines Webservices.

		Ein CORS Proxy akzeptiert Remote Requests von irgendwelchen Ursprungsadressen und leitet diesen dann
		an die entsprechenden Server weiter, die keine Cross-Origin-Requests erlauben.
		
		Wir haben uns entschieden selbst einen kleinen CORS Proxy in unsern Server zu integrieren,
		damit die Daten nicht über fremde Services fliessen und
		\eeppi\ auch in einem lokalen Netzwerk mit beschränktem Internetzugriff betrieben werden kann.
		
		\begin{figure}[H]
			\includegraphics[width=\textwidth]{architecture/media/img/eeppiDataflow.jpg}
			\centering
			\caption{Applikationsdatenfluss mit Beispieldatenquelle ADMentor}
			\label{fig:applicationDataFlow}
		\end{figure}		