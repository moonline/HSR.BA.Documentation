\begin{landscape}
\chapter{Verwendete Libraries, Frameworks und Tools}

	\section{Verwendete Frameworks und Libraries}		
	Um nicht von Grund auf alle Komponenten selbst programmieren zu müssen, 
	und um auf die Erkenntnisse anderer Entwickler aufbauen zu können,
	haben wir einige bestehende Frameworks verwendet.
	Diese sind nachfolgend, und die dazugehörenden Lizenzen in Absatz\ \ref{sec:licenses}, aufgeführt.
	
	
	\vspace{0.5cm}
	
	\newcommand{\addLib}[5]{
		#1 & #2 & \url{#3} & #4 & #5 \\
		\hline
	}
	
	
	\begin{tabularx}{\linewidth}{| l r | X | c | l |}
		\hline
		\textbf{Technologie} & \textbf{Version} & \textbf{URL} & \textbf{Lizenz} & \textbf{Verwendung} \\
		\hline \hline
		\addLib{Play Framework}{2.3.6}{https://www.playframework.com/}{Apache 2}{Server Framework}
		\addLib{Hibernate Entitymanager}{4.3.6.Final}{http://hibernate.org/orm/}{LGPL}{Server Library}
		\addLib{PostgreSQL Driver}{9.1-901.jdbc4}{http://mvnrepository.com/artifact/org.postgresql/postgresql}{PostgreSQL}{Server Library}
		\addLib{Jetbrains Annotations}{7.0.2}{http://mvnrepository.com/artifact/com.intellij/annotations}{Apache 2}{Server Code Library}
		\addLib{Mockito}{1.10.8}{https://code.google.com/p/mockito/}{MIT}{Server Test Library}
		\addLib{PowerMock}{1.5.6}{https://code.google.com/p/powermock/}{Apache 2}{Server Test Library}
		\addLib{Selenium}{2.43.1}{http://www.seleniumhq.org/}{Apache 2}{Server Test Library}
		\addLib{Angular JS}{1.3.0}{https://angularjs.org/}{MIT License}{Client Framework}
		\addLib{Jasmine}{2.0}{http://jasmine.github.io/}{MIT}{Client Test Framework}
	\end{tabularx}

	\section{Verwendete Tools und Technologien}
	
		\begin{tabularx}{\linewidth}{| l | l | X |}
			\hline
			TypeScript & \url{http://www.typescriptlang.org/} & Apache2 \\
			\hline
			Java & \url{http://www.oracle.com/technetwork/java/} & GPL, Java Community Process\\
			\hline
			Vagrant & \url{https://www.vagrantup.com/} & MIT \\
			\hline
			Virtualbox & \url{https://www.virtualbox.org/} &  GPL \\
			\hline
			Typedoc & \url{https://www.npmjs.com/package/typedoc} &  Apache 2 \\
			\hline
			Less & \url{http://lesscss.org/} &  Apache 2 \\
			\hline
		\end{tabularx}
		
			
	\section{Lizenzen}
	\label{sec:licenses}
	
	\newcommand{\addLicense}[6]{
		#1 & #2 & \url{#3} & #4 & #5 & #6	\\
		\hline
	}

	\begin{tabularx}{\linewidth}{| l | X | X | p{4.3cm} | p{4.7cm} | p{3.5cm} |}
		\hline
		\textbf{Kurzname} & \textbf{Voller Name} & \textbf{URL} & \textbf{Bedingung\footnote{\label{licensesFootnote}Quelle: gemäss \url{http://choosealicense.com/} \cite{github_choosing_2014}} } & \textbf{Erlaubt\footref{licensesFootnote}} & \textbf{Verboten\footref{licensesFootnote}} \\
		\hline \hline
		\addLicense{Apache 2}{Apache 2 License}{http://www.apache.org/licenses/LICENSE-2.0}{
			• Lizenz und Copyright Informationen beilegen,\newline
			• Änderungs"-his"-to"-rie\newline angeben
		}{
			• Kommerzielle Nutzung,\newline
			• Vertrieb,\newline
			• Veränderung,\newline
			• Patent Erteilung,\newline
			• Privater Gebrauch,\newline
			• Weitere Lizenz
		}{
			• Haftbar machen,\newline
			• Mar"-ken"-kenn"-zei"-chen verwenden
		}
		\addLicense{GPL}{GNU General Public License 2.0}{http://www.gnu.org/licenses/gpl-2.0.html}{
			• Source öffentlich,\newline
			• Lizenz und Copyright Hinweise,\newline
			• Änderungs"-his"-to"-rie\newline angeben
		}{
			• Kommerzielle Nutzung,\newline
			• Vertrieb,\newline
			• Veränderung,\newline
			• Patent Erteilung,\newline
			• Privater Gebrauch
		}{
			• Haftbar machen,\newline
			• Weitere Lizenz
		}
		\addLicense{LGPL}{GNU Lesser General Public License}{https://www.gnu.org/licenses/lgpl.html}{
			• Quellcode öffentlich,\newline
			• Library Verwendung,\newline
			• Lizenz und Copyright Informationen beilegen
		}{
			• Kommerzielle Nutzung,\newline
			• Vertrieb,\newline
			• Veränderung,\newline
			• Patent Erteilung,\newline
			• Privater Gebrauch,\newline
			• Weitere Lizenz
		}{
			• Haftbar machen
		}		
	\end{tabularx}
		
		
	\begin{tabularx}{\linewidth}{| l | X | X | p{4.3cm} | p{4.7cm} | p{3.5cm} |}
		\hline
		\textbf{Kurzname} & \textbf{Voller Name} & \textbf{URL} & \textbf{Bedingung\footref{licensesFootnote}} & \textbf{Erlaubt\footref{licensesFootnote}} & \textbf{Verboten\footref{licensesFootnote}} \\
		\hline \hline
		\addLicense{MIT}{MIT License}{http://opensource.org/licenses/MIT}{
			• Lizenz und Copyright Informationen beilegen
		}{
			• Kommerzielle Nutzung,\newline
			• Vertrieb,\newline
			• Veränderung,\newline
			• Privater Gebrauch,\newline
			• Weitere Lizenz
		}{
			• Haftbar machen
		}
		\addLicense{PostgreSQL}{PostgreSQL License}{http://opensource.org/licenses/postgresql}{
			ähnlich wie BSD oder MIT Lizenz
		}{
			ähnlich wie BSD oder MIT Lizenz
		}{
			ähnlich wie BSD oder MIT Lizenz
		}
	\end{tabularx}
\end{landscape}
