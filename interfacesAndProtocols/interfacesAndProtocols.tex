\chapter{Schnittstellen und Protokolle}
	
\section{RESTfull Schnittstelle}

	\subsection{Server}
		\eeppi\ besitzt eine RESTfull Schnittstelle, die andere Applikationen benutzen können, um \eeppi\ direkt anzusprechen, andererseits wird die Schnittstelle auch von der eigenen Clientapplikation benutzt.
		
		Für die Kommunikation mit Remote-Hosts stellt \eeppi\ Proxies zur Verfügung.
	
	
	\subsection{Client}
		Die \eeppi\ Clientapplikation nutzt die RESTfull Schnittstelle zum Datenaustausch mit dem Server sowie zur Kommunikation mit Remote-Hosts über die Proxies (Cross-Original Aufruf).
		
		Um auf dem Client einfach und flexibel Prototypen aus übertragenen JSON-Objekten instanziieren zu können, gibt es eine Object Factory.
		Diese baut anhand einer Factorykonfiguration, die jedes übertragbare Objekt deklarieren muss, Objekte zusammen und füllt sie mit Daten.
		Diese zusätzliche Konfiguration ist notwendig, da JavaScript nicht genügend Typeninformationen besitzt, aus denen sich die erforderlichen Informationen ermitteln liessen und TypeScript diese nicht automatisch generieren kann.
		Alternativ liesse sich die Prototypeninstanziierung  durch Factoryfunktionen pro Objekt umsetzen, dies hat sich jedoch während der Entwicklung als wartungsintensiv und Duplicated-Code-lastig erwiesen.