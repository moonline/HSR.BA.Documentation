\chapter{Glossar}
	\begin{description}
		\item[API]{Application Programming Interface, Schnittstelle eines Programms zum Austausch mit externen Anwendungen}
		\item[Boilerplate-Code]{Code der (wiederholt) geschrieben werden muss, ohne dass er wirklich etwas zum Projekt beiträgt}
		\item[CSS]{Cascading Style Sheets: Auszeichnungssprache für Webdokumente, um diese grafisch zu gestalten}
		\item[\cdar]{Collaborative Decision Management and Architectural Refactoring Tool \cite{tinner_collaborative_2014}}
		\item[CORS]{Cross-Origin Resource Sharing: Teilen von REST-Ressourcen über mehrere Origins (Server)}
		\item[DKS]{\dks}
		\item[\eeppi]{Entwurfsentscheidungen als Projektplanungsinstrument}
		\item[Entscheidungsprojekt]{siehe Solution Space}
		\item[IFS]{Institut für Software, HSR Hochschule für Technik: \url{http://www.ifs.hsr.ch/}}
		\item[Less]{Sprache, um CSS vereinfacht zu generieren}
		\item[LOC]{Lines of Code: Anzahl Zeilen Code}
		\item[Problem Space]{In \cdar\ abgelegtes Wissen, welches später über einen Solution Space hilft Projekte umzusetzen. (siehe auch Kapitel~\ref{userstoryDefinitions})}
		\item[\ppt]{Ein Tool, welches Tasks verwaltet. Es wird zum Planen und Durchführen von Projekten verwendet. (siehe auch Kapitel~\ref{userstoryDefinitions})}
		\item[REST]{Representational State Transfer: Art einer Schnittstelle von Webapplikationen. Jede Ressource hat eine eigene URL und kann mit der korrekten Verwendung der HTTP-Verben bearbeitet werden. (siehe auch \url{http://en.wikipedia.org/wiki/Representational_state_transfer})}
		\item[Selenium]{Testframework, welches einen Browser startet und direkt darin die Webseite testet (siehe auch \url{http://www.seleniumhq.org/})}
		\item[SLOC]{Source Lines of Code: Total Anzahl Zeilen eines Codes, inklusive der irrelevanten Zeilen wie Leerzeilen}
		\item[Solution Space]{Kopie eines Problem Spaces um konkret Entscheide für ein Projekt zu erstellen. (siehe auch Kapitel~\ref{userstoryDefinitions})}
		\item[Tasks]{Aufgabe, welche üblicherweise in einem \ppt\ abgelegt ist. (siehe auch Kapitel~\ref{userstoryDefinitions})}
		\item[TDD]{Test Driven Development: Ansatz des Entwicklungsprozesses, bei welchem zuerst die Tests geschrieben werden und erst danach der entsprechende Code}
		\item[Vagrant]{Virtualisierungsautomatisierungslösung von HashiCorp für verschiedene Virtualisierungsumgebungen: \url{http://www.vagrantup.com/}}
		\item[Wissensbaum]{siehe Problem Space}
		\item[Wissenskonsument]{Person, die \cdar\ und \eeppi\ für die Umsetzung von einem Projekt verwendet. (siehe auch Kapitel~\ref{userstoryDefinitions})}
		\item[Wissensproduzent]{Person, die Wissen im \cdar/\eeppi\ erfasst. (siehe auch Kapitel~\ref{userstoryDefinitions})}
		\item[MVW/MV*] Mode View Whatever: Fasst die Patterns MVC - Model View Controller, MVP - Model View Presenter, MVVM - Model View Viewmodel und ähnliche Patterns zusammen.
	\end{description}