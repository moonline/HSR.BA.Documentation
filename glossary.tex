\chapter{Glossar}
	\begin{description}
		\item[API]{Application Programming Interface, Schnittstelle eines Programms zum Austausch mit externen Anwendungen}
		\item[Boilerplate-Code]{Code der (wiederholt) geschrieben werden muss, ohne dass er wirklich etwas zum Projekt beiträgt}
		\item[Entscheidungsprojekt]{siehe Solution Space}
%		\item[Entscheidung]{Wahl mit mehreren Optionen. (siehe auch Seite~\pageref{userstoryDefinitions})}
%		\item[Entscheid]{Entscheidung mit gewählter Option. (siehe auch Seite~\pageref{userstoryDefinitions})}
%		\item[Entscheid, offen]{Entscheidung mit noch nicht gewählter Option, aber eine Option soll demnächst gewählt werden. (siehe auch Seite~\pageref{userstoryDefinitions})}
		\item[Problem space]{In \cdar\ abgelegtes Wissen, welches später über einen Solution space hilft Projekte umzusetzen. (siehe auch Kapitel~\ref{userstoryDefinitions})}
		\item[\ppt]{Ein Tool, welches Tasks verwaltet. Es wird zum Planen und Durchführen von Projekten verwendet. (siehe auch Kapitel~\ref{userstoryDefinitions})}
		\item[Solution space]{Kopie eines Problem spaces um konkret Entscheide für ein Projekt zu erstellen. (siehe auch Kapitel~\ref{userstoryDefinitions})}
		\item[Tasks]{Aufgabe, welche üblicherweise in einem \ppt\ abgelegt ist. (siehe auch Kapitel~\ref{userstoryDefinitions})}
		\item[Wissensbaum]{siehe Problem Space}
		\item[Wissenskonsument]{Person, die \cdar\ und \eeppi\ für die Umsetzung von einem Projekt verwendet. (siehe auch Kapitel~\ref{userstoryDefinitions})}
		\item[Wissensproduzent]{Person, die Wissen im \cdar/\eeppi\ erfasst. (siehe auch Kapitel~\ref{userstoryDefinitions})}
		\item[MVW/MV*] Mode View Whatever: Fasst die Patterns MVC - Model View Controller, MVP - Model View Presenter, MVVM - Model View Viewmodel und ähnliche Patterns zusammen.
	\end{description}