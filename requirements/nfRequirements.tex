\section{Weitere Anforderungen}

	\subsection{Qualitätsmerkmale}
		Die folgenden Qualitätsmerkmale orientieren sich an ISO/IEC 9126 \cite{wikipedia_iso/iec_2014} und beziehen sich, wenn nichts anderes angegeben ist, auf alle User Stories.

		\subsubsection{Functionality}
		\begin{description}
			\item[Interoperabilität] Die \textit{Schnittstelle der Datenquelle} der Entscheidungen sowie die \textit{Schnittstellen für das \ppt} sollen konfigurierbar gestaltet sein, sodass sie für andere Systeme als \cdar\ und Jira/Redmine umkonfiguriert werden können. 
			Für \textit{Software eigene Schnittstellen} sollen Aufrufparameter, Rückgabewerte und jeweils mindestens zwei Beispielaufrufe und Beispielantwortdatensätze dokumentiert werden, sodass zukünftige Projekte mit \eeppi\ interagieren können.
			\item[Sicherheit] Im System gespeicherte Daten sollen vor Zugriff Dritter geschützt werden.
			Es soll nicht möglich sein, Informationen über Entscheidungen, Tasks oder Benutzer auszulesen oder zu verändern, 
			ohne authentisiert zu sein.
			\item[Konformität] Zur Umsetzung der Schnittstellen und des Mapping Konzeptes sollen die folgenden existierende Formate und Protokolle eingesetzt werden: HTTP, JSON, REST.
			\item[Kompatibilität] Die Clientapplikation soll mit modernen Browsern (Firefox/Chrome/Internet Explorer/Safari, jeweils neuste zwei Versionen) kompatibel sein. Ältere Versionen sollen nicht unterstützt werden.
		\end{description}
		
		
		\subsubsection{Zuverlässigkeit}
		\begin{description}
			\item[Robustheit] Das System soll auch im Fehlerfall einen konsistenten Zustand annehmen und den Benutzer in Form einer lesbaren Fehlermeldung über das Problem informieren sowie Anleitung zum Beheben des Fehlers geben.
		\end{description}
		

		\subsubsection{Usability}
		\begin{description}
			\item[Verständlichkeit] Ein Benutzer soll anhand der Dokumentation selbstständig die Mapping Konzepte erlernen und erstellen können, um damit Tasks generieren zu können. Das selbstständige Einarbeiten soll nicht länger als einen halben Arbeitstag dauern.
			\item[Task Template Mapping] Die Darstellung des Task Template Mappings soll mit bis zu 50 gemappten Task Templates gleich flüssig und angenehm bedienbar sein wie mit 5 gemappten Task Templates.
			\item[Task Template listing] Die Darstellung der Task Templates wird im Durchschnitt 10-30 Einträge enthalten und soll mit bis zu 100 Einträgen flüssig bedienbar sein.
		\end{description}

		
		\subsubsection{Portability}
		\begin{description}
			\item[Anpassbarkeit] Die Software soll keine plattformspezifische Konfiguration oder Modifikation an der Umgebung (Betriebsystem, Virtuellen Maschine, Weberver) erfordern und soll somit mindestens bei Heroku\footnote{Heroku Cloud applications: \url{https://www.heroku.com/}} installiert werden können.
			\item[Installierbarkeit] Die Installation von \eeppi\ soll gut dokumentiert sein, sodass die Installation und Konfiguration der Anwendung nicht mehr als drei Stunden dauert.
		\end{description}