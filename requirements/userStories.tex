\section{User Stories}
	\subsection{Personas}
		\begin{description}
			\item[Olivia Zander]\label{olivia}\ \newline
				\begin{minipage}[t]{0.35\textwidth} 
					\begin{figure}[H]
						\vspace{-0.75cm}
						\includegraphics[trim=0cm 0cm 0cm 0cm, clip=true, width=5cm]{requirements/media/img/oliviaZander.jpg}
						\caption[Symbolbild Persona Olivia Zander\newline 
							\license{CC BY 2.0 \url{https://creativecommons.org/licenses/by/2.0/}  Official GDC \url{https://www.flickr.com/photos/officialgdc/}}
						]
						{\label{Olivia Zander}}
					\end{figure}
				\end{minipage}
				\begin{minipage}[t]{0.55\textwidth}
					52 Jahre alt.
					Olivia ist eine erfahrene \textbf{Softwarearchitekt}in und arbeitet schon viele Jahre auf dem Beruf.
					Aktuell arbeitet sie in einem kleinen Beratungsunternehmen, welches andere Firmen beim Umstrukturieren von Softwareapplikationen unterstützt.
					Vor einiger Zeit hat sie eine Weiterbildung im Bereich Cloud Computing gemacht.
					Seither hat sie selbst Erfahrungen damit sammeln können, nämlich in verschiedene Beratungsprojekten in welchen es darum ging, bestehende Anwendungen in die Cloud zu bringen.
				\end{minipage}
			\item[Thomas Bucher]\label{thomas}\ \newline
				\begin{minipage}[t]{0.35\textwidth} 
					\begin{figure}[H]
						\vspace{-0.75cm}
						\includegraphics[trim=0cm 0cm 0cm 0cm, clip=true, width=5cm]{requirements/media/img/thomasBucher.jpg}
						\caption[Symbolbild Persona Thomas Bucher\newline
							\license{CC BY 2.0 \url{https://creativecommons.org/licenses/by/2.0/} Steve wilson \url{https://www.flickr.com/photos/125303894@N06/}}
						]
						{\label{Thomas Bucher}}
					\end{figure}
				\end{minipage}
				\begin{minipage}[t]{0.55\textwidth}
					29 Jahre alt.
					Thomas hat vor ein paar Jahren in Rapperswil den Informatik-Bachelor abgeschlossen und arbeitet seither in der gleichen Firma wie Olivia.
					Er arbeitet da als \textbf{Projektplaner} und unterstützt in dieser Funktion aktuell eine externe Firma eine Anwendung mit täglich rund 10'000 Benutzern von ihren lokalen Servern in die Cloud zu transferieren.
				\end{minipage}
		\end{description}
		
	\subsection{Definitionen}\label{userstoryDefinitions}
		Folgende Wörter werden in den User Stories verwendet und sind dafür zum genauen Verständnis hier definiert.
		\begin{description}
			\item[Wissensproduzent] Person, die aktiv neue Entscheidungen und formelle Tasks erfasst (als Beispiel kann hier \hyperref[olivia]{Olivia} dienen).
			\item[Wissensbaum] Ein Projekt im CDAR, in welchem ein Wissensproduzent Wissen ablegt.
			\item[Wissenskonsument] Person, die bestehende Entscheidungen benutzt um damit Entscheide zu fällen (als Beispiel kann hier \hyperref[thomas]{Thomas} dienen).
			\item[Entscheidungsprojekt] Ein Projekt im CDAR, aus welchem ein Wissenskonsument Wissen konsumiert.
				Es ist jeweils eine Kopie eines Wissensbaums.
			\item[Administrator] Person, die für die Konfiguration und den Betrieb von \eeppi\ verantwortlich ist.
			\item[Abbildung] Mit einer Abbildung lässt sich ein Datensatz in einen anderen Datensatz umwandeln.
			\item[erstellen aus] Als Grundlage wird ein Datensatz genommen, aus welchem ein neuer Datensatz erstellt wird.
				Anschliessend sind die beiden Datensätze voneinander unabhängig und Änderungen an einem Datensatz beeinflussen den anderen Datensatz nicht.
			\item[Task] Datensatz, welcher eine Aufgabe beschreibt.
			\item[Task-Vorlage] Datensatz, um später daraus Tasks in einem Projektplanungstool zu erstellen.
			\item[Entscheidungs-Task] Task, welcher zum Treffen einer Entscheidung erledigt werden muss.
				Er ist einer Entscheidung angehängt.
			\item[Operativer Task] Task, welcher durch das Treffen einer Option entsteht.
				Er ist dementsprechend einer Option angehängt.
			\item[Entscheidungs-Vorlage] Datensatz, der eine Wahl mit mehreren Optionen darstellt.
				Der Datensatz ist jedoch nur eine Vorlage, die Entscheidung kann nicht getroffen werden
			\item[Entscheidung] Datensatz, der eine Wahl mit mehreren Optionen darstellt.
				Er wird aus einer Entscheidungs-Vorlage erstellt.
				Die Entscheidung kann jetzt getroffen werden.
			\item[Entscheid] Getroffene (entschiedene) Entscheidung.
			\item[Option] Möglichkeit, wie eine Entscheidung entschieden wird.
			\item[entscheiden] Tätigkeit, in welcher für Entscheidungen der Entscheid gefällt wird.
			\item[importieren] Anwenden einer Abbildung zur Aufnahme von Datensätzen in das \eeppi.
			\item[exportieren] Anwenden einer Abbildung zur Ausgabe von Datensätzen aus dem \eeppi.
			\item[Projektplanungstool] Externes Programm, welches Tasks verwaltet und für diese Tasks eine eigene Form erwartet.
			\item[API] Schnittstelle eines Programms (sowohl bei externen, als auch CDAR und \eeppi)
		\end{description}

	\subsection{Übersicht über die User Stories}
	
		\eeppi\ hat eine enge Verbindung zu CDAR und deshalb werden in Abbildung~\ref{fig:UserStoryDiagram} auch in einem ersten Schritt gemeinsam die übergeordneten User Stories beschrieben.
	
		\begin{figure}[H]
			\begin{minipage}[b]{\linewidth}
				\includegraphics[width=\textwidth]{media/diagrams/UserStoryDiagram.png}
				\centering
				\caption{Übergeordnete User Stories (inklusive CDAR)}
				\label{fig:UserStoryDiagram}
			\end{minipage}
		\end{figure}
		
		Dabei repräsentieren die drei Aktore (Wissensproduzent, -konsument und Administrator) Personen wie in Abschnitt~\ref{userstoryDefinitions} beschrieben,
		der Business-Aktor (ganz rechts) repräsentiert ein beliebiges Projektplanungstool und die roten Kästchen referenzieren den dazugehörenden Issue im \eeppi-Projektplanungstool.
		Nachfolgend sind die Erklärungen für die sieben aufgezeigten User Stories.
		Die User Stories sind jeweils im Format nach Mike Cohn\cite{jonathan_rasmusson_agile_2012} geschrieben:
		\begin{quote}
			\textbf{As a} <type of user>,\newline
			\textbf{I want} <to perform some task>\newline
			\textbf{so that I can} <achieve some goal/benefit/value>.
		\end{quote}
		
	\subsubsection{neue Abbildung erfassen}
		\begin{description}
			\item[Jira-Issue:] \url{http://bajira.democ.ch/browse/BA-20}
			\item[Scope:] \eeppi
			\item[Beschreibung:]\ \newline
				Als Administrator\newline
				will ich verschiedene Abbildungen von Task-Vorlagen auf Projektplanungstool-APIs erstellen\newline
				um damit Benutzern den Export von Tasks in ein Projektplanungstool ermöglichen zu können.
		\end{description}

	\subsubsection{neuen Wissensbaum erstellen}
		\begin{description}
			\item[Jira-Issue:] \url{http://bajira.democ.ch/browse/BA-24}
			\item[Scope:] CDAR (nicht \eeppi)
			\item[Beschreibung:]\ \newline
				Als Wissensproduzent\newline
				will ich Entscheidungs-Vorlagen erfassen und untereinander verknüpfen\newline
				um mein Wissen an Wissenskonsumenten weiter zu geben.
		\end{description}

	\subsubsection{neue Task-Vorlagen erfassen}
		\begin{description}
			\item[Jira-Issue:] \url{http://bajira.democ.ch/browse/BA-19}
			\item[Scope:] \eeppi
			\item[Beschreibung:]\ \newline
				Als Wissensproduzent,\newline
				will ich Abbildungen von Entscheidungen auf formelle Tasks erstellen\newline
				um damit aus getroffenen Entscheiden Tasks generieren zu können.
		\end{description}

	\subsubsection{Task-Vorlage zu Entscheidungs-Vorlage hinzufügen}
		\begin{description}
			\item[Jira-Issue:] \url{http://bajira.democ.ch/browse/BA-32}
			\item[Scope:] \eeppi
			\item[Beschreibung:]\ \newline
				Als Wissensproduzent\newline
				will ich eine Entscheidungs-Vorlage auswählen und dieser Task-Vorlagen zuordnen können\newline
				um die Abbildung von Entscheidungs-Vorlagen auf Task-Vorlagen zu realisieren.
		\end{description}

	\subsubsection{neues Entscheidungsprojekt erstellen}
		\begin{description}
			\item[Jira-Issue:] \url{http://bajira.democ.ch/browse/BA-38}
			\item[Scope:] CDAR (nicht \eeppi)
			\item[Beschreibung:]\ \newline
				Als Wissenskonsument\newline
				will ich aus einem Wissensbaum ein neues Entscheidungsprojekt erstellen\newline
				um damit für ein konkretes Projekt Entscheidungen zu treffen.
		\end{description}

	\subsubsection{Entscheid treffen}
		\begin{description}
			\item[Jira-Issue:] \url{http://bajira.democ.ch/browse/BA-23}
			\item[Scope:] CDAR (nicht \eeppi)
			\item[Beschreibung:]\ \newline
				Als Wissenskonsument\newline
				will ich Entscheide für Entscheidungen treffen\newline
				um damit weitere Entscheidungen zu sehen und in meinem Projekt weiter zu kommen.
		\end{description}

	\subsubsection{Tasks exportieren}
		\begin{description}
			\item[Jira-Issue:] \url{http://bajira.democ.ch/browse/BA-25}
			\item[Scope:] \eeppi
			\item[Beschreibung:]\ \newline
				Als Wissenskonsument\newline
				will ich Task-Vorlagen (Operative Tasks und/oder Entscheidungstasks) in ein Projektplanungstool exportieren\newline
				um sie darin verwalten zu können.
		\end{description}

	
