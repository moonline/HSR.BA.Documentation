\section{User Stories}
	\subsection{Personas}
		\begin{description}
			\item[Olivia Zander]\label{olivia} 52 Jahre alt.
				Olivia ist eine erfahrene \textbf{Softwarearchitekt}in und arbeitet schon viele Jahre auf dem Beruf.
				Aktuell arbeitet sie in einem kleinen Beratungsunternehmen, welches andere Firmen beim Umstrukturieren von Softwareapplikationen unterstützt.
				Vor einiger Zeit hat sie eine Weiterbildung im Bereich Cloud Computing gemacht.
				Seither hat sie selbst Erfahrungen damit sammeln können, nämlich in verschiedene Beratungsprojekten in welchen es darum ging, bestehende Anwendungen in die Cloud zu bringen.
			\item[Thomas Bucher]\label{thomas} 29 Jahre alt.
				Thomas hat vor ein paar Jahren in Rapperswil den Informatik-Bachelor abgeschlossen und arbeitet seither in der gleichen Firma wie Olivia.
				Er arbeitet da als \textbf{Projektplaner} und unterstützt in dieser Funktion aktuell eine externe Firma eine Anwendung mit täglich rund 10'000 Benutzern von ihren lokalen Servern in die Cloud zu transferieren.
		\end{description}
		
	\subsection{Definitionen}\label{userstoryDefinitions}
		Folgende Wörter werden in den User Stories verwendet und sind dafür zum genauen Verständnis hier definiert.
		\begin{description}
			\item[Wissensproduzent] Person, die aktiv neue Entscheidungen und formelle Tasks erfasst (als Beispiel kann hier \hyperref[olivia]{Olivia} dienen).
			\item[Wissenskonsument] Person, die bestehende Entscheidungen benutzt um damit Entscheide zu fällen (als Beispiel kann hier \hyperref[thomas]{Thomas} dienen).
			\item[Administrator] Person, die für die Konfiguration und den Betrieb von EEPPI verantwortlich ist.
			\item[Abbildung] Mit einer Abbildung lässt sich ein Datensatz in einen anderen Datensatz umwandeln.
			\item[Task] Datensatz, welcher eine Aufgabe beschreibt.
			\item[Formeller Task] Task in allgemein gültiger Form (in von Projektplanungstools unabhängiger Form).
			\item[Task Vorlage] Datensatz, um später daraus formelle Tasks zu erstellen.
			\item[Entscheidungstask] Task, welcher die Aufgabe des Entscheidens für eine Entscheidung beschreibt.
			\item[Entscheidung] Datensatz, der eine Wahl mit mehreren Optionen darstellt.
			\item[Entscheid] Entscheidung mit gewählter Option.
			\item[offener Entscheid] Entscheidung mit noch nicht gewählter Option, aber eine Option soll demnächst gewählt werden.
			\item[entscheiden] Tätigkeit, in welcher für Entscheidungen der Entscheid gefällt wird.
			\item[importieren] Anwenden einer Abbildung zur Aufnahme von Datensätzen in das EEPPI.
			\item[exportieren] Anwenden einer Abbildung zur Ausgabe von Datensätzen aus dem EEPPI.
			\item[Projektplanungstool] Externes Programm, welches Tasks verwaltet und für diese Tasks eine eigene Form erwartet.
			\item[API] Schnittstelle eines Programms (sowohl bei externen, als auch CDAR und EEPPI)
		\end{description}

	\subsection{Übersicht über die User Stories}
	
	TODO: Bild 'Uebersicht User Stories.asta' hier integrieren
		
	\subsubsection{TODO alle Stories hier auflisten}

	\url{http://bajira.democ.ch/issues/?jql=issuetype\%20\%3D\%20Story\%20ORDER\%20BY\%20id}
