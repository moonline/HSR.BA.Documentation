\chapter{Anforderungen}
\section{Allgemeine Beschreibung}

\subsection{Produktfunktion}
Softwarearchitektinnen und Softwarearchitekten\footnote{Einfachheitshalber wird zukünftig nur Softwarearchitekten geschrieben, es werden aber natürlich auch Softwarearchitektinnen gemeint. Das Selbe gilt auch für Projektplaner, womit auch Projektplanerinnen gemeint sind.} treffen und dokumentieren mit CDAR Architekturentscheidungen.
"`EEPPI"' soll es ihnen ermöglichen, ein Mapping zwischen Architekturentscheidungen und Projekttasks zu erstellen.
Ebenfalls soll es möglich sein, ein Mapping von Taskeigenschaften auf Felder einer API einer Projektplanungssoftware zu erstellen.
Über diese beiden Mappings soll dem Architekten ermöglicht werden, aus Architekturentscheidungen Tasks in der Projektplanungssoftware zu erstellen.

\subsection{Benutzer-Charakteristik}
Zielgruppe des "`EEPPI"' sind Softwarearchitekten und Projektplaner.

\subsection{Einschränkungen}
Voraussetzung für die Nutzung von "`EEPPI"' ist gundsätzliches Wissen über Softwarearchitektur, Architekturentscheidungen, Projektplanung sowie über die grundlegende Funktion eines Projektplanungstools.

