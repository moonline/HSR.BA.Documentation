\chapter{Anforderungen}
\section{Allgemeine Beschreibung}

\subsection{Produktfunktion}
Softwarearchitekten\footnote{Einfachheitshalber wird jeweils auf die weibliche Form verzichtet, gemeint sind natürlich auch Softwarearchitektinnen. Das Selbe gilt auch für Projektplaner, womit auch Projektplanerinnen gemeint sind.} treffen und dokumentieren mit CDAR Architekturentscheidungen.
"`EEPPI"' soll es ihnen ermöglichen, ein Mapping zwischen Architekturentscheidungen und Projekttasks zu erstellen.
Ebenfalls soll es möglich sein, ein Mapping von Taskeigenschaften auf Felder einer API einer Projektplanungssoftware zu erstellen.
Über diese beiden Mappings soll dem Architekten ermöglicht werden, aus Architekturentscheidungen Tasks in der Projektplanungssoftware zu erstellen.

\subsection{Benutzer-Charakteristik}
Zielgruppe des "`EEPPI"' sind primär Softwarearchitekten und Projektplaner. "`EEPPI"' eignet sich jedoch grundsätzlich für Personen, die in einem Projekt tätig sind und das CDAR in Kombination mit einem Projektplanungstool einsetzen.

\subsection{Einschränkungen}
Voraussetzung für die Nutzung von "`EEPPI"' ist gundsätzliches Wissen über Softwarearchitektur, Architekturentscheidungen, Projektplanung sowie über die grundlegende Funktion eines Projektplanungstools.

