\chapter{Anforderungen}

Diese Kapitel beschreibt die Anforderungen an \eeppi\ in Form von User Stories und nicht-funktionalen Anforderungen.

\section{Allgemeine Beschreibung}

Softwarearchitekten\footnote{Einfachheitshalber wird jeweils auf die weibliche Form verzichtet, gemeint sind natürlich auch Softwarearchitektinnen. Das Selbe gilt auch für alle weiteren geschlechtsspezifischen Formulierungen.} treffen und dokumentieren mit DKS\footnote{Decision Knowledge System: Entscheidungswissensverwaltung, siehe auch Abschnitt \ref{userstoryDefinitions}.}-Tools wie \cdar\ Architekturentscheidungen.
\eeppi\ soll es ihnen ermöglichen, ein Mapping zwischen Architekturentscheidungen und Projekttasks zu erstellen.
Ebenfalls soll es möglich sein, ein Mapping von Taskeigenschaften auf Felder einer API eines \ppt s zu erstellen.
Über diese beiden Mappings soll dem Architekten ermöglicht werden, aus Architekturentscheidungen Tasks in ein \ppt\ zu erstellen.

\subsection{Benutzer-Charakteristik}
Zielgruppe des \eeppi\ sind primär Softwarearchitekten und Projektplaner. \eeppi\ eignet sich jedoch grundsätzlich für Personen, die in einem Projekt tätig sind und ein \dks\ in Kombination mit einem \ppt\ einsetzen wollen.

\subsection{Einschränkungen}
Voraussetzung für die Nutzung von \eeppi\ ist grundsätzliches Wissen über Softwarearchitektur, Architekturentscheidungen, Projektplanung sowie über die grundlegende Funktion eines \ppt s.

Wissen, wie es in Software Engineering Ausbildungen vermittelt wird,
oder in Knowledge Hubs\footnote{Beispielsweise das Knowledge Hub des IFS Institut für Software der HSR: \url{http://www.ifs.hsr.ch/Architectural-Knowledge-Hubs.13193.0.html}} zu finden ist.