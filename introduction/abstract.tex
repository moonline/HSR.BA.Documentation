\chapter{Abstract}
	% Aufgabe, Problem, Lösung, Reflexion jeweils ~2 Sätze

	% Aufgabe
	Die vorliegende Arbeit befasst sich mit der Frage,
	ob und wie sich Aufgaben aus Architekturentscheidungen eines Softwareprojektes teilautomatisiert erzeugen lassen.

	% Problem 
	Jedes Projekt erfordert das Treffen von Entscheidungen wobei aus einer bestimmten Entscheidung häufig ähnliche Aufgaben resultieren.
	Sowohl auf Seite der Entscheidungsverwaltung wie auf Seiten der Projektplanungstools existieren bereits verschiedene Tools.
	Ziel von \eeppi\ ist es, diese Lücke zu schliessen und eine Brücke zwischen Entscheidungsmanagement und Projektplanung zu bilden.

	% Lösung
	Entstanden ist eine Webapplikation, die mögliche Entscheidungen aus einem angebundenen Wissensverwaltungssystem bezieht
	und dem Benutzer mit einem Metamapping ermöglicht,
	die daraus für konkrete Projekte entstehenden Entscheidungen mit eigenen Aufgaben zu verknüpfen.
	Die Applikation bietet ihm auch eine weitgehende Konfiguration der angebundenen Systeme und der Aufgabenerzeugung. 
	Dazu wurde ein Templatingmechanismus entwickelt,
	der dem Benutzer das Verwenden von eigenen Verarbeitungsfunktionen, sogenannten Processors, ermöglicht.

	% Reflexion
	\eeppi\ zeigt, was kommerzielle Produkte in diesem Bereich anbieten könnten,
	aber auch die Design-Herausforderungen einer solchen Software: 
	Hohe Flexibilität und Konfigurierbarkeit.
	\eeppi\ legt somit einen wichtigen Meilenstein im Forschungsbereich des interdisziplinären Entscheidungs- und Projektmanagements und
	zeigt auf, wohin die Reise zukünftiger Tools führen könnte.
