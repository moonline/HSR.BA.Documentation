\chapter*{Abstract}
	\documentPartEntry{Abstract}
	
	% Aufgabe, Problem, Lösung, Reflexion jeweils ~2 Sätze

	% Aufgabe
	"'Entwurfsentscheidungen als Projektplanungsinstrument"', kurz EEPPI. 
	Diesem Thema widmet sich die vorliegende Arbeit und befasst sich mit der Frage, 
	ob sich aus Projektentscheidungen Aufgaben ableiten lassen.
	Dabei wird untersucht, ob sich dieser Prozess automatisieren lässt.

	% Problem 
	Jedes Projekt erfordert das Treffen von Entscheidungen wobei aus einer bestimmten Entscheidung häufig ähnliche Aufgaben resultieren.
	Sowohl auf Seite der Entscheidungsverwaltung wie auf Seiten der Projektplanung existieren bereits verschiedene Werkzeuge.
	Ziel von \eeppi\ ist es, eine Brücke zwischen Entscheidungsmanagement und Projektplanung zu bilden.

	% Lösung
	Im Rahmen der Arbeit wurde eine Webapplikation entwickelt, 
	die mögliche Entscheidungen aus einem angebundenen Wissensverwaltungssystem bezieht
	und dem Benutzer mit einem Metamapping ermöglicht,
	Projektentscheidungen mit eigenen Aufgaben zu verknüpfen.
	Die Applikation bietet ihm auch eine weitgehende Konfiguration der angebundenen Systeme und der Aufgabenerzeugung. 
	Ein dazu entwickelter Templatingmechanismus ermöglicht dem Benutzer eigene Verarbeitungsfunktionen, sogenannten Processors, zu verwenden.

	% Reflexion
	\eeppi\ zeigt, was kommerzielle Produkte in diesem Bereich anbieten könnten,
	aber auch die Design-Herausforderungen einer solchen Software: 
	Hohe Flexibilität und Konfigurierbarkeit.
	\eeppi\ legt somit einen wichtigen Meilenstein im Forschungsbereich des interdisziplinären Entscheidungs- und Projektmanagements und
	zeigt den möglichen Weg zukünftiger Tools auf.
