\chapter{Abstract}
	%Projektbeschrieb, Erreichtes (Fakten), Details, Fazit (Wertend)

	Die Bachelorarbeit ``Entwurfsentscheidungen als Projektplanungsinstrument'' - kurz \eeppi\ befasst sich mit der Frage,
	ob und wie sich teilautomatisiert Aufgaben erzeugen lassen aus Architekturentscheidungen eines Softwareprojektes.

	Jedes Softwareprojekt erfordert das Treffen von Architekturentscheidungen.
	Sowohl aus noch offenen wie bereits getroffenen Entscheidungen lassen sich Aufgaben ableiten.
	So führt die die offene Entscheidung ``Welche Art von Session State soll verwendet werden?'' zum Beispiel zu den Aufgaben
	``Session State evaluieren'' und ``Prototyp umsetzen''.
	Wurde die Entscheidung getroffen wobei die Option ``Database Session State'' ausgewählt wurde,
	so resultieren aus dieser Entscheidung zum Beispiel die Aufgaben ``Datenbank installieren'' und
	``Session Persistenz implementieren''.

	Sowohl auf Seite der Entscheidungsverwaltung wie auf Seiten der Projektplanungstools existieren bereits verschiedene Tools.
	Ziel von \eeppi\ ist es, diese Lücke zu schliessen und eine Brücke zwischen Entscheidungsmanagement und Projektplanung zu schlagen.

	% was wurde erreicht?
	Entstanden ist eine moderne Webapplikation, die Entscheidungen aus einem angebundenen Wissensverwaltungssystem bezieht,
	dem Benutzer ermöglicht Vorlagen für Aufgaben zu erfassen und diese mit Entscheidngen zu verknüpfen.
	Anhand dieser Verknüpfungen kann der Benutzer Aufgaben erzeugen und in ein Projektplanungstool übertragen lassen.
	Ein Administrationsbereich ermöglicht dem Anwender die Konfiguration der angebundenen Systeme,
	der zur Verfügung stehende Eigenschaften für Vorlagen, der Schnittstellenaufrufe sowie der Verarbeitungsmechanismen der Vorlagen für Aufgaben.

	% Details
	\eeppi\ setzt sich aus einer Server- und einer Clientkomponente zusammen, die auf modernen MVC Frameworks basieren.
	Sowohl die Server und die Clientkomponente wie der Server und die umliegenden Systeme kommunizieren lediglich über Schnittstellen miteinander,
	was die Komponenten unabhängig von Einander und austauschbar macht.
	Technologisch kommen auf Server- wie Client typisierte Sprachen zum Einsatz, was sich positiv auf die Codequalität auswirkt.

	% Fazit
	Mit \eeppi\ ist es möglich, teilautomatisiert Aufgaben aus Projektentscheidungen zu erzeugen.
	\eeppi\ zeigt, was kommerzielle Produkte in diesem Bereich anbieten könnten, wenn sie denn existeren würden.
	Allerdings zeigt \eeppi\ auch die Knackpunkte einer solchen Software:
	Sowohl die Verknüpfung von Entscheidungen und Aufgaben wie auch die Anbindung an die umliegenden Systeme müssen sehr flexibel gestaltet sein.
	Mit \eeppi\ ist dies gelungen, doch es gibt noch viele Erweiterungsmöglichkeiten.
	\eeppi\ legt somit einen wichtigen Meilenstein im Forschungsbereich des interdisziplinären Entscheidungs- und Projektmanagement und
	zeigt auf, wohin die Reise zukünftiger Tools führen könnte.
