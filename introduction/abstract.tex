\chapter{Abstract}
	%Projektbeschrieb, Erreichtes (Fakten), Details, Fazit (Wertend)

	\section{Ausgangslage}

	Die vorliegende Arbeit befasst sich mit der Frage,
	ob und wie sich teilautomatisiert Aufgaben aus Architekturentscheidungen eines Softwareprojektes erzeugen lassen.

	Jedes Softwareprojekt erfordert das Treffen von Entscheidungen.
	So führt die Entscheidung \flqq Welche Art von Session State soll verwendet werden?\frqq\ zum Beispiel zu den Aufgaben
	\flqq Session State evaluieren\frqq\ und \flqq Prototyp umsetzen\frqq.
	Wird bei dieser Entscheidung die Option \flqq Database Session State\frqq\ ausgewählt,
	so resultieren aus diesem Entscheid beispielsweise die Aufgaben \flqq Datenbank installieren\frqq\ und
	\flqq Session Persistenz implementieren\frqq.

	Sowohl auf Seite der Entscheidungsverwaltung wie auf Seiten der Projektplanungstools existieren bereits verschiedene Tools.
	Ziel von \eeppi\ ist es, diese Lücke zu schliessen und eine Brücke zwischen Entscheidungsmanagement und Projektplanung zu bilden.


	\section{Vorgehen}

	Aufbauend auf den Schnittstellen von Wissensverwaltungssystemen und Projektmanagementtools wurde vom Entwicklungsteam eine Applikation entworfen,
	die einerseits eine flexible Konfiguration der Schnittstellen ermöglicht
	und andererseits dem Benutzer ein Werkzeug für das Verknüpfen von Entscheidungen mit Aufgabenvorlagen zur Verfügung stellen soll.

	Mit einem Prototypen wurde die Machbarkeit dessen überprüft
	und anschliessend im Rahmen mehrer Iterationen eine auf Webtechnologie basierende Applikation gebaut.

	Zusammen mit dem Ansprechspartner der Kundengruppe wurden Usability- und Workflowtests durchgeführt, um die Benutzeroberfläche zu testen
	und den Datenfluss vom Entscheidungsverwaltungssystem bis ins Projektplanungstool zu validieren.

	Abschliessend folgte zur Stabilisierung eine Überarbeitungsphase zur Steigerung der Code Qualität.


	\section{Ergebnis}

	Mit \eeppi\ ist ein Tool entstanden, das Aufgaben aus Projektentscheidungen teilautomatisiert erzeugen kann.
	Das Metamapping - Verknüpfung von Entscheidungen und Aufgabenvorlagen - ermöglicht dem Benutzer das anpassen zu erzeugender Aufgaben.
	Dazu wird dem Benutzer eine breite Palette an Konfigurationsmöglichkeiten geboten.
	Beispielsweise kann der Benutzer selbst steuern,
	wie die zu generierenden Aufgaben aufgebaut werden. Dazu wurde ein Templatingmechanismus entwickelt,
	der dem Benutzer das Verwenden von eigenen Verarbeitungsfunktionen, sogenannten Processors, ermöglicht.

	\eeppi\ zeigt, was kommerzielle Produkte in diesem Bereich anbieten könnten.
	Allerdings zeigt \eeppi\ auch die Design-Herausforderungen einer solchen Software:
	Sowohl die Verknüpfung von Entscheidungen und Aufgaben wie auch die Anbindung an die umliegenden Systeme müssen sehr flexibel gestaltet sein.
	Mit \eeppi\ ist dies gelungen, doch es gibt noch viele Erweiterungsmöglichkeiten.
	\eeppi\ legt somit einen wichtigen Meilenstein im Forschungsbereich des interdisziplinären Entscheidungs- und Projektmanagements und
	zeigt auf, wohin die Reise zukünftiger Tools führen könnte.
