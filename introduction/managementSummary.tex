\chapter*{Management Summary}
	\documentPartEntry{Management Summary}
	\captionsetup[figure]{labelformat=empty} % disable figure numbering

	
	
	\section*{Ausgangslage}
	
	"'Entwurfsentscheidungen als Projektplanungsinstrument"', kurz EEPPI. 
	Diesem Thema widmet sich die vorliegende Arbeit und befasst sich mit der Frage, 
	ob sich aus Projektentscheidungen Aufgaben ableiten lassen.
	Dabei wird untersucht, ob sich dieser Prozess automatisieren lässt.
	
	Jedes Projekt erfordert das Treffen von Entscheidungen.
	So führt die Entscheidung "<Welche Art Session State soll verwendet werden?"> zum Beispiel zu den Aufgaben
	"<Session State evaluieren"> und "<Prototyp umsetzen">.
	Wird bei dieser Entscheidung die Option "<Database Session State"> ausgewählt,
	so resultieren aus diesem Entscheid beispielsweise die Aufgaben "<Datenbank installieren"> und
		"<Session Persistenz implementieren">.
	
	Sowohl auf Seite der Entscheidungsverwaltung wie auf Seiten der Projektplanung existieren bereits verschiedene Werkzeuge.
	Ziel von \eeppi\ ist es, eine Brücke zwischen Entscheidungsmanagement und Projektplanung zu bilden.
	
	\begin{figure}[H]
		\includegraphics[width=\textwidth]{introduction/img/eeppiVision.png}
		\centering
		\caption{EEPPI bildet eine Brücke zwischen Entscheidungsmanagement- und Projektplanung}
		\label{fig:eeppiBridgeBetweenDecisionsAndTasks}
	\end{figure}
	
	
	\section*{Vorgehen}
	
	Aufbauend auf den Schnittstellen von Wissensverwaltungssystemen und Projektplanungstools wurde eine Applikation entworfen,
	die eine flexible Konfiguration der Schnittstellen ermöglicht.
	Benutzer sollen Aufgabenvorlagen erstellen, diese mit Entscheidungen verknüpfen und in ein Projektplanungstool übertragen können.
	
	Mittels Prototyp wurde die Machbarkeit dessen überprüft
	und anschliessend im Rahmen mehrerer Iterationen eine Webapplikation entwickelt.
	Zusammen mit dem Ansprechpartner der Kundengruppe wurden Usability- und Workflowtests durchgeführt, um Benutzeroberfläche
	und Datenfluss vom Entscheidungsverwaltungssystem bis ins Projektplanungstool zu validieren.
	Abschliessend folgte zur Stabilisierung eine Überarbeitungsphase.
	
	
	\section*{Ergebnis}
		
	Im Rahmen der Arbeit wurde eine Webapplikation entwickelt, 
	die mögliche Entscheidungen aus einem angebundenen Wissensverwaltungssystem bezieht
	und dem Benutzer mit einem Metamapping ermöglicht,
	Projektentscheidungen mit eigenen Aufgaben zu verknüpfen.	
	
	\begin{figure}[H]
		\includegraphics[width=\textwidth]{introduction/img/eeppiDecisionsAndTaskTemplates.png}
		\centering
		\caption{Metamapping: Verknüpfung von Entscheidungen und Aufgabenvorlagen}
		\label{fig:metamapping}
	\end{figure}	
	
	Über einen  Administrationsbereich konfiguriert der Benutzer die Applikation nach seinen Bedürfnissen.
	Beispielsweise kann der Benutzer selbst den Aufbau der zu generierenden Aufgaben steuern. 
	Ein dazu entwickelter Templatingmechanismus ermöglicht ihm eigene Verarbeitungsfunktionen, sogenannten Processors, zu verwenden.
	
	\begin{figure}[H]
		\includegraphics[width=\textwidth]{introduction/img/simpleProcessWorkflow.jpg}
		\centering
		\caption{Übertragung: Ausführen von Processors und Übertragen der Aufgaben}
		\label{fig:metamapping}
	\end{figure}
	
	\eeppi\ zeigt, was kommerzielle Produkte in diesem Bereich anbieten könnten,
	aber auch die Design-Herausforderungen einer solchen Software: 
	Sowohl die Verknüpfung von Entscheidungen und Aufgaben wie auch die Anbindung an die umliegenden Systeme müssen sehr flexibel gestaltet sein.
	Mit EEPPI ist dies gelungen, doch es gibt noch viele Erweiterungsmöglichkeiten.
	\eeppi\ legt somit einen wichtigen Meilenstein im Forschungsbereich des interdisziplinären Entscheidungs- und Projektmanagements und
	zeigt den möglichen Weg zukünftiger Tools auf.
	
	\captionsetup[figure]{labelformat=default} % reenable figure numbering