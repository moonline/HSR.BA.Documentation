\chapter{Einleitung}
	% Beschrieb, Ziele, Abgrenzung, Optionale Erweiterungen, Mehrwert gegenüber bestehenden Lösungen, Motivation

	"<Lassen sich aus Projektentscheidungen Aufgaben ableiten?">
	Mit dieser Frage befasst sich die vorliegende Bachelorarbeit \eeppi.
		
	Jedes Projekt erfordert das Treffen verschiedenster Entscheidungen.
	Sowohl das Treffen einer Entscheidung wie die Entscheidung zugunsten einer Wahlmöglichkeit implizieren Aufgaben in einem Projekt.
	So führt die Entscheidung "<Welche Art Session State soll verwendet werden?"> zum Beispiel zu den Aufgaben
	"<Session State evaluieren"> und "<Prototyp umsetzen">.		
	Wird bei dieser Entscheidung die Option "<Database Session State"> ausgewählt,
	so resultieren aus diesem Entscheid beispielsweise die Aufgaben "<Datenbank installieren"> und
		"<Session Persistenz implementieren">.
	
	Sowohl auf Seite der Entscheidungsverwaltung wie auf Seiten der \ppt s existieren bereits verschiedene Tools.
	Ziel von EEPPI ist es, eine Brücke zwischen Entscheidungsmanagement und Projektplanung zu bilden.
	
	Aufgaben unterscheiden sich abhängig von Kontext, Unternehmung, Zeitplanung und Detaillierungsgrad der Entscheidung.
	Daraus schliessen wir, das eine Applikation, die das Erzeugen von Aufgaben aus Projektentscheidungen ermöglich will, eine hohe Flexibilität und Konfigurierbarkeit aufweisen muss.
	
