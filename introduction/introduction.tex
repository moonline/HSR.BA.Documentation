\chapter{Einleitung}
	% Beschrieb
	"<Lassen sich aus Projektentscheidungen Aufgaben ableiten?">
	Mit dieser Frage befasst sich die vorliegende Bachelorarbeit \eeppi.
		
	Sowohl das Treffen einer Entscheidung wie die Entscheidung zugunsten einer Wahlmöglichkeit implizieren Aufgaben in einem Projekt.
	So führt die Entscheidung "<Welche Art Session State soll verwendet werden?"> zum Beispiel zu den Aufgaben
	"<Session State evaluieren"> und "<Prototyp umsetzen">.		
	Wird bei dieser Entscheidung die Option "<Database Session State"> ausgewählt,
	so resultieren aus diesem Entscheid beispielsweise die Aufgaben "<Datenbank installieren"> und "<Session Persistenz implementieren">.
	
	
	\section{Ziele}
	Es soll herausgefunden werden, ob sich ein Metamapping formulieren lässt,
	das Entscheidungen und Aufgaben verknüpft.
	Weiter soll ermittelt werden, ob dieser Prozess automatisierbar und in einer Software umsetzbar ist.
	Dazu soll eine Webapplikation entwickelt werden, die Entscheidungen aus einem \dks\ bezieht, 
	dem Benutzer für das Metamapping ein Werkzeug anbietet und Aufgaben in ein \ppt\ exportiert.
	
	
	\section{Abgrenzung}
	Die bestehenden Systeme sollen dabei nicht verändert werden. So sollen keine Informationen aus dem \ppt\ zurück ins \dks\ fliessen.
	Auch die Zuständigkeitsbereiche der einzelnen Tools sollen klar getrennt bleiben. So soll \eeppi\ keine Funktionalität zum modellieren von Entscheidungen enthalten und auch keine Projektplanungsfunktionalität integrieren.
	Zudem soll der Fokus der Applikation im Bereich der Konzepte und deren Umsetzung liegen. Die Implementierung von Nebenfunktionen wie z.B. Rechte- und Rollenkonzepten besitzt entsprechend eine tiefere Priorität und soll daher schlank umgesetzt werden.
	
	
	\section{Optionale Erweiterungen}
	Optional sind beispielsweise eine Strukturierung von Aufgaben oder Funktionalität zum Import und Export von Aufgaben und Verknüpfungen denkbar.	
	Ebenfalls denkbar ist die Anbindung mehrerer \dks e und eine Verschmelzung deren Entscheidungen oder Mandantenfähigkeit der Applikation.
	
	
	\section{Mehrwert gegenüber bestehenden Lösungen}
	Lösungen zur Verknüpfung von Entscheidungsmanagement und Projektplanung sind ein aktuelles Forschungsthema, 
	entsprechend gibt es noch keine uns bekannten Produkte auf dem Markt.
	 		
		
	%\section{Motivation}
	