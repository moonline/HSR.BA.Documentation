%Pakete;
%A4, Report, 12pt
\documentclass[ngerman,a4paper,12pt]{scrreprt}
\usepackage[a4paper, right=20mm, left=20mm,top=30mm, bottom=30mm, marginparsep=5mm, marginparwidth=5mm, headheight=7mm, headsep=15mm,footskip=15mm]{geometry}

%Papierausrichtungen
\usepackage{pdflscape}
\usepackage{lscape}

%pdf include
\usepackage{pdfpages}
\usepackage[backend=bibtex,style=numeric]{biblatex}
\addbibresource{EEPPI.bib}

%Deutsche Umlaute, Schriftart, Deutsche Bezeichnungen
\usepackage[utf8]{inputenc}
\usepackage[T1]{fontenc}
\usepackage[ngerman]{babel}

%quellcode, csv
\usepackage{listings}
%\usepackage{styles/csvsimple}

%tabellen
\usepackage{tabularx}
\usepackage{longtable,tabu}
\usepackage{array}
\usepackage{booktabs}

%listen und aufzählungen
\usepackage{paralist}

%farben
%\usepackage[svgnames,table,hyperref]{xcolor}

%font
\usepackage{helvet}
\renewcommand{\familydefault}{\sfdefault}

%Abkürzungsverzeichnisse
\usepackage[printonlyused]{acronym}

%Bilder
\usepackage{graphicx} %Bilder
\usepackage{float}	  %"Floating" Objects, Bilder, Tabellen...
\usepackage{wrapfig}

%Kopf- /Fusszeile
\usepackage{fancyhdr}
\usepackage{lastpage}

%Eigene Templates
% (IBM UMF Template for Decision Log)
%
% @var #1 decisionHeader
% @var #2 decisionContent
%
% @example 
%	\decision{
%		\decisionHeader{1}{Server architecture}{Architecture}{Server Patter}
%	}{
%		\decisionContent{Small server}{What server architecture is best for us?}{}{}{}{}{}{}{}
%	}
\newcommand{\decision}[2]{
	\gdef\decisionId{Test}%
	\begin{figure}[H]%
		{\footnotesize%
		\vspace{0.5cm}%
		\noindent%
		\begin{tabularx}{\linewidth}{|p{0.15\textwidth}|X|l|l|}%
			\hline%
			#1%
			\hline%
			#2%
			\hline%
		\end{tabularx}%
		\newline%
		}%
		\caption{\decisionName}%
		\label{\decisionId}%
		\vspace{0.5cm}%
	\end{figure}%
}

% @var #1 id
% @var #2 name
% @var #3 thema
% @var #4 subject area
\newcommand{\decisionHeader}[4]{%
	% Define global variable with id to use outside for figure label
	\gdef\decisionId{#1} %
	\gdef\decisionName{#2} %
	\textbf{Themengebiet}	& #4	& \textbf{Thema}	& #3 \\%
	\hline%
	\textbf{Name}		& #2	& \textbf{ID}		& #1 \\%
}

\newcommand{\tableCellPage}[1]{%
	\begin{minipage}[b]{0.80\linewidth}%
		\vspace{0.1cm}%
		#1%
	\end{minipage}%
}

% @var #1 decision made
% @var #2 issue/problem
% @var #3 assumptions
% @var #4 motivation
% @var #5 alternatives
% @var #5 justification
% @var #6 implications
% @var #7 derived requirements
% @var #8 related decisions
\newcommand{\decisionContent}[9]{%
	\textbf{Getroffene Entscheidung} %
		& \multicolumn{3}{l|}{\tableCellPage{#1}} \\%
	\hline%
	\textbf{Problemstellung}%
		& \multicolumn{3}{l|}{\tableCellPage{#2}} \\%
	\hline%
	\textbf{Voraussetzung}%
		& \multicolumn{3}{l|}{\tableCellPage{#3}} \\%
	\hline%
	\textbf{Motivation}%
		& \multicolumn{3}{l|}{\tableCellPage{#4}} \\%
	\hline%
	\textbf{Alternativen}%
		& \multicolumn{3}{l|}{\tableCellPage{#5}} \\%
	\hline%
	\textbf{Begründung}%
		& \multicolumn{3}{l|}{\tableCellPage{#6}} \\%
	\hline%
	\textbf{Annahmen}%
		& \multicolumn{3}{l|}{\tableCellPage{#7}} \\%
	\hline%
	\textbf{Abgeleitete Anforderungen}%
		& \multicolumn{3}{l|}{\tableCellPage{#8}} \\%
	\hline%
	\textbf{Verknüpfte Entscheidungen}%
		& \multicolumn{3}{l|}{\tableCellPage{#9}} \\%
}

% @var #1 decision name
% @var #2 decision id
\newcommand{\decisionRef}[2]{%
	"`#1"' Abb. \ref{#2} S. \pageref{#2}%
}

% inhaltsverzeichnis
\setcounter{secnumdepth}{3}
\setcounter{tocdepth}{3}

\pagestyle{fancy}
	\fancyhf{} %alle Kopf- und Fußzeilenfelder bereinigen
	\fancyhead[L]{Bachelorarbeit} %Kopfzeile links
	\fancyhead[C]{} %Kopfzeile mitte
	\fancyhead[R]{\project} %Kopfzeile rechts
	\renewcommand{\headrulewidth}{0.4pt} %obere Trennlinie
	\fancyfoot[L]{Seite \thepage/\pageref{LastPage}} %Fusszeile links
	\fancyfoot[C]{} %Fusszeile mitte
	\fancyfoot[R]{\today{}} %Fusszeile rechts
	\renewcommand{\footrulewidth}{0.4pt} %untere Trennlinie

%Kopf-/ Fusszeile auf chapter page
\fancypagestyle{plain} {
	\fancyhf{} %alle Kopf- und Fußzeilenfelder bereinigen
	\fancyhead[L]{Bachelorarbeit} %Kopfzeile links
	\fancyhead[C]{} %Kopfzeile mitte
	\fancyhead[R]{\project} %Kopfzeile rechts
	\renewcommand{\headrulewidth}{0.4pt} %obere Trennlinie
	\fancyfoot[L]{Seite \thepage/\pageref{LastPage}} %Fusszeile links
	\fancyfoot[C]{} %Fusszeile mitte
	\fancyfoot[R]{\today{}} %Fusszeile rechts
}

\usepackage{changepage}

% zusätzliches Verzeichnis
\usepackage{tocloft}
%\newcommand{\contentOverviewList}{Inhaltsüberblick}
%\newlistof{contentOverview}{contentOverviewCounter}{\contentOverviewList}
\newcommand{\listdocumentPartsName}{Inhaltsübersicht}
\newlistof{documentPart}{parts.tmp}{\listdocumentPartsName}

\newcommand{\documentPartEntry}[1]{
	\refstepcounter{documentPart}
	\addcontentsline{parts.tmp}{documentPart}
	{\protect\numberline{}#1}\par
}

%links, verlinktes Inhaltsverzeichnis, PDF Inhaltsverzeichnis
\usepackage[bookmarks=true,
bookmarksopen=true,
bookmarksnumbered=true,
breaklinks=true,
colorlinks=true,
linkcolor=black,
anchorcolor=black,
citecolor=black,
filecolor=black,
menucolor=black,
pagecolor=black,
urlcolor=black
]{hyperref} % Paket muss unbedingt als letzes eingebunden werden!

%Dokumenteigenschaften
\providecommand{\project}{Entwurfsentscheidungen als Projektplanungsinstrument}
\author{Laurin Murer, Tobias Blaser}
\providecommand{\room}{1.X}
\providecommand{\teacher}{Prof. Dr. Olaf Zimmermann}
\providecommand{\eeppi}{EE\-PPI}
\providecommand{\cdar}{"CDAR"}
\providecommand{\ppt}{Projektplanungstool}
\providecommand{\smallThird}{0.382}
\providecommand{\largeThird}{0.618}
\title{\documentType \project}
\date{\today{}, Rapperswil}

\newcommand{\license}[1]{\textit{\textbf{Bildlizenz(en)}: #1}}

%Dokumenteigenschaften
\providecommand{\documentType}{Projektdokumentation}
\providecommand{\versionnumber}{0.3}


\begin{document}


%Titel und Inhaltsverzeichnis
% TODO: Replace by a title page like the hsr title page template.
\thispagestyle{empty}
\begin{titlepage}
	\begin{minipage}{0.5\textwidth}
		\begin{flushleft} \large
			\includegraphics[width=0.7\textwidth]{media/img/logoHSR.png}
		\end{flushleft}
	\end{minipage}
	~
	\begin{minipage}{0.5\textwidth}
		\begin{flushright} \large
			\includegraphics[width=0.55\textwidth]{media/img/ifsLogo.png}
		\end{flushright}
	\end{minipage}
	
	\vspace*{2cm}
	\begin{center}
		{\fontsize{50}{40} \selectfont \textbf{EEPPI} \\[10mm]}
	
		\begin{figure}[H]
			\centering
			\includegraphics[scale=0.40]{media/img/eeppiLogo.png}
		\end{figure}		
		\vspace*{0.5cm}	
	
		{\fontsize{28}{40} \selectfont \textbf{\project} \\}
		\vspace{0.25cm}
		{\fontsize{22}{40} \selectfont \textbf{Technischer Bericht} \\[10mm]}
	
		{\fontsize{18}{20} \selectfont 
			Abteilung Informatik\\
			Hochschule für Technik Rapperswil \\
				
			{\fontsize{14}{16} \selectfont Herbstsemester 2014\\}
		}
		
	\end{center}
	
	\vspace*{1.5cm}
	\begin{minipage}[b]{0.4\textwidth}
		\begin{flushleft}
			\begin{tabular}{ll}  
				Autoren: & Laurin Murer, Tobias Blaser \\ 
				Betreuer: & Prof. Dr. Olaf Zimmermann\\ 
				Projektpartner: & IFS, HSR\\
				Experte: & Dr. Gerald Reif, Innovation Process Technology AG\\
				Gegenleser: & Prof. Hans Rudin\\
				Abgabedatum: & 19. Dezember 2014\\
			\end{tabular}
		\end{flushleft}
	\end{minipage}	

\end{titlepage}
\clearpage

\listofdocumentPart
\clearpage


\includepdf[pages=-]{media/documents/Aufgabenstellung.pdf}


% abstract, management summary, vision
\chapter{Abstract}
	% Aufgabe, Problem, Lösung, Reflexion jeweils ~2 Sätze

	% Aufgabe
	Die vorliegende Arbeit befasst sich mit der Frage,
	ob und wie sich teilautomatisiert Aufgaben aus Architekturentscheidungen eines Softwareprojektes erzeugen lassen.

	% Problem 
	Jedes Projekt erfordert das Treffen von Entscheidungen wobei aus einer bestimmten Entscheidung häufig ähnliche Aufgaben resultieren.
	Sowohl auf Seite der Entscheidungsverwaltung wie auf Seiten der Projektplanungstools existieren bereits verschiedene Tools.
	Ziel von \eeppi\ ist es, diese Lücke zu schliessen und eine Brücke zwischen Entscheidungsmanagement und Projektplanung zu bilden.

	% Lösung
	Entstanden ist eine Webapplikation, die Entscheidungen aus einem angebundenen Wissensverwaltungssystem bezieht, 
	dem Benutzer das Verknüpfen von Entscheidungen und Aufgabenvorlagen ermöglicht 
	und dem Benutzer eine weitgehende Konfiguration der angebundenen Systeme und der Aufgabenerzeugung bietet. 
	Dazu wurde ein Templatingmechanismus entwickelt,
	der dem Benutzer das Verwenden von eigenen Verarbeitungsfunktionen, sogenannten Processors, ermöglicht.

	% Reflexion
	\eeppi\ zeigt, was kommerzielle Produkte in diesem Bereich anbieten könnten aber auch die Design-Herausforderungen einer solchen Software: 
	Hohe Flexibilität und Konfigurierbarkeit.
	\eeppi\ legt somit einen wichtigen Meilenstein im Forschungsbereich des interdisziplinären Entscheidungs- und Projektmanagements und
	zeigt auf, wohin die Reise zukünftiger Tools führen könnte.

\chapter*{Management Summary}
		
	\section*{Ausgangslage}
	
	Die vorliegende Arbeit befasst sich mit der Frage,
	ob und wie sich Aufgaben aus Architekturentscheidungen eines Softwareprojektes teilautomatisiert erzeugen lassen.
	
	Jedes Projekt erfordert das Treffen von Entscheidungen.
	So führt die Entscheidung "<Welche Art Session State soll verwendet werden?"> zum Beispiel zu den Aufgaben
	"<Session State evaluieren"> und "<Prototyp umsetzen">.
	Wird bei dieser Entscheidung die Option "<Database Session State"> ausgewählt,
	so resultieren aus diesem Entscheid beispielsweise die Aufgaben "<Datenbank installieren"> und
		"<Session Persistenz implementieren">.
	
	Sowohl auf Seite der Entscheidungsverwaltung wie auf Seiten der Projektplanungstools existieren bereits verschiedene Tools.
	Ziel von EEPPI ist es, diese Lücke zu schliessen und eine Brücke zwischen Entscheidungsmanagement und Projektplanung zu bilden.
	
	\begin{figure}[H]
		\includegraphics[width=\textwidth]{introduction/img/eeppiVision.png}
		\centering
		\caption{EEPPI bildet eine Brücke zwischen Entscheidungs- und Projektmanagement}
		\label{fig:eeppiBridgeBetweenDecisionsAndTasks}
	\end{figure}
	
	
	\section*{Vorgehen}
	
	Aufbauend auf den Schnittstellen von Wissensverwaltungssystemen und Projektplanungstools wurde eine Applikation entworfen,
	die eine flexible Konfiguration der Schnittstellen ermöglicht.
	Benutzer sollen Aufgabenvorlagen erstellen, diese mit Entscheidungen verknüpfen und in ein Projektplanungstool übertragen können.
	
	Mittels Prototyp wurde die Machbarkeit dessen überprüft
	und anschliessend im Rahmen mehrerer Iterationen eine Webapplikation entwickelt.
	Zusammen mit dem Ansprechpartner der Kundengruppe wurden Usability- und Workflowtests durchgeführt, um Benutzeroberfläche
	und Datenfluss vom Entscheidungsverwaltungssystem bis ins Projektplanungstool zu validieren.
	Abschliessend folgte zur Stabilisierung eine Überarbeitungsphase.
	
	
	\section*{Ergebnis}
		
	Entstanden ist eine Webapplikation, die mögliche Entscheidungen aus einem angebundenen Wissensverwaltungssystem bezieht
	und dem Benutzer durch ein Metamapping ermöglicht, für konkrete Projekte entstehenden Entscheidungen mit eigenen Aufgaben zu Verknüpfen.	
	
	\begin{figure}[H]
		\includegraphics[width=\textwidth]{introduction/img/eeppiDecisionsAndTaskTemplates.jpg}
		\centering
		\caption{Metamapping: Verknüpfung von Entscheidungen und Aufgabenvorlagen}
		\label{fig:metamapping}
	\end{figure}
	
	Über einen  Administrationsbereich konfiguriert der Benutzer die Applikation nach seinen Bedürfnissen.
	Beispielsweise kann der Benutzer selbst den Aufbau der zu generierenden Aufgaben steuern. Dazu wurde ein Templatingmechanismus entwickelt,
	der dem Benutzer das Verwenden von eigenen Verarbeitungsfunktionen, sogenannten Processors, ermöglicht.
	
	EEPPI zeigt, was kommerzielle Produkte in diesem Bereich anbieten könnten.
	Allerdings zeigt EEPPI auch die Design-Herausforderungen einer solchen Software:
	Sowohl die Verknüpfung von Entscheidungen und Aufgaben wie auch die Anbindung an die umliegenden Systeme müssen sehr flexibel gestaltet sein.
	Mit EEPPI ist dies gelungen, doch es gibt noch viele Erweiterungsmöglichkeiten.
	EEPPI legt somit einen wichtigen Meilenstein im Forschungsbereich des interdisziplinären Entscheidungs- und Projektmanagements und
	zeigt auf, wohin die Reise zukünftiger Tools führen könnte.
	



% Title page
\thispagestyle{empty}
\begin{titlepage}
	\begin{minipage}{0.5\textwidth}
		\begin{flushleft} \large
			\includegraphics[width=0.7\textwidth]{media/img/logoHSR.png}
		\end{flushleft}
	\end{minipage}
	~
	\begin{minipage}{0.5\textwidth}
		\begin{flushright} \large
			\includegraphics[width=0.55\textwidth]{media/img/ifsLogo.png}
		\end{flushright}
	\end{minipage}
	
	\vspace*{2cm}
	\begin{center}
		{\fontsize{50}{40} \selectfont \textbf{EEPPI} \\[10mm]}
	
		\begin{figure}[H]
			\centering
			\includegraphics[scale=0.40]{media/img/eeppiLogo.png}
		\end{figure}		
		\vspace*{0.5cm}	
	
		{\fontsize{28}{40} \selectfont \textbf{\project} \\}
		\vspace{0.25cm}
		{\fontsize{22}{40} \selectfont \textbf{Technischer Bericht} \\[10mm]}
	
		{\fontsize{18}{20} \selectfont 
			Abteilung Informatik\\
			Hochschule für Technik Rapperswil \\
				
			{\fontsize{14}{16} \selectfont Herbstsemester 2014\\}
		}
		
	\end{center}
	
	\vspace*{1.5cm}
	\begin{minipage}[b]{0.4\textwidth}
		\begin{flushleft}
			\begin{tabular}{ll}  
				Autoren: & Laurin Murer, Tobias Blaser \\ 
				Betreuer: & Prof. Dr. Olaf Zimmermann\\ 
				Projektpartner: & IFS, HSR\\
				Experte: & Dr. Gerald Reif, Innovation Process Technology AG\\
				Gegenleser: & Prof. Hans Rudin\\
				Abgabedatum: & 19. Dezember 2014\\
			\end{tabular}
		\end{flushleft}
	\end{minipage}	

\end{titlepage}
\documentPartEntry{Technischer Bericht}
\clearpage

% Inhaltsverzeichnis
\tableofcontents

% Einleitung & projektorganisation
\chapter{Einleitung}
	% Beschrieb
	"<Lassen sich aus Projektentscheidungen Aufgaben ableiten?">
	Mit dieser Frage befasst sich die vorliegende Bachelorarbeit \eeppi.
		
	Sowohl das Treffen einer Entscheidung wie die Entscheidung zugunsten einer Wahlmöglichkeit implizieren Aufgaben in einem Projekt.
	So führt die Entscheidung "<Welche Art Session State soll verwendet werden?"> zum Beispiel zu den Aufgaben
	"<Session State evaluieren"> und "<Prototyp umsetzen">.		
	Wird bei dieser Entscheidung die Option "<Database Session State"> ausgewählt,
	so resultieren aus diesem Entscheid beispielsweise die Aufgaben "<Datenbank installieren"> und "<Session Persistenz implementieren">.
	
	
	\section{Ziele}
	Es soll herausgefunden werden, ob sich ein Metamapping formulieren lässt,
	das Entscheidungen und Aufgaben verknüpft.
	Weiter soll ermittelt werden, ob dieser Prozess automatisierbar und in einer Software umsetzbar ist.
	Dazu soll eine Webapplikation entwickelt werden, die Entscheidungen aus einem \dks\ bezieht, 
	dem Benutzer für das Metamapping ein Werkzeug anbietet und Aufgaben in ein \ppt\ exportiert.
	
	
	\section{Abgrenzung}
	Die bestehenden Systeme sollen dabei nicht verändert werden. So sollen keine Informationen aus dem \ppt\ zurück ins \dks\ fliessen.
	Auch die Zuständigkeitsbereiche der einzelnen Tools sollen klar getrennt bleiben. So soll \eeppi\ keine Funktionalität zum modellieren von Entscheidungen enthalten und auch keine Projektplanungsfunktionalität integrieren.
	Zudem soll der Fokus der Applikation im Bereich der Konzepte und deren Umsetzung liegen. Die Implementierung von Nebenfunktionen wie z.B. Rechte- und Rollenkonzepten besitzt entsprechend eine tiefere Priorität und soll daher schlank umgesetzt werden.
	
	
	\section{Optionale Erweiterungen}
	Optional sind beispielsweise eine Strukturierung von Aufgaben oder Funktionalität zum Import und Export von Aufgaben und Verknüpfungen denkbar.	
	Ebenfalls denkbar ist die Anbindung mehrerer \dks e und eine Verschmelzung deren Entscheidungen oder Mandantenfähigkeit der Applikation.
	
	
	\section{Mehrwert gegenüber bestehenden Lösungen}
	Lösungen zur Verknüpfung von Entscheidungsmanagement und Projektplanung sind ein aktuelles Forschungsthema, 
	entsprechend gibt es noch keine uns bekannten Produkte auf dem Markt.
	 		
		
	%\section{Motivation}
	
\chapter{Projektorganisation}

	\section{Projektteam}
	\begin{figure}[H]
		\begin{minipage}[b]{0.5\linewidth}
			\includegraphics[width=0.5\textwidth]{projectPlan/media/img/lmurer.jpg}
			\centering
			\caption{Laurin Murer}
			\label{fig:laurinmurer}
		\end{minipage}
		\begin{minipage}[b]{0.5\linewidth}
			\includegraphics[width=0.5\textwidth]{projectPlan/media/img/tblaser.jpg}
			\centering
			\caption{Tobias Blaser}
			\label{fig:tobiasblaser}
		\end{minipage}
	\end{figure}

	\section{Projektbegleitung}
	\begin{figure}[H]
		\begin{minipage}[b]{0.5\linewidth}
			\includegraphics[width=0.5\textwidth]{projectPlan/media/img/ozimmermann.jpg}
			\centering
			\caption{\teacher}
			\label{fig:olafzimmermann}
		\end{minipage}
	\end{figure}
	\teacher\ betreut die Semesterarbeit und begleitet das Team durch regelmässige Meetings.


% architektur analyse
\chapter{Architektur}
	\section{Architekturübersicht}

\section{Technologie}
	\subsection{Client}
		\subsubsection{Sprache}
			\begin{tabularx}{\textwidth}{|lXX|}
				\hline
					\textbf{} & \textbf{Vorteile} & \textbf{Nachteile}\\
				\hline
					\textbf{JavaScript} & 
					\begin{itemize}
						\item UI der bestehenden Applikation ist auch in JavaSscript geschrieben
					\end{itemize} & 
					\begin{itemize}
						\item Fehler tauchen erst zur Runtime auf
					\end{itemize} \\
				\hline
					\textbf{TypeScript} &
					\begin{itemize}
						\item Wird kompiliert (zu Javascript), weniger Fehler zur Runtime
						\item Optisch besser lesbar als JavaScript
					\end{itemize} &
					\begin{itemize}
						\item Erfordert TSC-Compiler
						\item Code Overhead bei Inheritance
					\end{itemize} \\
				\hline
			\end{tabularx}
			
			
		\subsubsection{Architektur-Framework}
			\begin{tabularx}{\textwidth}{|lXX|}
				\hline
					\textbf{} & \textbf{Vorteile} & \textbf{Nachteile}\\
				\hline
					\textbf{Angular JS} &
					\begin{itemize}
						\item bekanntes MVW- und Templating Framework, erlaubt eine saubere Trennung von Logik und Darstellung
						\item bindet ViewModel Properties und Functions ans Template, wodurch sich Observerkonstrukte sparen lassen
						\item ist stabil, zuverlässig, gut erweiterbar und bringt von sich aus schon sehr viel mit
						\item wurde auch schon für die bestehende Applikation eingesetzt
					\end{itemize} &
					\begin{itemize}
						\item Attribute Binding besitzt gewissen Overhead
					\end{itemize} \\
				\hline
					\textbf{Ember JS} &				
					\begin{itemize}
						\item Sehr Modular und anpassbar
					\end{itemize} &
					\begin{itemize}
						\item Bringt wesentlich weniger mit als Angular JS, mehr Eigenaufwand notwendig
					\end{itemize} \\
				\hline
					\textbf{Kein Framework} &
					\begin{itemize}
						\item Vollständig freie Architekturgestaltung
					\end{itemize} &
					\begin{itemize}
						\item Hoher Implementationsaufwand ohne Gewinn
					\end{itemize} \\
				\hline
			\end{tabularx}
			
			\subsubsection{Require.js}
				Require.js eignet sich gut zur Strukturierung und zum Autolading der Klassen und komponenten, 
				insbesondere während der Entwicklung.

		\subsection{UI Frameworks}
			\subsubsection{LESS}
				Less soll als CSS Generator eingesetzt werden, da es den CSS Code stark verschlankt und Vorteile wie Variablen und Mixins bietet. LESS kann bei einem Node.js Server serverseitig compiled werden um den Client zu entlasten.
				
		\subsection{Testing}
			Testing Framework Anforderungen:
			\begin{itemize}
				\item Einfach einzubinden
				\item Einfach zu erweitern
				\item Bekannte Benutzung mit Tests und Asserts
				\item Möglichkeit zur Anbindung eines Build Tools
			\end{itemize}

			\subsubsection{JsUnit / QUnit}
				JsUnit wie QUnit arbeiten mit einem realen Browser (keine Browsersimulation), 
				sind einfach handzuhaben und bieten typische Assert-Syntax.
				
				
	\subsection{Server}

		\subsubsection{Sprache}
			\begin{tabularx}{\textwidth}{|lXX|}
				\hline
					\textbf{} & \textbf{Vorteile} & \textbf{Nachteile}\\
				\hline
					\textbf{TypeScript / JavaScript} &
					\begin{itemize}
						\item Gleiche Technologie wie Client
					\end{itemize} &
					\begin{itemize}
						\item Vergrössert den Technologiepark da andere Technologie als bestehende Applikation
					\end{itemize} \\
				\hline
				\textbf{Java} &
					\begin{itemize}
						\item Gleiche Technologie wie bestehende Applikation
					\end{itemize} &
					\begin{itemize}
						\item Zu schwergewichtig
					\end{itemize} \\
				\hline
			\end{tabularx}
				

				
		\subsubsection{Framework / Technologie}		
		
			\begin{tabularx}{\textwidth}{|lXX|}
				\hline
					\textbf{} & \textbf{Vorteile} & \textbf{Nachteile}\\
				\hline
					\textbf{Node.js} &
					\begin{itemize}
						\item Schlank
						\item Guten Packagemanager
					\end{itemize} &
					\begin{itemize}
						\item Andere Technologie als bestehende Applikation
					\end{itemize} \\
				\hline
					\textbf{Tomcat/Jersey} &
					\begin{itemize}
						\item Einfacher Aufbau einer Rest API
					\end{itemize} &
					\begin{itemize}
						\item Umständlich, Konfigurationsfehleranfällig
					\end{itemize} \\
				\hline
			\end{tabularx}





	\section{Entwurf, Begründungen und Domain Model}
		Nachfolgend werden die wichtigsten Bezeichnungen von Domänenobjekten eingeführt:
		
		\begin{description}
			\item[Problem] Beschreibt eine Vorlage für ein Designproblem eines Software Projektes, 
				zum Beispiel "<Session State">. 
				Im \dks\ ADRepo werden Problems als "<Problem Template"> bezeichnet.
			\item[Alternative] Beschreibt eine Vorlage für eine Wahlmöglichkeit eines Problems.
				Für das Problem "<Session State"> wären dies zum Beispiel "<Server Session State"> und 	"<Database Session State">. 
				Im \dks\ ADRepo werden Alternatives als "<Option Template"> bezeichnet.
			\item[Decision] Beschreibt eine konkrete Problem-Instanz.
				Decisions entstehen, wenn Problems auf konkrete Projekte angewendet werden.
				Decisions sind sowohl geschlossene wie noch zu treffende Entscheidungen.
				Im \dks\ ADRepo werden Decisions als "<Problem Occurrences"> bezeichnet.
			\item[Option] Beschreibt eine Wahlmöglichkeit einer Decisions 
				und somit eine konkrete Instanz einer Alternative.
				Eine Option kann gewählt, nicht gewählt oder noch offen sein.
				Im \dks\ ADRepo werden Options als "<Option Occurrences"> bezeichnet.
			\item[\ttpl] Beschreibt eine Vorlage zum Erstellen von konkreten Tasks (Aufgabenvorlage).
				\ttpl s enthalten generische Werte, wie zum Beispiel "<Project Manager"> als
				Attributwert für die Eigenschaft "<Assignee">.
			\item[Task] Beschreibt einen aus einem \ttpl\ erzeugten konkreten Task.
				Task entstehen während dem Übertragen der Informationen eines \ttpl s an ein \ppt.
			\item[Mapping] Bezeichnet die Verknüpfung von Problems oder Alternatives mit einem \ttpl. 
				Anhand dieser Verknüpfung werden die, für die Übertragung eines Tasks an ein \ppt, benötigten Daten erstellt.
			\item[Requesttemplate] Bezeichnet eine Vorlage für einen HTTP-Request um 
				in einem spezifischen \ppt\ Tasks anzulegen.
				Requesttemplates beinhalten Platzhalter (Variablen und Funktionen, sog. Processors),
				die mit Daten der \ttpl s und Decisions oder Options ersetzt werden. 
			\item[Processor] Vererbeitungsfunktion (zu dt. Prozessor), 
				die Daten eines Mappings verarbeitet und einen Rückgabewert liefert, 
				der anstelle der Processorsignatur ins Requesttemplate eingefügt wird.
		\end{description}	
	
	
		\subsection{Konzeptionelle Domäne}
			Die \eeppi\ Domäne setzt sich aus drei Typen von Objekten zusammen: 
			\begin{itemize}
				\item Einer Abstraktion der Objekte hinter der Schnittstelle der angebundenen Entscheidungswissensverwaltung
				\item Einer Abstraktion\footnote{In Form des, durch das Request-Template gebundenen, Schnittstellenaufrufs} der Objekte hinter der Schnittstelle eines oder mehreren angebundenen \ppt s
				\item Den eigenen Objekten und Schnittstellen
			\end{itemize}
			
			Innerhalb von \eeppi\ werden nur die eigenen Objekte persistiert.
			Die Objekte der andern System werden On-Demand über die Schnittstellen geladen.
		
			\begin{landscape}
				\begin{figure}[H]
					\includegraphics[width=0.9\linewidth]{architecture/media/img/domain.png}
					\centering
					\caption{Konzeptionelle \eeppi -Domäne}
					\label{fig:domain}
				\end{figure}				
			\end{landscape}
			
			\eeppi\ arbeitet mit den Objekten des \dks. Das ADRepo als Referenz-\dks\ ist wie folgt aufgebaut:
			\begin{figure}[H]
				\includegraphics[width=\linewidth]{architecture/media/img/dksDomain.png}
				\centering
				\caption[ADRepo-Domäne\newline 
					\license{Eclipse Public License \url{https://www.eclipse.org/legal/epl-v10.html} Lukas Wegmann, IFS HSR \url{https://www.ifs.hsr.ch}}
				]{ADRepo-Domäne\footnote{Quelle: IFS HSR, Abbildungsverzeichnis}}
				\label{fig:dksDomain}
			\end{figure}
			
			Dabei werden die Objekte wie folgt gemappt:
			\begin{description}
				\item[Element] Node
				\item[ProblemTemplate] Problem
				\item[ProblemOccurrence] Decision
				\item[OptionTemplate] Alternative
				\item[OptionOccurrence] Option
			\end{description}
			
			
		\subsection{Metamapping}
			Metamapping bezeichnet das Konzept, 
			wie die Domäne des \dks s (DKS) und die \eeppi -Domäne verknüpft werden sollen.
			Übergreifend gesehen geht es dabei um nichts Geringeres als die Verbindung von Entscheidungsmanagement und Projektmanagement.			
		
			\begin{figure}[H]
				\includegraphics[width=\linewidth]{architecture/media/img/metaMapping.png}
				\centering
				\caption{Metamapping}
				\label{fig:metamapping}
			\end{figure}		
			
			Dazu wird die \dks -Domäne von \eeppi\ abstrahiert und mit Task\-tem\-plates verknüpft.
			Jedes Mapping steht für eine Verknüpfung eines \ttpl s mit einem Problem oder einer Alternative.
			
			Beim Erzeugen von konkreten Tasks müssen zu jedem Mapping alle Occurrences (siehe ADRepo-Domäne Abbildung \ref{fig:dksDomain}) gefunden werden. %TODO: gell, das war ADRepo und nicht generell DKS?
			Anschliessend können zusammen mit den Daten des \ttpl s und dessen Eigenschaften (siehe \eeppi -Domäne Abschnitt \ref{fig:domain}) Tasks erzeugt werden.
			
			
		\subsection{Implementationsdomäne}
			Die konzeptionelle Domäne von \eeppi\ berücksichtig auch mögliche Erweiterungsaspekte.
			Die Implementationsdomäne bricht aus der konzeptionellen die Teilmenge heraus, die effektiv umgesetzt wird und konkretisiert diese.
			
			\begin{landscape}
				\begin{figure}[H]
					\includegraphics[width=0.9\linewidth]{architecture/media/img/implementationDomain.png}
					\centering
					\caption{\eeppi -Implementationsdomäne}
					\label{fig:implementationDomain}
				\end{figure}				
			\end{landscape}	
			
			Ebenfalls enthält die Implementationsdomäne die Abstraktion der DKS Objekte.
			Sie zeigt auch das gegenüber der konzeptionellen Domäne einfacher gehaltene Benutzer- und Projekt-Konzept.
			
			
		\subsection{Umbenennen und Löschen von Domänenobjekten}
			Um Probleme mit referenzierten Domänenobjekten zu vermeiden,
			darf es Benutzern nur möglich sein, \ttpl s umzubenennen,
			nicht jedoch zu löschen.			

			Das Löschen von referenzierten \ttpl s würde dazu führen, 
			das ein Teil der Export-Historie verloren gienge und Mappings kein zugeordnetes \ttpl\ mehr besitzen würden.
			 Dies wiederum würde zu fehlerhaften Exports, bzw. leeren Exports führen.
			 Aus diesem Grund ist das Löschen von referenzierten \ttpl s keine Option, höchstend das Löschen von noch nicht referenzierten.			
			
			Könnten Benutzer \ttpl s löschen, so müsste geprüft werden, 
			ob diese referenziert werden. Sobald der Benutzer jedes \ttpl einmal exportiert hätte, könnte er ebenfalls keines mehr löschen.
			Die Möglichkeit zum Löschen ist entsprechend sowieso nur in einem neu aufgesetzten System gegeben.
			
			Das konsequente Umsetzen der "<Umbenennen statt Löschen">-Strategie ermöglich Benutzern, 
			falsch angelegte \ttpl s weiterzuverwenden, vermeidet jedoch, 
			dass sich die Applikation unterschiedlich verhält für bereits referenziert und noch nicht referenzierte \ttpl s.
			
			Das Gleiche gilt auch für Task Properties.
			Dies dürfen auf keinen Fall vom Benutzer entfernt werden 
			und dürfen entsprechend nur umbenannt werden.
			
		
		\subsection{\ttpl s Strukturierung}
			Es gibt verschiedene Möglichkeiten, 
			Benutzer eine Strukturierung von \ttpl s anzubieten.
			\ttpl s können selbst in eine Struktur gebracht werden
			oder durch externe Strukturen geordnet werden.
		
			\begin{figure}[H]
				\includegraphics[width=\textwidth]{architecture/media/img/taskTemplateStructure.png}
				\centering
				\caption{Strukturierungsmöglichkeiten von Tasks}
				\label{fig:taskTemplateStructure}
			\end{figure}
			
		\decision{
			\decisionHeader{DOM-TT-STRUC}{\ttpl s Strukturierung}{Architecture}{Domain}
		}{
			\decisionContent{Keine Strukturierung, allenfalls auf Wunsch des Vertreters der Anforderungsgruppe nicht hierarchische Kategorien, Smart Filters}
			{Welches Konzept soll zur Strukturierung von \ttpl s eingesetzt werden?}
			{\ppt\ s unterstützen überhaupt hierarchische Tasks, ansonsten sind entsprechende Strukturierungsmöglichkeiten wenig sinnvoll}{Benutzer sollen einfach und schnell erstellte \ttpl s wieder finden.}
			{
				\begin{description}
					\item[Keine Strukturierung] \
						\begin{description}
							\item[Vorteile] Einfach zu implementieren, einfach verständlich für den Benutzer
							\item[Nachteile] Bei vielen \ttpl s unübersichtlich, führt zu doppelten \ttpl s,
							da existierende nicht gefunden werden. 
						\end{description}
						
					\item[Labels/Hierarchische Labels] \
						Versehen der Elemente mit einem oder mehreren Labels. Benutzer können nach Labels suchen oder Filtern, um Vorlagen anzuzeigen.	
						\begin{description}
							\item[Vorteile] Einfach verständlich für den Benutzer
							\item[Nachteile] Benutzer könnten zu faul sein, 
								Labels anzulegen und zuzuordnen da es aufwändiger ist, als Kategorisieren
						\end{description}
					
					\item[Direkte Hierarchisierung der \ttpl s] \
						Elemente werden direkt mit Elternelementen verknüpft und bilden einen hierarchischen Baum.	
						\begin{description}
							\item[Vorteile] Einfach zu implementieren
							\item[Nachteile] Schwer verständlich für den Benutzer,
								 da die Hierarchisierung unter Umständen nicht mit dem Workflow zusammenpasst
						\end{description}
					
					\item[Prototyping statt Strukturierung] \
						Elemente erben Funktionalität von einander, statt strukturiert zu werden.
						\begin{description}
							\item[Vorteile] Verringert die Anzahl \ttpl s massiv, 
								da Eigenschaften vererbt werden können
							\item[Nachteile] Schwieriger umzusetzen, schwieriger zu verstehen für Benutzer
						\end{description}					
				\end{description}
			}
			{
				Keine Strukturierung hat für die Entwicklung sowie für den Benutzer Vorteile.
				So gibt es zu diesem Punkt kaum Entwicklungsaufwand 
				und der Benutzer findet trotzdem dank Suchfunktionen die gesuchten \ttpl s.
				Ausserdem muss er keine Zeit aufwenden um die \ttpl s zu strukturieren.

				Falls Vertreter der Anforderungsgruppe jedoch eine weitergehende Strukturierung wünschen, 
				empfehlen wir nicht-hierarchische Kategorien sowie Smart Filters.
				Kategorien erlauben eine Strukturierung auf einfache Weise. 

				Smart Filter sind eine Ergänzung zu den beiden Optionen und ermöglichen das schnelle Finden anhand von Eigenschaften,
				ohne dass diese der Benutzer erfassen muss.
			}
			{keine}
			{Mögliche zukünftige Erweiterung durch Strukturierungsmöglichkeiten muss bei der Implementation berücksichtigt werden}
			{keine}
		}
		
		
		\subsection{Verknüpfungen von \ttpl s und Entscheidungs-Vorlagen}
			Wissensproduzenten können \ttpl s zu Entscheidungs-Vorlagen zuordnen.
			Dabei kann und soll auch ein \ttpl\ an verschiedene Entscheidungs-Vorlagen zugeordnet werden können.
			Ebenso können Entscheidungs-Vorlagen natürlich mehrere \ttpl s zugeordnet erhalten.
			
			\subsubsection{Arten der Zuordnung}
				\ttpl s können mit Entscheidungs-Vorlagen auf zwei Arten verknüpft werden:
				\begin{enumerate}
					\item Sie können dann fällig werden, wenn eine Entscheidung getroffen wurde (operativer Task).
					\item Ein \ttpl\ dient dazu, Entscheidungen zu treffen (Entscheidungstask).
				\end{enumerate}
				Auf die \ttpl s selbst hat dies keinen Einfluss, sie sind unabhängig davon. 
				Ob es sich um einen operativen Task oder einen Entscheidungstask handelt, hängt nur davon ab,
				ob das \ttpl\ mit einer Entscheidung (Node) oder einer Option einer Entscheidung (Subnode) des Entscheidungsbaumes verknüpft ist.
				
				Sollte ein Unterscheidung dennoch einmal notwendig sein, 
				so kann dies mittels Processors umgesetzt werden (Siehe Abschnitt \ref{subsec:transmissionWorkflow}), da die Daten des \dks\ s Typeninformationen enthalten.
				

			\subsection{Übertragung von \ttpl s in \ppt}
				Aus \ttpl s werden beim Übertag in ein \ppt\ Tasks generiert.
								
				\ttpl s sind generische Vorlagen, die ständig weiterentwickelt werden sollen.
				Aus diesem Grund wäre es unpraktisch, wenn ein Benutzer für jedes Mapping dessen \ttpl anpassen müsste, auch wenn es sich um das gleich \ttpl handelt.
				Die Anzahl \ttpl s, die der Benutzer aktuell halten müsste, würde mit der Anzahl Mappings wachsen und schnell einen unüberblickbare Grösse erreichen.
				Darum werden \ttpl s mit Mappings verknüpft (referenziert) und nicht kopiert.
				Der Benutzer kann ein \ttpl anpassen. Exportiert er das nächste Mal Tasks, 
				so wird für alle gemappten Problems das aktuelle \ttpl verwendet.
				
				\begin{figure}[H]
					\includegraphics[width=\textwidth]{architecture/media/img/decisionTaskRelation.png}
					\centering
					\caption{Übertragen von Entscheidungen und \ttpl s}
					\label{fig:DecisionTaskRelation}
				\end{figure}
				
				Beim Übertragen werden aus \ttpl s (konkrete) Tasks.
				Mit Optionen von Entscheidungen verknüpfte Tasks werden entsprechend zu Sub-Tasks.
				
				Benutzer wollen bei der Übertragung ins \ppt\ die vom \ttpl\ vorgegebenen Werte möglicherweise anpassen, wie zum Beispiel den erwarteten Aufwand für den Task.
				Daher ist es sinnvoll, die Eigenschaften der \ttpl s in die (konkreten) Tasks zu kopieren, anstatt sie lediglich zu verknüpfen.
				Gleiches gilt für Entscheidungen. Würde jemand im \dks\ diese verändern oder löschen,
				so würde dies die History zerstören.
			
		
		\subsection{Tasktransmission Workflow}
			\label{subsec:transmissionWorkflow}		
		
			Aus \ttpl s erzeugte Tasks müssen zur Übertragung in ein \ppt\
			toolspezifisch umgewandelt werden. Dazu werden "<Processors"> eingesetzt.
			Processors stellen kleine Funktionalitäten dar, die Daten umwandeln.
			Beispiele für Processors sind:
			\begin{itemize}
				\item "<Date processor">, der Kalenderdaten umwandelt
				\item "<Issue type processor">, der Issuetypen konvertiert
				\item "<User processor">, der Relationen zu Benutzern so umwandelt, dass das \ppt\ den User korrekt verknüpfen kann
				\item "<Conditional processor"> und "<Option processor">, die Bedingungen verarbeiten
			\end{itemize}
			Ebenfalls denkbar ist ein Processor, der Felder aggregieren kann und damit zum Beispiel nicht gemappte Felder in die Beschreibung überführen kann.
			
			\begin{figure}[H]
				\includegraphics[width=\textwidth]{architecture/media/img/transmissionWorkflow.png}
				\centering
				\caption{Übertragen von Tasks}
				\label{fig:transmissionWorkflow}
			\end{figure}
			Dem Transmissionworkflow liegt das "<Pipes and filters">-Pattern zugrunde
			 \cite{hope_enterprise_2003}, welches eine Verarbeitungs- und Filterkette definiert.
			
			Der komplette Tasktransmission Workflow soll auf dem Client durchgeführt werden und nicht auf dem Server. 
			Die Wahl der Servertechnologie schränkt die Möglichkeiten für dynamische Processors ein, während die Client Technologie dies ermöglicht.
			
			Im Laufe der Erarbeitung dieses Workflows wurde auch darüber nachgedacht, 
			wie Eigenschaften verarbeitet werden sollen, die nicht gemappt wurden. 
			Ursprünglich wurde entschieden, diese in Listenform in die Beschreibung des Tasks
			überzuführen. 
			Die Entscheidung für ein sehr flexibles Mapping führte dazu, 
			dass dieser Anwendungsfall überflüssig wurde, 
			weil der Administrator dazu selbst einen Processor definieren kann.
			Womit die Entscheidung darüber bei ihm bleibt und nicht von uns vorgegeben wird.
			Zudem müsste in jedem Fall bekannt sein, 
			bei welchem Feld es sich um das Beschreibungsfeld handelt, 
			was nicht gegeben ist, wenn der Administrator vergisst, dies zu konfigurieren.
			
		
		\subsection{Mapping Methode}
		\decision{
			\decisionHeader{DOM-TT-MM}{Mapping Methode}{Architecture}{Domain}
		}{
			\decisionContent{Konfiguration/Block in Form von Templates mit Platzhaltern}
			{Wie sollen Mappingkonfigurationen erstellt werden?}
			{keine Speziellen}
			{Von der Mapping Method hängt die Architektur des Mappings und die Schnittstellen der Processors und Filters ab.}
			{
				\begin{description}					
					\item[Hierarchische/Element basierte Konfiguration] \
					Das Mapping wird durch das Anlegen von verknüpften Elementen erzeugt.
					\begin{description}
						\item[Vorteile] Gegebene Validierung durch die Struktur, kein Parser notwendig
						\item[Nachteile] Aufwändiger umzusetzen, insbesondere das UI, weniger flexibel
					\end{description}
				\end{description}
			}
			{Eine Textblock/Template-basierte Konfiguration erhöht zwar die Fehlermöglichkeiten für den Administrator,
			ermöglicht diesem jedoch grössere Flexibilität und damit ein Abdecken einer grösseren Bandbreite an \ppt s.}
			{Der Administrator kann mit Templates und Platzhaltern umgehen oder es lernen}
			{Das Mapping benötigt kein eigenes Datenmodell in Form von verknüpften Objekten. 
			Es kann als einfache Text-Elemente an ein Projekt angeknüpft werden.}
			{keine}
		}
		
		Der Ablauf für einen Administrator sieht entsprechend wie folgt aus:
		\begin{enumerate}
			\item \ppt\ definieren
			\item Taskeigenschaften erstellen
			\item Mapping Taskeigenschaften -> \ppt\ erstellen
		\end{enumerate}					
				
		
		\subsection{Kommunikation}
		\eeppi\ verwendet verschiedene Schnittstellen. Deren Verwendungen sind nachfolgend beschrieben.
			\subsubsection{Kommunikation zwischen \eeppi\ und dem \dks}
				Die Übertragung zwischen dem \dks\ und \eeppi\ basiert auf einer einer Ein-Weg-Kommunikation.
				\eeppi\ lädt die Daten, gibt jedoch keine Informationen zurück, ob dies erfolgreich war oder nicht.
				
				\eeppi\ schreibt auch keine Daten zurück ins \dks, wie beispielsweise Informationen über die Verwendung der Problems und Alternatives.
				Grund dafür ist die dazu notwendige Komplexität der Schnittstelle sowie die mangelnde Unterstützung seitens \dks.
				Auch würde diese Funktionalität den Rahmen der Arbeit sprengen, als mögliche Erweiterung von \eeppi\ ist sie jedoch denkbar.
				
			
			\subsubsection{Kommunikation zwischen \eeppi\ und dem \ppt}
				Die Übertragung der erzeugten Tasks ins \ppt\ basiert auf einer Zwei-Weg-Kommunikation.
				Zum einen werden die Tasks ans \ppt\ übertragen
				und zum anderen die Rückmeldung über die erfolgreiche Erzeugung der Tasks im \ppt\ wieder im \eeppi\ gespeichert.
				Ebenfalls zurückgeliefert werden minimale Informationen über den erstellten Task, 
				wie zum Beispiel die ID. 
				\eeppi\ verwendet diese Informationen, um dem Benutzer Secondary-Processors anzubieten.
				Also Processors, die zur Zeit der Übertragung ausgeführt werden und zum Beispiel Informationen über einen Parent Task in das Requesttemplate einweben können.
				
				Der "<Transmission">-Workflow als Gesamtes ist jedoch ein One-Way-Procedure.
				Dies bedeutet, dass keine Daten aus dem \ppt\ zurück in \eeppi\ fliessen.
				Technisch zwar möglich, stellt eine Rückkopplung der Daten oder sogar eine Synchronisierung der Tasks eine grosse Herausforderung dar, 
				die den Rahmen der Arbeit sprengen würde.
				Als mögliche zukünftige Erweiterung ist dies jedoch denkbar.
%\begin{landscape}
\chapter{Lizenzen und verwendete externe Produkte}
	\section{\eeppi}
	\label{sec:eeppiLisences}
		
	\eeppi\ und die zugehörige Dokumentation stehen unter den folgenden Lizenzen:
	\vspace{0.5cm}
	
	\begin{tabularx}{\linewidth}{|l|X|}
		\hline
		\eeppi\ Source Code & Apache 2\\
		\hline
		Bachelorarbeit "<Entwurfsentscheidungen als Projektplanungsinstrument"> & CC 4.0\\
		\hline
	\end{tabularx}
	

	\section{Verwendete Frameworks und Libraries}
	\label{sec:usedLibrariesAndFrameworks}
	Um nicht von Grund auf alle Komponenten selbst programmieren zu müssen 
	und um auf die Erkenntnisse anderer Entwickler aufbauen zu können,
	haben wir einige bestehende Frameworks verwendet.
	Diese sind nachfolgend, und die dazugehörenden Lizenzen in Absatz~\ref{sec:licenses} aufgeführt.
	
	
	\vspace{0.5cm}
	
	\newcommand{\addLib}[5]{
		#5 & #1 & #2 & \url{#3} & #4 \\
		\hline
	}
	
	
	\begin{tabularx}{\linewidth}{| l | l r | X | c |}
		\hline
		\textbf{Verwendung} & \textbf{Framework/Library} & \textbf{Version} & \textbf{URL} & \textbf{Lizenz} \\
		\hline \hline
		\addLib{AngularJS}{1.3.0}{https://angularjs.org/}{MIT License}{Framework (Client)}
		\addLib{Play Framework}{2.3.6}{https://www.playframework.com/}{Apache 2}{Framework (Server)}
		\addLib{Hibernate Entitymanager}{4.3.6.Final}{http://hibernate.org/orm/}{LGPL}{Library (Server)}
		\addLib{PostgreSQL Driver}{9.1-901.jdbc4}{http://mvnrepository.com/artifact/org.postgresql/postgresql}{PostgreSQL}{Library (Server)}
		\addLib{Jetbrains Annotations}{7.0.2}{http://mvnrepository.com/artifact/com.intellij/annotations}{Apache 2}{Code Library (Server)}
	\end{tabularx}
		
	\begin{tabularx}{\linewidth}{| l | l r | X | c |}
		\hline
		\textbf{Verwendung} & \textbf{Framework/Library} & \textbf{Version} & \textbf{URL} & \textbf{Lizenz} \\
		\hline \hline
		\addLib{Jasmine}{2.0}{http://jasmine.github.io/}{MIT}{Test Framework (Client)}
		\addLib{Mockito}{1.10.8}{https://code.google.com/p/mockito/}{MIT}{Test Library (Server)}
		\addLib{PowerMock}{1.5.6}{https://code.google.com/p/powermock/}{Apache 2}{Test Library (Server)}
		\addLib{Selenium}{2.43.1}{http://www.seleniumhq.org/}{Apache 2}{Test Library (Server)}
	\end{tabularx}

	\section{Verwendete Tools und Technologien}
	
		\begin{tabularx}{\linewidth}{| l | l | l | X |}
			\hline
			\textbf{Verwendung} & \textbf{Tool/Technologie} & \textbf{URL} & \textbf{Lizenz} \\
			\hline \hline
			Client & TypeScript & \url{http://www.typescriptlang.org/} & Apache2 \\\hline
			Client Dokumentation & Typedoc & \url{https://www.npmjs.com/package/typedoc} &  Apache 2 \\ \hline
			Client \& Server & Less & \url{http://lesscss.org/} &  Apache 2 \\ \hline
			Server & Java & \url{http://www.oracle.com/technetwork/java/} & GPL, Java Community Process\\ \hline
			Projekt & Vagrant & \url{https://www.vagrantup.com/} & MIT \\ \hline
			Projekt & Virtualbox & \url{https://www.virtualbox.org/} &  GPL \\ \hline
		\end{tabularx}
		
			
	\section{Lizenzen}
	\label{sec:licenses}
	
	\newcommand{\addLicense}[6]{
		#1 & #2 & \url{#3} & #4 & #5 & #6	\\
		\hline
	}

	\begin{tabularx}{\linewidth}{| l | X | X | p{4.3cm} | p{4.7cm} | p{3.5cm} |}
		\hline
		\textbf{Kurzname} & \textbf{Voller Name} & \textbf{URL} & \textbf{Bedingung\footnote{\label{licensesFootnote}Quelle: gemäss \url{http://choosealicense.com/} \cite{github_choosing_2014}} } & \textbf{Erlaubt\footref{licensesFootnote}} & \textbf{Verboten\footref{licensesFootnote}} \\
		\hline \hline
		\addLicense{Apache 2}{Apache 2 License}{http://www.apache.org/licenses/LICENSE-2.0}{
			• Lizenz und Copyright Informationen beilegen,\newline
			• Änderungs"-his"-to"-rie\newline angeben
		}{
			• Kommerzielle Nutzung,\newline
			• Vertrieb,\newline
			• Veränderung,\newline
			• Patent Erteilung,\newline
			• Privater Gebrauch,\newline
			• Weitere Lizenz
		}{
			• Haftbar machen,\newline
			• Mar"-ken"-kenn"-zei"-chen verwenden
		}
		\addLicense{CC 4.0}{ Creative Commons Attribution 4.0 International License}{http://creativecommons.org/licenses/by/4.0/}{
			• Lizenz und Copyright Informationen beilegen
		}{
			• Kommerzielle Nutzung,\newline
			• Kopieren und weitergeben in beliebigem Format und Medium\newline
			• Kombinieren, verändern und darauf aufbauen
		}{
			• Originalen Urheber unterschlagen
		}
	\end{tabularx}
		
		
	\begin{tabularx}{\linewidth}{| l | X | X | p{4.3cm} | p{4.7cm} | p{3.5cm} |}
		\hline
		\textbf{Kurzname} & \textbf{Voller Name} & \textbf{URL} & \textbf{Bedingung\footref{licensesFootnote}} & \textbf{Erlaubt\footref{licensesFootnote}} & \textbf{Verboten\footref{licensesFootnote}} \\
		\hline \hline
		\addLicense{GPL}{GNU General Public License 2.0}{http://www.gnu.org/licenses/gpl-2.0.html}{
			• Source öffentlich,\newline
			• Lizenz und Copyright Hinweise,\newline
			• Änderungs"-his"-to"-rie\newline angeben
		}{
			• Kommerzielle Nutzung,\newline
			• Vertrieb,\newline
			• Veränderung,\newline
			• Patent Erteilung,\newline
			• Privater Gebrauch
		}{
			• Haftbar machen,\newline
			• Weitere Lizenz
		}
		\addLicense{LGPL}{GNU Lesser General Public License}{https://www.gnu.org/licenses/lgpl.html}{
			• Quellcode öffentlich,\newline
			• Library Verwendung,\newline
			• Lizenz und Copyright Informationen beilegen
		}{
			• Kommerzielle Nutzung,\newline
			• Vertrieb,\newline
			• Veränderung,\newline
			• Patent Erteilung,\newline
			• Privater Gebrauch,\newline
			• Weitere Lizenz
		}{
			• Haftbar machen
		}		
		\addLicense{MIT}{MIT License}{http://opensource.org/licenses/MIT}{
			• Lizenz und Copyright Informationen beilegen
		}{
			• Kommerzielle Nutzung,\newline
			• Vertrieb,\newline
			• Veränderung,\newline
			• Privater Gebrauch,\newline
			• Weitere Lizenz
		}{
			• Haftbar machen
		}
		\addLicense{PostgreSQL}{PostgreSQL License}{http://opensource.org/licenses/postgresql}{
			ähnlich wie die BSD oder MIT Lizenz (siehe oben)
		}{
			ähnlich wie die MIT Lizenz (siehe oben)
		}{
			ähnlich wie die MIT Lizenz (siehe oben)
		}
	\end{tabularx}
\end{landscape}


% interfaces & protocols
\chapter{Schnittstellen und Protokolle}
	
\section{RESTfull HTTP Schnittstelle}

	\eeppi\ besitzt eine RESTfull Schnittstelle, die andere Applikationen benutzen können, um \eeppi\ direkt anzusprechen.
	Diese Schnittstelle, auch API\footnote{Cascading Style Sheets} genannt, wird auch von der eigenen Clientapplikation benutzt.

	Das API ist auf dem REST\footnote{Representational State Transfer}-Level 2\footnote{REST Maturity Model: \url{http://de.wikipedia.org/wiki/Representational\_State\_Transfer\#REST\_Maturity\_Model}}.
	Das heisst, die einzelnen Elemente haben ihre eigenen Adressen
	und können über die HTTP-Verben GET, POST und DELETE verwendet werden.
	
\section{Dokumentation des API}
	\label{sec:apiDokumentationCreation}
	Sowohl für die Entwicklung des Clients wie auch für die weitere Entwicklung von \eeppi\ wurde das API des Servers dokumentiert (siehe Abschnitt~\ref{sec:apiDocumentation}).
	Ein Ausschnitt der API-Dokumentation ist in Abbildung~\ref{fig:apiScreenshot} abgebildet.
	Das Konzept und die Methodik der Dokumentation wurde als Teil von \eeppi\ erstellt
	und ist strukturell an das API von Jira\footnote{\url{https://docs.atlassian.com/jira/REST/latest/}} angelehnt.
	
	\begin{figure}[H]
		\includegraphics[width=\textwidth]{interfacesAndProtocols/media/img/apiDocumentation.png}
		\centering
		\caption{Auszug aus der API-Dokumentation}
		\label{fig:apiScreenshot}
	\end{figure}

	\subsection{Herkunft der Daten}
		Damit das API mit möglichst geringem Aufwand auf dem neusten Stand bleibt, wurde darauf geachtet,
		die darin angezeigten Daten möglichst direkt aus den Originalquellen zu beziehen oder zumindest von möglichst nah davon.
		\subsubsection{Liste aller API-Methoden}
			Die Liste der verfügbaren Methoden wird direkt aus dem Programmcode abgeleitet.
			Dazu wird mit Hilfe von Reflections\footnote{\url{http://docs.oracle.com/javase/7/docs/api/java/lang/reflect/package-summary.html}} in einem ersten Schritt die Liste aller Controller-Klassen eruiert
			und in einem zweiten Schritt deren Methoden extrahiert, die einen API-Aufruf repräsentieren.
		
		\subsubsection{HTTP-Verb und Pfad}
			HTTP-Verb und Ressourcenpfad werden aus der "<routes">-Datei geladen.
			Auch das Serverframework (Play Framework) lädt diese Informationen aus dieser Datei zur Delegation von Anfragen von Clients an den richtigen Controller.
			Damit ist sichergestellt, dass diese Daten in der API-Dokumentation stets aktuell sind.
			
		\subsubsection{Beschreibungen}
			Die verschiedenen Beschreibungen für die Methoden werden aus Annotationen direkt bei der Controller-Methode generiert.
			In Abbildung~\ref{fig:apiAnnotations} ist ein Beispiel eines Controllers mit Annotationen abgebildet.
			Die Annotationen beschreiben:
			\begin{itemize}
				\item{alle Parameter}
				\item{die Methode im Ganzen}
				\item{alle möglichen Rückgabestatus und deren konkrete Bedeutung}
				\item{ob eine Authentifizierung nötig ist}
				\item{die Beispielaufrufe (siehe \ref{subsubsec:exampleQueries})}
			\end{itemize}
			\begin{figure}[H]
				\includegraphics[width=0.9\textwidth]{interfacesAndProtocols/media/img/apiAnnotations.png}
				\centering
				\caption{Annotations im DecisionKnowledgeSystemController}
				\label{fig:apiAnnotations}
			\end{figure}
			
	\subsection{Beispielaufrufe}
	\label{subsubsec:exampleQueries}
		Um dem Benutzer garantieren zu können, dass und wie die Methode wirklich funktioniert,
		werden für jede Methode Beispielaufrufe und deren Antworten live generiert und angezeigt.
		Dazu werden bei jeder Generierung der API-Dokumentation zuerst einige Beispieldaten erstellt
		und anschliessend auf diesen Daten die Methode wie ein externer, unabhängiger Client aufgerufen.
		Die erhaltenen Ergebnisse werden dann in der API als "<The previous command \textbf{did} return"> angezeigt.
		Falls eine solche Simulation, beispielsweise aufgrund von externen Abhängigkeiten, nicht möglich ist,
		so ist die vom Entwickler erwartete Antwort auch in der Annotation definiert und wird dem Benutzer als
		"<The previous command \textbf{would probably} return"> angezeigt.
		
		Damit die dafür generierten Beispieldaten nicht mit den echten Daten des Systems interferieren,
		existiert für diesen Teil eine separate Datenbank.
		Diese muss der Benutzer jedoch nicht konfigurieren.
		Da die Daten nur während der Generation der API-Dokumentation benötigt werden
		und nicht längerfristig persistiert werden müssen,
		werden sie lediglich in einer SQLite-Datenbank\footnote{Einfache Datenbank, in welcher die Daten in einer einzigen Datei gespeichert werden} gespeichert.
	


\section{Client}
		Die \eeppi\ Clientapplikation nutzt die RESTfull Schnittstelle zum Datenaustausch mit dem Server sowie zur Kommunikation mit Remote-Hosts über die Proxies (Cross-Original Aufruf).
		
		Um auf dem Client einfach und flexibel Prototypen aus übertragenen JSON-Objekten instanziieren zu können, gibt es eine Object Factory.
		Diese baut anhand einer Factorykonfiguration, die jedes übertragbare Objekt deklarieren muss, Objekte zusammen und füllt sie mit Daten.
		Diese zusätzliche Konfiguration ist notwendig, da JavaScript nicht genügend Typeninformationen besitzt, aus denen sich die erforderlichen Informationen ermitteln liessen und TypeScript diese nicht automatisch generieren kann.
		Alternativ liesse sich die Prototypeninstanziierung  durch Factoryfunktionen pro Objekt umsetzen, dies hat sich jedoch während der Entwicklung als wartungsintensiv und Duplicated-Code-lastig erwiesen.
		
\section{HTTP Verben}
		RESTfull impliziert die Verwendung der richtigen Verwendung der HTTP Verben: 
		\begin{description}
			\item[GET] für Abfragen
			\item[POST] für Create-Operationen
			\item[DELETE] für Lösch-Operationen
			\item[PUT oder POST] für Update-Operationen.
		\end{description}
		
		Viele Netzwerkadministratoren erlauben PUT und DELETE nicht und blockieren entsprechenden Verkehr. Aus diesem Grund setzen Entwickler häufig nur GET und POST ein.
		
		Wir haben uns entschieden, serverseitig beides zu ermöglichen:
		Um Objekte zu löschen kann entweder 'DELETE /<entity>' oder 'POST /<entity>/delete' aufgerufen werden.
		
		Die Client Applikation verwendet nur POST und GET, dies kann allerdings umkonfiguriert werden.
		PUT Requests werden von \eeppi\ in der Standardkonfiguration gar nicht verwendet,
		es kommt stattdessen immer POST zum Einsatz.

\chapter{Benutzeroberfläche}
	\documentSubPartEntry{Benutzeroberfläche}
	
	\section{Userinterface-Mocking}
	
		Für einen Erstentwurf des Userinterface hat das Team ein Wireframe-Brainstorming durchgeführt.
		Dazu hat jedes Teammitglied Wireframes und Workflows entworfen.
	
		\begin{figure}[H]
			\includegraphics[width=\linewidth]{interfacesAndProtocols/media/img/wireframesTobias1.jpg}
			\centering
			\caption{Wireframes Tobias}
			\label{fig:wireframesTobias1}
		\end{figure}
		
		\begin{figure}[H]
			\begin{minipage}[b]{0.5\linewidth}
				\includegraphics[width=\linewidth]{interfacesAndProtocols/media/img/wireframesLaurin1.jpg}
			\end{minipage}
			\begin{minipage}[b]{0.5\linewidth}	
				\includegraphics[width=\linewidth]{interfacesAndProtocols/media/img/wireframesLaurin2.jpg}
			\end{minipage}
		\end{figure}
		
		\begin{figure}[H]
			\begin{minipage}[b]{\linewidth}		
				\includegraphics[width=\linewidth]{interfacesAndProtocols/media/img/wireframesLaurin3.jpg}
			\end{minipage}			
			\centering
			\caption{Wireframes Laurin}
			\label{fig:wireframesLaurin}
		\end{figure}
		
		Anschliessend wurden diese Entwürfe gemeinsam gesichtet, 
		Wireframes aussortiert oder ausgewählt und Ideen zu neuen Wireframes kombiniert.
		
		\begin{figure}[H]
			\begin{minipage}[b]{0.5\linewidth}
				\includegraphics[width=\linewidth]{interfacesAndProtocols/media/img/dashboard.jpg}
			\end{minipage}
			\begin{minipage}[b]{0.5\linewidth}	
				\includegraphics[width=\linewidth]{interfacesAndProtocols/media/img/tasks.jpg}
			\end{minipage}
			\caption{Dashboard \& Tasks}
			\label{fig:dashboardAndTasks}
		\end{figure}
		
		Für den Taskexport fiel die Entscheidung auf einen Assistent mit drei Schritten:
		\begin{enumerate}
			\item Auswählen von Projekt und \dks\ sowie des Mapping sets
			\item Bearbeiten der zu exportierenden Tasks, entfernen von nicht gewünschten
			\item Auswahl des \ppt s, exportieren sowie Übersicht über den Status eines laufenden Exports
		\end{enumerate}
		
		Eine Progressbar im unteren Bereich soll dem Benutzer jederzeit anzeigen, bei welchem Schritt er sich befindet.
		
		\begin{figure}[H]
			\includegraphics[width=\linewidth]{interfacesAndProtocols/media/img/exportWorkflow.jpg}
			\caption{Export Assistent}
			\label{fig:exportAssistent}
		\end{figure}	
		
		Der Benutzer kann nur weitergehen, nicht jedoch zurück, da dies durch den Umstand, 
		dass aus \ttpl s Tasks generiert werden, 
		zu Datenverlust führen könnte.
		
		\begin{figure}[H]
			\centering
			\includegraphics[width=0.3\linewidth]{interfacesAndProtocols/media/img/administration.jpg}
			\caption{Export Assistent}
			\label{fig:administration}
		\end{figure}	
		
		Der Administrationsbereich setzt sich vorwiegend aus einer aufklappbaren Liste zusammen (Accordion). Dadurch wird dem Administration eine gute Übersicht und einen schnellen Zugriff gewährleistet.
		
		Die Navigation soll entweder unterhalb des Headers oder auf der linken Seite angebracht werden. Dazu soll während der Umsetzung überprüft werden, ob auf der linken Seite genügend Platz vorhanden ist, wenn die Mapping Ansicht geöffnet ist.
		
		
	\section{Finales Userinterface}
	
		Anhand der Mockups und entwickelten Workflows wurde anschliessend das finale Interface umgesetzt.
		Dabei wurden iterativ Bereiche angepasst und verbessert.
		Dies war insbesondere in der Darstellung der Problems \& \ttpl s notwendig, 
		da wesentlich mehr Daten visualisiert werden sollten, als ursprünglich im Mockup vorgesehen.
		
		\begin{figure}[H]
			\centering
			\includegraphics[width=\linewidth]{tutorial/img/eeppiHomeScreen.jpg}
			\caption{EEPPI Home Screen im Browser}
			\label{fig:eeppiHomeScreen}
		\end{figure}	
		
		Auf die Umsetzung des Dashboard wurde verzichtet, da viele der dafür notwendigen Informationen im tief priorisierten und darum nicht umgesetzten Feature "<Inform user about network and system status"> enthalten waren und entsprechend nicht verfügbar waren.
		Anstelle wurde ein einfacher Home Screen implementiert (siehe Abbildung~\ref{fig:eeppiHomeScreen}).
		
		
		\begin{figure}[H]
			\centering
			\includegraphics[width=\linewidth]{tutorial/img/eeppiDecisionsAndTaskTemplates.png}
			\caption{Problems \& \ttpl s}
			\label{fig:eeppiDecisionsAndTaskTemplates}
		\end{figure}	
		
		Die Ansicht der verfügbaren Problems, der \ttpl s und deren Mapping wurde nach dem Mockup umgesetzt und anschliessend noch um viele zusätzliche Informationen erweitert (Abbildung~\ref{fig:eeppiDecisionsAndTaskTemplates}).
		So kam beispielsweise zur Detailansicht der Mappings noch eine Detailansicht des ausgewählten Problems hinzu.
		Damit kann sich der Benutzer eine bessere Übersicht darüber verschaffen,
		ob er das richtige Problem mappt.
		Insbesondere bei ähnlichen Namen der Problems ist dies von Vorteil.
		In diesen Bereich wurde auch eine HTML-Unterstützung für Notes eingebaut, 
		sodass diese mit HTML-Tags ausgezeichnet werden können.
		
		
		\begin{figure}[H]
			\centering
			\includegraphics[width=\linewidth]{tutorial/img/administrationDKS_cut.jpg}
			\caption{Administrationsbereich}
			\label{fig:eeppiAdministration}
		\end{figure}	
		
		Beim Administrationsbereich (Abbildung~\ref{fig:eeppiAdministration}) haben wir uns nur sehr beschränkt an die Mockups gehalten, 
		da wir bald gemerkt haben, dass die darzustellenden Informationen nicht mit dem Mockup zusammenpassen.
		Zum Zeitpunkt, als wir die Mockups erstellten, 
		war noch nicht genau klar, welche Informationen in diesem Bereich untergebracht werden sollen.		
		
		
		\begin{figure}[H]
			\centering
			\includegraphics[width=\linewidth]{tutorial/img/accountPPTAccount_cut.jpg}
			\caption{Accountverwaltung}
			\label{fig:eeppiAccountManagement}
		\end{figure}	
		
		Für den Accountverwaltungsbereich (Abbildung~\ref{fig:eeppiAccountManagement}) gab es keine Mockups, da zum damaligen Zeitpunkt noch Unklarheit über das Usermanagement bestand.
		Entsprechend wurde dieses analog der Administration gestaltet.
		
			
		\begin{figure}[H]
			\centering
			\includegraphics[width=\linewidth]{tutorial/img/transmit2.png}
			\caption{Übertragen von Tasks an ein \ppt}
			\label{fig:eeppiTransmissionScreen}
		\end{figure}	
		
		Im Übertragungsbereich (Abbildung~\ref{fig:eeppiTransmissionScreen}) ist gegenüber den Mockups die Editierfunktion für zu exportierende Tasks entfallen,
		dieses Feature wurde vom Betreuer, als Ansprechpartner der Kundengruppe, mit niedriger Priorität eingestuft.

% performance analyse

% Erkenntnisse, Schlussfolgerungen
%\chapter{Erkenntnisse} 

%TODO: @Tobi: Was soll in das Kapitel? Ist das nicht einfach ein Synonym für Schlussfolgerungen? Und evtl. die beiden Kapitel in "Ergebnisse" umtaufen?
\chapter{Schlussfolgerungen}
%TODO: evtl. in "Ergebnisse" umtaufen
	
	\section{Zielerreichung}
		Das Ziel von \eeppi\ gemäss Aufgabenstellung war:
		\begin{quote}
			Ziel für den Kunden ist es, aus noch offenen und aus bereits getroffenen Architekturentscheidungen  Aufgaben (Tasks) abzuleiten und in eine Taskmanagementsoftware zu überführen, um diese anschliessend in dieser Software verwalten zu können. In dieser Arbeit sollen das Mapping-Konzept und die Tool-Architektur entworfen sowie eine Implementierung in Form eines Tools erstellt werden.
			
			Das Mapping-Konzept beinhaltet die Art der Abbildung von Entscheidungen aus dem CDAR-Tool auf Tasks eines Projektmanagementtools. Die zu entwerfende Toolarchitektur zeigt den konkreten Aufbau einer solchen Applikation.
		\end{quote}
	Und daraus waren die folgenden drei Fragen abgeleitet:
	\begin{enumerate}
		\item{Wie lässt sich Entscheidungsbedarf in Form von agilen Planungsitems darstellen?}
		\item{Welche Umsetzungstasks ergeben sich aus getroffenen Entscheidungen?}
		\item{Wie können die Metamodelle und Werkzeugschnittstellen der beiden Domänen Architekturentscheidungen und Projektplanung aufeinander abgebildet und miteinander integriert werden?}
	\end{enumerate}
	
	\subsection{Hauptziele}
		Das Hauptziel der Arbeit war es, für das Ableiten von Tasks basierend auf Architekturentscheidungen:
		\begin{enumerate}
			\item{ein Mapping-Konzept zu erstellen}
			\item{eine Tool-Architektur zu entwerfen}
			\item{eine Implementierung in Form eines Tools zu erstellen}
		\end{enumerate}
		Alle diese Ziele wurden erfüllt,
		die ersten beiden Ziele waren eine Vorbedingung für das dritte Ziel
		und da die Webapplikation \eeppi\ existiert, ist auch das dritte Ziel erreicht.
	
	\subsection{Beantwortung der Fragen}
		\subsubsection{Wie lässt sich Entscheidungsbedarf in Form von agilen Planungsitems darstellen?}
			Mit \eeppi\ lassen sich sowohl getroffene als auch noch offene Entscheidungen verwalten,
			und der Benutzer kann beiden ein Task-Template zuordnen.
			
			Im Allgemeinen kann gesagt werden,
			dass auch das Treffen einer Entscheidung eine Aufgabe ist
			und deshalb durch einen Task in einem \ppt\ repräsentiert werden kann.
			Je nach Grösse, Komplexität und Wichtigkeit der Entscheidung kann der Benutzer dafür einen kleinen Tasks (wie "'Entscheid treffen"') oder auch mehrere grosse Tasks erstellen (wie "'Entscheidungssitzung abhalten"' oder "'Alternativen evaluieren"').

		\subsubsection{Welche Umsetzungstasks ergeben sich aus getroffenen Entscheidungen?}
			Dies kommt sehr stark auf die entsprechende Entscheidung an.
			Je nach Projekt kann es durchaus generische Tasks wie "'Entscheid dokumentieren"' oder "'Entscheid kommunizieren"' geben,
			doch die meisten Umsetzungstasks sind sehr spezifisch für einzelne Entscheidungen.
		
		\subsubsection{Wie können die Metamodelle und Werkzeugschnittstellen der beiden Domänen Architekturentscheidungen und Projektplanung aufeinander abgebildet und miteinander integriert werden?}
		%TODO Tobi: Bin mir nicht sicher, ob ich die Frage richtig verstanden und beantwortet habe: bitte nochmals genau durchschauen
			\eeppi\ bindet sowohl Architekturentscheidungstools als auch \ppt s als externe Systeme an
			und ermöglicht dem Benutzer \eeppi\ sehr flexibel an die beiden Domänen anzupassen.
			
			Modellmässig stellt \eeppi\ sehr wenige Anforderungen an die beiden Domänen,
			diese Anforderungen sind jedoch nötig um eine gemeinsame Basis finden zu können.
			Konkret muss ein Architekturentscheidungstool folgende Eigenschaften aufweisen:
			\begin{itemize}
				\item{Unterscheidung zwischen Vorlagen und konkreten Elementen}
				\item{Unterscheidung zwischen Eltern- (z.B. Problemen) und Kinder-Elementen (z.B. Optionen)}
				\item{Möglichkeit zur Auflistung aller Elemente}
			\end{itemize}
			
			Und ein \ppt\ muss im Prinzip lediglich einzelne Elemente (z.B. Tasks) erstellen können.
			
			Konkrete Anforderungen an die Elemente der beiden Domänen gibt es nicht,
			es ist aber für eine effiziente Verwendung förderlich,
			wenn sie eine Vielzahl von gleichen oder zumindest ähnlichen Attribute aufweisen.
			Diese können dann mit \eeppi\ vom Architekturentscheidungstool ins \ppt\ übernommen werden.

	\subsection{Erfolgsfaktoren}
		In der Aufgabenstellung wurden auch drei kritische Erfolgsfaktoren festgelegt,
		die jetzt hier nochmals genau analysiert werden.
		
		\subsubsection{Niedrige Einstiegshürden für User}
			Dieser Erfolgsfaktor wird in drei Teile zerlegt.
			Der erste Punkt ist der geringe Installationsaufwand.
			Die Installation von \eeppi\ ist so einfach wie möglich gehalten und im Abschnitt\ \ref{sec:installation} genau beschrieben.
			Ein weiterer Punkt sind Lizenzfragen,
			eine zu einschränkende Lizenz würde viele Firmenbenutzer abschrecken.
			Deshalb ist \eeppi, wie in Abschnitt\ \ref{sec:licensing} beschrieben, unter einer offenen Lizenz.
			Und der letzte Punkt ist die Robustheit im Betrieb.
			\eeppi\ war als Forschungsprojekt nicht längere Zeit im Produktivbetrieb
			und konnte sich deshalb damit nicht beweisen.
			Zur Verbesserung der Robustheit hat das Projektteam jedoch entschieden,
			als letzte Iteration während der Entwicklung keine Funktionen mehr einzubauen,
			sondern \eeppi\ zu stabilisieren und Fehler auszumerzen.
		
		\subsubsection{Modularität und Erweiterbarkeit}
			Bei diesem Punkt geht es darum, dass \eeppi\ gute Schnittstellen und eine gute Dokumentation aufweisen.
			Schnittstellen bietet wie in der Architektur beschrieben lediglich der Server
			und auch nur der Server verwendet externe Schnittstellen.
			Der Client verwendet nur die Schnittstelle des Servers
			und damit zeigt sich auch, dass die Schnittstelle des Servers funktioniert - zumindest in einem Fall.
			Und wie in Abschnitt \ref{sec:apiDocumentation} erläutert, gibt es auch eine ausführliche Dokumentation dazu.
			Der grundlegende Aufbau von \eeppi\ ist mit diesem Dokument hier dokumentiert
			und kann als Einstieg für Folgearbeiten verwendet werden.
			
		\subsubsection{Reife der Konzepte}
			Bei diesem Erfolgsfaktor geht es um die Konfigurierbarkeit, die Flexibilität und die Eleganz der Mapping Konzepte von \eeppi.
			Bei der Entwicklung von \eeppi\ hat das Projektteam äusserst Wert auf genau diese Punkte gelegt.
			Man kann sehr viele Einstellungen konfigurieren und auch die grundlegende Funktion,
			das Übertragen von Tasks an ein \ppt\ ist sogar durch den Benutzer im Frontend konfigurierbar.
			Das Mapping wurde so ausgelegt, dass der Benutzer nicht direkt Tasks erstellen muss,
			sondern dass er in Form eines Metamappings Task-Vorlagen erstellt,
			aus denen bei einer Übertragung in ein \ppt\ dann Tasks erstellt werden.


	
	\section{offene Punkte, Bugs/technische Probleme}
	%TODO: maybe write this! :-)
	
	
	\section{Zusätzliche Features}
	%TODO: write this! :-)
	hier z.B. API-Doku
	
	
	
	\section{Erweiterungs-Möglichkeiten}
		Eine Software ist nie perfekt und kann noch immer erweitert werden.
		Entsteht das Bedürfnis, \eeppi\ zu erweitern, bieten sich verschiedene Möglichkeiten, die nachfolgend kurz angerissen werden.
		Welche Möglichkeit empfohlen wird, kann nicht generell gesagt werden.
		Denn es unterscheidet sich je nachdem, welche Funktion erweitert werden soll.

		\subsection{Erweiterung der bestehenden Applikation}
			Die naheliegendste Erweiterungsmöglichkeit ist, direkt die bestehende Implementierung von \eeppi\ anzupassen.
	
		\subsection{Ersatz des Servers}
			Der Server verwaltet primär die Persistenz.
			Soll diese grundlegend verändert werden oder werden die verwendeten Technologien veraltet,
			kann man den ganzen Server ersetzen, ohne den Client anpassen zu müssen.
	
		\subsection{Ersatz des Clients}
			Der Client besitzt den grössten Teil der Logik, so auch die Processors,
			und ist verantwortlich für die Darstellung auf dem Bildschirm.
			Soll dies grundlegend verändert werden oder werden die verwendeten Technologien veraltet,
			kann man den ganzen Client ersetzen oder parallel zum bestehenden Client einen neuen erstellen.

		\subsection{Erstellung eines Proxies}
			Da \eeppi\ eine geringe Kopplung zu den externen Systemen aufweist,
			 ist es mit ziemlich wenig Aufwand möglich,
			gewisse Funktionen als Proxy\footnote{Proxy-Pattern: \url{http://c2.com/cgi/wiki?ProxyPattern}} zu implementieren.
			Dabei würde der bestehende Teil von \eeppi\ nicht verändert werden,
			sonder lediglich neue Referenzen zu den externen Systemen konfiguriert.
			Dieser Ansatz würde sich beispielsweise gut für die Implementierung von weiteren \ppt-Schnittstellen eignen (siehe \ref{subsec:morePPTInterfaces}).
			
		\subsection{Verwendung der API}
			Gewisse mögliche neuen Funktionen basieren auf der Kernfunktionen von \eeppi,
			bieten aber einen weitergehenden Nutzen für den Benutzer
			(wie beispielsweise der Rückfluss von Informationen des Tasks zurück in das Task-Template, siehe \ref{subsec:informationFlowbackFeature}).
			Solche Funktionen könnten durch eine eigenständige Applikation implementiert werden,
			welche das API von \eeppi\ verwendet.


	\section{Mögliche Erweiterungen}
		\eeppi\ deckt die notwendige Kernfunktionalität ab, damit Benutzer angenehm arbeiten können.
		Im Rahmen einer Weiterentwicklung sind viele Möglichkeiten denkbar. 
		Einige wie ein Rechte-System sind eher praktischer Natur, 
		andere wie beispielsweise Vererbungsmöglichkeiten für Task-Templates würden die Wiederverwendbarkeit verbessern und dem Benutzer einen echten Mehrwert bieten.
		Nachfolgend sind einige denkbare Erweiterungen von \eeppi\ erklärt:
		
		\subsection{Rechte und Rollen}
			Mit einem Rechte- und Rollen-Konzept könnten der Zugriff auf die Daten in \eeppi\ eingeschränkt werden,
			sodass sich nicht mehr jedermann registrieren könnte und damit Zugriff auf alles hätte.
			
			Zudem würde ein Rechte- und Rollen-Konzept die Möglichkeit der Mandantenfähigkeit bieten.
			\eeppi\ könnte als Cloud-Service angeboten werden und eine einzige Instanz könnte für mehrere Kunden verwendet werden.
			
		
		\subsection{Vererbende Tasktemplate}
			Task-Templates, die Eigenschaften von andern Templates erben können, 
			minimieren die Anzahl notwendiger Templates und erhöhen die Wiederverwendbarkeit.
			
			Es könnte beispielsweise ein Task-Template für Sitzungen erstellt werden
			und davon abgeleitet dann für verschiedene Sitzungsarten wiederum Task-Templates.
			Und dem ersten Task-Template könnten dann Subtasks angehängt werden
			(wie "'Zur Sitzung einladen"' oder "'Sitzungs-Protokoll versenden"'),
			welche dann bei den Task-Templates der verschiedenen Sitzungsarten auch dabei wären.
			
		
		\subsection{Reporting}
			Eine anderer Erweiterungsmöglichkeit für \eeppi\ wäre dem Benutzer eine Anzeige zu bieten,
			in welcher eine Liste der in das \ppt\ exportierten Tasks angezeigt würde.
			Dies würde ihm den Umweg über das \ppt\ ersparen.
		
		
		\subsection{Undo von übertragenen Tasks}
			Unter Umständen möchte ein Benutzer erstellte Tasks wieder zurückziehen,
			beziehungsweise wieder aus dem \ppt\ löschen.
			Sofern \ppt s dies unterstützen, würde dies dem Benutzer eine Undo-Möglichkeit für fehlerhafte erstellte Tasks anbieten.
		
		
		\subsection{Bearbeitungsmöglichkeiten für zu übertragende Tasks}
			Unter Umständen möchte der Benutzer die Tasks nicht direkt so in das \ppt\ exportieren,
			wie es im Task-Template entworfen ist.
			Eine weitere Möglichkeit wäre es darum dem Benutzer vor dem definitiven Export eine Möglichkeit zu bieten,
			die zu exportierenden Tasks noch zu bearbeiten.
			Aktuell muss er sie zuerst exportieren und dann direkt im \ppt\ anpassen.
			
			
		\subsection{Kaskadierende Processors}
			Die aktuelle Ausgabe von \eeppi\ unterstützt keine Processors innerhalb von Processors.
			Dem Benutzer würde das verwenden von Processorwerten innerhalb eines Processors die Möglichkeit bieten, Processors wiederzuverwenden und generischer zu gestalten.
			
			
		\subsection{System Status Notification}
			Ein weiteres kleineres Feature wäre eine Anzeige über den Status der angebundenen Systeme,
			eventuell in der Form eines einfachen Icons.
			Ob die Systeme (korrekt) konfiguriert sind, ob sie erreichbar sind
			und je nach System unter Umständen auch noch ob sie korrekt arbeiten.
		
		
		
	
		\subsection{Verwendung mehrerer DKS}
			Aktuell unterstützt \eeppi\ nur ein DKS, welches aber konfiguriert werden kann.
			Eine mögliche Erweiterung wäre, dass der Benutzer mehrere DKS konfigurieren könnte
			und dann auch zum Benutzen auswählen könnte.
		
		
		\subsection{Verwendung mehrerer Projekte}
			Wie beim vorherigen Punkt, \eeppi\ unterstützt auch nur ein einziges Projekt.
			Die Idee ist auch hier, dieses für den Benutzer konfigurierbar zu machen,
			damit er insbesondere Task-Templates auch nur innerhalb eines Projekts verwenden kann
			und dadurch auch dieses geistige Eigentum schützen kann.


		\subsection{Unterscheidung von verschiedenen \ppt s}
			Benutzer können aktuell in \eeppi\ Request-Templates sowie \ppt-Accounts einem \ppt-Typ zuordnen.
			Allerdings unterstützt \eeppi\ aktuell nur einen einzigen \ppt-Typ(-Identifikator)
			und dem werden deshalb auch alle Request-Templates und Accounts zugeordnet.
			
			Eine Unterstützung für mehreren \ppt-Typ(-Identifikatoren) würde Benutzern Verwirrung ersparen,
			welchen Account sie jetzt für welches Request-Template verwenden können.
			
		\subsection{Rückfluss von Informationen des Tasks zurück in Task-Template}
		\label{subsec:informationFlowbackFeature}
			Wenn Benutzer ein Task aus einem Task-Template erstellen landet dieser im \ppt.
			Der Benutzer bearbeitet ihn dort und aktualisiert ihn aufgrund des Projektverlaufs.
			Die Erfahrungen, die dabei gemacht werden, bleiben im Task im \ppt
			und bringen weiteren Projekten keinen Gewinn.
			
			Eine Erweiterungsmöglichkeit wäre es,
			Erkenntnisse aus Projekten wieder in die Task-Templates zurück zu bringen.
			Unter Umständen gäbe es gar die Möglichkeit, dies teilautomatisiert zu tun.

		\subsection{Weitere \ppt-Schnittstellen}
		\label{subsec:morePPTInterfaces}
			Aktuell unterstützt \eeppi\ \ppt s, die über eine Json-Schnittstelle verfügen
			und eine Authentifizierung über Http-Basic-Authentication ermöglichen.
			Für diese Art der Authentifizierung muss Benutzername und Passwort in Klartext vorliegen.
			
			Für eine erhöhte Sicherheit und Kompatibilität wäre eine weitere Erweiterungsmöglichkeit von \eeppi,
			dass \ppt s auch über weitere Schnittstellen (beispielsweise XML)
			und weitere Authentifizierungs-Methoden (beispielsweise OAuth\footnote{Ein offenes Authentifizierungs-Protokoll, welches Authentifizierung auch ohne das effektive Passwort ermöglicht: \url{http://oauth.net/2/}}) angesprochen werden können.
			

%	\section*{Rückblick Tobias Blaser}
		% Persönliche Berichte einschliesslich (selbst-)kritische Reflexion der Studierenden zu ihren Erfahrungen bei der Arbeit


\chapter{Glossar}
	\begin{description}
		\item[API]{Application Programming Interface, Schnittstelle eines Programms zum Austausch mit externen Anwendungen}
		\item[Boilerplate-Code]{Code der (wiederholt) geschrieben werden muss, ohne dass er wirklich etwas zum Projekt beiträgt}
		\item[CSS]{Cascading Style Sheets: Auszeichnungssprache für Webdokumente, um diese grafisch zu gestalten}
		\item[\cdar]{Collaborative Decision Management and Architectural Refactoring Tool \cite{tinner_collaborative_2014}}
		\item[CORS]{Cross-Origin Resource Sharing: Teilen von REST-Ressourcen über mehrere Origins (Server)}
		\item[DKS]{\dks}
		\item[\eeppi]{Entwurfsentscheidungen als Projektplanungsinstrument}
		\item[Entscheidungsprojekt]{siehe Solution Space}
		\item[IFS]{Institut für Software, HSR Hochschule für Technik: \url{http://www.ifs.hsr.ch/}}
		\item[Less]{Sprache, um CSS vereinfacht zu generieren}
		\item[LOC]{Lines of Code: Anzahl Zeilen Code}
		\item[Problem Space]{In \cdar\ abgelegtes Wissen, welches später über einen Solution Space hilft Projekte umzusetzen. (siehe auch Kapitel~\ref{userstoryDefinitions})}
		\item[\ppt]{Ein Tool, welches Tasks verwaltet. Es wird zum Planen und Durchführen von Projekten verwendet. (siehe auch Kapitel~\ref{userstoryDefinitions})}
		\item[REST]{Representational State Transfer: Art einer Schnittstelle von Webapplikationen. Jede Ressource hat eine eigene URL und kann mit der korrekten Verwendung der HTTP-Verben bearbeitet werden. (siehe auch \url{http://en.wikipedia.org/wiki/Representational_state_transfer})}
		\item[Selenium]{Testframework, welches einen Browser startet und direkt darin die Webseite testet (siehe auch \url{http://www.seleniumhq.org/})}
		\item[SLOC]{Source Lines of Code: Total Anzahl Zeilen eines Codes, inklusive der irrelevanten Zeilen wie Leerzeilen}
		\item[Solution Space]{Kopie eines Problem Spaces um konkret Entscheide für ein Projekt zu erstellen. (siehe auch Kapitel~\ref{userstoryDefinitions})}
		\item[Tasks]{Aufgabe, welche üblicherweise in einem \ppt\ abgelegt ist. (siehe auch Kapitel~\ref{userstoryDefinitions})}
		\item[TDD]{Test Driven Development: Ansatz des Entwicklungsprozesses, bei welchem zuerst die Tests geschrieben werden und erst danach der entsprechende Code}
		\item[Vagrant]{Virtualisierungsautomatisierungslösung von HashiCorp für verschiedene Virtualisierungsumgebungen: \url{http://www.vagrantup.com/}}
		\item[Wissensbaum]{siehe Problem Space}
		\item[Wissenskonsument]{Person, die \cdar\ und \eeppi\ für die Umsetzung von einem Projekt verwendet. (siehe auch Kapitel~\ref{userstoryDefinitions})}
		\item[Wissensproduzent]{Person, die Wissen im \cdar/\eeppi\ erfasst. (siehe auch Kapitel~\ref{userstoryDefinitions})}
		\item[MVW/MV*] Mode View Whatever: Fasst die Patterns MVC - Model View Controller, MVP - Model View Presenter, MVVM - Model View Viewmodel und ähnliche Patterns zusammen.
	\end{description}
\nocite{*}
\printbibliography
\listoffigures

\appendix
\documentPartEntry{Anhang Technischer Bericht}


% requirements
\chapter{Anforderungen}
\section{Allgemeine Beschreibung}

\subsection{Produktfunktion}
Softwarearchitekten treffen und dokumentieren mit CDAR Architekturentscheidungen.
"`EEPPI"' soll es ihnen ermöglichen, ein Mapping zwischen Architekturentscheidungen und Projekttasks zu erstellen.
Ebenfalls soll es möglich sein, ein Mapping von Taskeigenschaften auf Felder einer API einer Projektplanungssoftware zu erstellen.
Über diese beiden Mappings soll dem Architekten ermöglicht werden, aus Architekturentscheidungen Tasks in der Projektplanungssoftware zu erstellen.

\subsection{Benutzer-Charakteristik}
Zielgruppe des "`EEPPI"' sind Softwarearchitekten und Projektplaner.

\subsection{Einschränkungen}
Voraussetzung für die Nutzung von "`EEPPI"' ist gundsätzliches Wissen über Softwarearchitektur, Architekturentscheidungen, Projektplanung sowie über die grundlegende Funktion eines Projektplanungstools.


\section{User Stories}
	\subsection{Personas}
		\begin{description}
			\item[Olivia Zander]\label{olivia}\ \newline
				\begin{minipage}[t]{0.35\textwidth} 
					\begin{figure}[H]
						\vspace{-0.75cm}
						\includegraphics[trim=0cm 0cm 0cm 0cm, clip=true, width=5cm]{requirements/media/img/oliviaZander.jpg}
						\caption[Symbolbild Persona Olivia Zander\newline 
							\license{CC BY 2.0 \url{https://creativecommons.org/licenses/by/2.0/}  Official GDC \url{https://www.flickr.com/photos/officialgdc/}}
						]
						{\label{Olivia Zander}}
					\end{figure}
				\end{minipage}
				\begin{minipage}[t]{0.55\textwidth}
					52 Jahre alt.
					Olivia ist eine erfahrene \textbf{Softwarearchitekt}in und arbeitet schon viele Jahre auf dem Beruf.
					Aktuell arbeitet sie in einem kleinen Beratungsunternehmen, welches andere Firmen beim Umstrukturieren von Softwareapplikationen unterstützt.
					Vor einiger Zeit hat sie eine Weiterbildung im Bereich Cloud Computing gemacht.
					Seither hat sie selbst Erfahrungen damit sammeln können, nämlich in verschiedene Beratungsprojekten in welchen es darum ging, bestehende Anwendungen in die Cloud zu bringen.
				\end{minipage}
			\item[Thomas Bucher]\label{thomas}\ \newline
				\begin{minipage}[t]{0.35\textwidth} 
					\begin{figure}[H]
						\vspace{-0.75cm}
						\includegraphics[trim=0cm 0cm 0cm 0cm, clip=true, width=5cm]{requirements/media/img/thomasBucher.jpg}
						\caption[Symbolbild Persona Thomas Bucher\newline
							\license{CC BY 2.0 \url{https://creativecommons.org/licenses/by/2.0/} Steve wilson \url{https://www.flickr.com/photos/125303894@N06/}}
						]
						{\label{Thomas Bucher}}
					\end{figure}
				\end{minipage}
				\begin{minipage}[t]{0.55\textwidth}
					29 Jahre alt.
					Thomas hat vor ein paar Jahren in Rapperswil den Informatik-Bachelor abgeschlossen und arbeitet seither in der gleichen Firma wie Olivia.
					Er arbeitet da als \textbf{Projektplaner} und unterstützt in dieser Funktion aktuell eine externe Firma eine Anwendung mit täglich rund 10'000 Benutzern von ihren lokalen Servern in die Cloud zu transferieren.
				\end{minipage}
		\end{description}
		
	\subsection{Definitionen}\label{userstoryDefinitions}
		Folgende Wörter werden in den User Stories verwendet und sind dafür zum genauen Verständnis hier definiert.
		\begin{description}
			\item[Wissensproduzent] Person, die aktiv neue Entscheidungen und formelle Tasks erfasst (als Beispiel kann hier \hyperref[olivia]{Olivia} dienen).
			\item[Problem space (Wissensbaum)] Ein Projekt im CDAR, in welchem ein Wissensproduzent Wissen ablegt.
			\item[Wissenskonsument] Person, die bestehende Entscheidungen benutzt um damit Entscheide zu fällen (als Beispiel kann hier \hyperref[thomas]{Thomas} dienen).
			\item[Solution space (Entscheidungsprojekt)] Ein Projekt im CDAR, aus welchem ein Wissenskonsument Wissen konsumiert.
				Es ist jeweils eine Kopie eines Wissensbaums.
			\item[Administrator] Person, die für die Konfiguration und den Betrieb von \eeppi\ verantwortlich ist.
			\item[Abbildung] Mit einer Abbildung lässt sich ein Datensatz in einen anderen Datensatz umwandeln.
			\item[erstellen aus] Als Grundlage wird ein Datensatz genommen, aus welchem ein neuer Datensatz erstellt wird.
				Anschliessend sind die beiden Datensätze voneinander unabhängig und Änderungen an einem Datensatz beeinflussen den anderen Datensatz nicht.
			\item[Task] Datensatz, welcher eine Aufgabe beschreibt.
			% Task-Vorlage, Entscheidungs-Vorlage: Wortwahl gemäss Duden Regel 22
			\item[Task-Vorlage] Datensatz, um später daraus Tasks in einem \ppt\ zu erstellen.
			\item[Entscheidungs-Task] Task, welcher zum Treffen einer Entscheidung erledigt werden muss.
				Er ist einer Entscheidung angehängt.
			\item[Operativer Task] Task, welcher durch das Treffen einer Option entsteht.
				Er ist dementsprechend einer Option angehängt.
			\item[Problem space item (Entscheidungs-Vorlage)] Datensatz, der eine Wahl mit mehreren Optionen darstellt.
				Der Datensatz ist jedoch nur eine Vorlage, die Entscheidung kann nicht getroffen werden
			\item[Solution space item (Entscheidung)] Datensatz, der eine Wahl mit mehreren Optionen darstellt.
				Er wird aus einer Entscheidungs-Vorlage erstellt.
				Die Entscheidung kann jetzt getroffen werden.
			\item[Entscheid] Getroffene (entschiedene) Entscheidung.
			\item[Option] Möglichkeit, wie eine Entscheidung entschieden wird.
			\item[entscheiden] Tätigkeit, in welcher für Entscheidungen der Entscheid gefällt wird.
			\item[importieren] Anwenden einer Abbildung zur Aufnahme von Datensätzen in das \eeppi.
			\item[exportieren] Anwenden einer Abbildung zur Ausgabe von Datensätzen aus dem \eeppi.
			\item[\ppt] Externes Programm, welches Tasks verwaltet und für diese Tasks eine eigene Form erwartet.
			\item[API] Schnittstelle eines Programms (sowohl bei externen, als auch CDAR und \eeppi)
		\end{description}

		
	\begin{landscape}
	\subsection{Übersicht über die User Stories}
	
		\eeppi\ hat eine enge Verbindung zu CDAR und deshalb werden in Abbildung~\ref{fig:UserStoryDiagram} auch in einem ersten Schritt gemeinsam die übergeordneten User Stories beschrieben.
	
		\begin{figure}[H]
			\begin{minipage}[b]{\linewidth}
				\includegraphics[width=0.95\textwidth]{media/diagrams/UserStoryDiagram.png}
				\centering
				\caption{Übergeordnete User Stories (inklusive CDAR)}
				\label{fig:UserStoryDiagram}
			\end{minipage}
		\end{figure}
		
		Dabei repräsentieren die drei Aktore (Wissensproduzent, -konsument und Administrator) Personen wie in Abschnitt~\ref{userstoryDefinitions} beschrieben,
		der Business-Aktor (ganz rechts) repräsentiert ein beliebiges \ppt\ und die roten Kästchen referenzieren den dazugehörenden Issue im \eeppi-\ppt.
		Nachfolgend sind die Erklärungen für die sieben aufgezeigten User Stories.
		Die User Stories sind jeweils im Format nach Mike Cohn\cite{jonathan_rasmusson_agile_2012} geschrieben:
		\begin{quote}
			\textbf{As a} <type of user>,\newline
			\textbf{I want} <to perform some task>\newline
			\textbf{so that I can} <achieve some goal/benefit/value>.
		\end{quote}
	\end{landscape}

		\includepdf[pages=-, pagecommand={}, scale=0.975, landscape=true]{requirements/media/documents/stories.pdf}
\section{Weitere Anforderungen}

	\subsection{Qualitätsmerkmale}


		\subsubsection{Functionality}
		\begin{description}
			\item[Interoperabilität] 
			\item[Sicherheit] 
		\end{description}

		\subsubsection{Usability}
		\begin{description}
			\item[Verständlichkeit] 
			\item[Robustheit] 
		\end{description}

		\subsubsection{Portability}
		\begin{description}
			\item[Anpassbarkeit] 
			\item[Installierbarkeit]
			\item[Austauschbarkeit] 
		\end{description}

	\subsection{Schnittstellen}


\documentPartEntry{Anhang Dokumentation}
% personal feedback stories
%	\section*{Rückblick Tobias Blaser}
		% Persönliche Berichte einschliesslich (selbst-)kritische Reflexion der Studierenden zu ihren Erfahrungen bei der Arbeit
%\section*{Rückblick Laurin Murer}
	Das Projekt hat mir aus verschiedenen Gründen sehr grossen Spass bereitet.
	Der erste Grund ist, dass ich mit meinem Projektpartner grosses Glück hatte.
	Obwohl ich bei diesem Projekt das erste Mal mit ihm zusammengearbeitet habe,
	hat die Zusammenarbeit äusserst gut geklappt und wir haben uns oft optimal ergänzt.
	Zudem war das Thema inhaltlich sehr spannend.
	Ich kann mir gut vorstellen, \eeppi\ oder ein ähnliches Tool in der Zukunft einmal beruflich einzusetzen.
	Und auch toll war es, dass mehrheitlich alles sehr gut geklappt hat und wir es in etwa so umsetzen konnten wie wir es auch geplant hatten.
	
	Weniger gefallen hat mir der Umgang mit \LaTeX\ zur Dokumentation des Projekts,
	denn die Formatierung richtig hin zu kriegen war teilweise recht aufwändig
	und nicht mehr mit heutigen technischen Standards vergleichbar.
	Aber Microsoft Word als Alternative hätte ich auch nicht gewollt, da wären einfach andere Probleme aufgetreten.
	
	Spannend fand ich TypeScript kennenzulernen. Diese Sprache habe ich bis jetzt noch nicht gekannt.
	In der Studienarbeit haben wir neue Frameworks und Technologien eingesetzt, was gerade sehr viel Neues war.
	Jetzt in der Bachelorarbeit hatte ich eine neue Technologie und Tobias Blaser ein neues Framework kennengelernt.
	Aus meiner Sicht hat sich dies gelohnt. Es ist immer spannend, etwas Neues zu lernen
	und trotzdem bleibt man nicht andauernd stecken, weil man sich mit der Technologie/dem Framework noch nicht auskennt.
	
	Besonders stolz bin ich auf die API-Dokumentation und wie die darin dargestellten Daten erstellt werden.
	Es gefällt mir sehr, dass ich als Entwickler die Dokumentation gerade im Code erstellen kann
	und dass die angezeigte Rückgabe der API aus echten Aufrufen der API besteht.
	
	Bei einem zukünftigen Projekt würde ich mich für einen leistungsfähigeren Entwicklungsserver einsetzen.
	Bei \eeppi\ hatten wir das Problem, dass der Server zu wenig Leistung besitzt um die für Tests benötigten Vagrant-Umgebungen auszuführen.
	Während dem Projekt wurde ich ein grosser Fan von Vagrant und möchte diese Technologie unbedingt auch in zukünftigen Projekten einsetzen.


% self reliance declaration & uage lisence
\includepdf[pages=-]{media/documents/selfRelianceDeclaration.pdf}
\includepdf[pages=-, pagecommand={}]{media/documents/UsageLicense.pdf}


% projektplan
\chapter{Projektplan}
	Meilensteine, Roadmap, Gant, Zeitausweisung

\chapter*{Risikomanagement}
	\documentPartEntry{Risikomanagement}
	
	\label{risiken}
	
	\section*{Übersicht über die Risiken}
		Um die Risiken im Projekt zu kennen und Massnahmen zu deren Minimierung umsetzen zu können, haben wir zu Beginn des Projekts eine Risikoanalyse erstellt und diese während des Projektverlaufs nach dem Abschluss jedes Meilensteins überarbeitet.
		In Abbildung~\ref{fig:RiskMatrix} sind die im Kapitel~\ref{einzelneRisiken} detailliert aufgeführten Risiken in einer Übersicht dargestellt.
		
		\begin{figure}[H]
			\includegraphics[width=0.9\textwidth]{projectPlan/media/img/risikomatrix.png}
			\centering
			\caption{Risikomatrix}
			\label{fig:RiskMatrix}
		\end{figure}
	
		Abbildung~\ref{fig:RiskMatrix} zeigt übersichtlich in Form eines X/Y-Diagramms, welches Risiko wie wahrscheinlich ist (X-Achse) und wie gross der Schaden maximal bei einem Eintritt geschätzt wurde (Y-Achse).
		Dabei sind jeweils für jedes Risiko die geschätzten Werte bei den Meilensteinen eingetragen.
		
		Es ist deutlich zu sehen, dass das Risiko~1 dabei zu Beginn am kritischsten war und deshalb möglichst früh angegangen wurde.
		Natürlich haben wir auch für die übrigen Risiken Massnahmen getroffen.
		Die für jedes Risiko getroffenen vorbeugenden Massnahmen sind im Kapitel~\ref{einzelneRisiken} direkt bei jedem Risiko selbst aufgeführt.
	

	\section*{Einzelne Risiken}\label{einzelneRisiken}
		\newcounter{riskidcounter}
		
		\newcommand{\riskTable}[3]{
			%Die Spalten werden aufgeteilt auf dem goldenen Schnitt
			\noindent
			\refstepcounter{riskidcounter}
			\begin{tabular}{|p{\smallThird\textwidth} | p{\largeThird\textwidth} |}
				\hline	
				Risiko-ID 		& Risiko \theriskidcounter \\
				\hline
				Titel 			& #1 \\
				Beschreibung 		& #2 \\
				Bewertung		&
				\begin{minipage}[b]{\linewidth}
					\begin{figure}[H]
						\includegraphics[width=0.9\textwidth]{projectPlan/media/img/risiko\theriskidcounter.png}
						\centering
						\caption{Bewertung Risiko \theriskidcounter}
						\label{fig:Risk\theriskidcounter}
					\end{figure}
				\end{minipage} \\
				Vorbeugung/Massnahmen		& #3 \\
				\hline
			\end{tabular}
			\hspace{0.5cm}
			\newline	
		}
		
		\riskTable{Qualität der \dks\ Schnittstelle}
		{Die Schnittstelle entspricht nicht den Anforderungen des Projektes und liefert zu wenige Informationen.}
		{Eignung der Schnittstelle durch den Prototyp abklären.}
		
		\riskTable{Einarbeitung Play Framework}
		{Die Einarbeitung des Teammitglieds, welches das Play Framework noch nicht kennt, dauert länger als angenommen.}
		{Rechtzeitiges Einarbeiten ins Framework}
		
		\riskTable{Mapping Complexity}
		{Die Abbildung des Metamappings ist wesentlich komplexer als angenommen.}
		{Minimales Mapping bereits im Prototyp umsetzen um Komplexität abschätzen zu können. Mapping Features priorisieren, die Kernfunktionalität abdecken, Nebenfunktionalität weglassen.}

		\riskTable{Schnittstelleneinheitlichkeit}
		{Die Schnittstellen der gängigen \ppt s sind zu unterschiedlich, als das sie über einen Adapter mit einer Konfiguration abgedeckt werden können.}
		{Bereits während der Prototypenphase verschiedene Schnittstellen berücksichtigen.}

		\riskTable{Ausfall Infrastruktur}
		{Die Infrastruktur des Projekts fällt aus.}
		{Regelmässig Backups der kritischen Daten erstellen.}

\chapter{Infrastruktur}
	\section{Hardware}
		\begin{itemize}
			\setlength{\itemsep}{-\parsep}
			\item Persönliche Entwicklungsgeräte für jedes Teammitglied, bevorzugt Laptop (eigene Geräte)
			\item Zugewiesene Arbeitsplätze im Zimmer 1.206
			\item Virtual Server für Projektmanagement und Virtueller Server als Entwicklungsserver
		\end{itemize}

		
	\section{Tools}
		\subsection{Projektmanagement}	
			Jira und Redmine bieten identische Funktionalität und es wurden sowohl mit Jira wie mit Redmine gute Erfahrungen gemacht. Redmine ist in der Basiskonfiguration eher auf RUP ausgerichtet, für Agile Entwicklung werden Plugins benötigt. Jira ist in der Basiskonfiguration auf Agile Entwicklung ausgerichtet. Die Bentzeroberfläche von Jira ist etwas moderner und benutzerfreundlicher gestaltet, ansosten sind sich beide Oberflächen jedoch ähnlich.
		
			\begin{description}
				\item[Evaluierte Produkte] Jira, Redmine
				\item[Ausgewähltes Produkt] Jira
				\item[Begründung] Jira bietet eine benutzerfreundliche Oberfläche, die erforderlichen Funktionalität sowie ist grundsätzlich auf Agile Entwicklung ausgerichtet
			\end{description}


		\subsection{Versionsverwaltung}
			\subsubsection{Git}
				Git ist ein bewährtes Versionsverwaltungstool, bietet den Vorteil von lokalen Repositories, ist sehr schlank und bringt eine gute Merge-Automatik mit.

			\subsubsection{GitHub}
				Mit GitHub besitzen die Studenten durch andere Projekte bereits Erfahrung. Als
				Studenten haben sie Zugriff auf kostenlose "`Private-Repositories"'. Zudem
				bietet GitHub noch zusätzliche Funktionen wie Wiki, RST- und MD-Viewer sowie
				Repository-Zugriff und Dateibearbeitung über ein Webinterface.
				
			\subsubsection{Git Flow}
				Git Flow automatisiert häufige Git Operationen für einen Entwicklungs Workflow mit Master-, Develop-, und Featurebranches.

			\subsubsection{Backup}
				Ein zusätzliches Backup ist nicht notwendig, da durch die Versionierung mit Git die komplette Versionshistorie bei jedem Teilnehmer vorhanden ist. Somit ist das gesamte Projekt dreifach abgelegt (bei den Entwicklern sowie bei GitHub).


		\subsection{Dokumentation}
			\subsubsection{Für grosse Dokumentationen und Abgabedokumente: \LaTeX}
				\LaTeX\ ist perfekt geeignet für grosse, gemeinsam zu erarbeitende Dokumente,
				weil die Source-Dateien über Git versioniert und gemergt werden können und wenig
				Platz verbrauchen. Zudem besteht ein sehr kleines Risiko auf Dokumentenverlust
				bzw. Dokumentenfehler durch die Software, weil \LaTeX\ die Source-Dateien gar
				nicht verändert, im Unterschied zu einer Office-Applikation.

			\subsubsection{Für Notizen \& Meetingprotokolle: Restructured Text (rst), txt, Markdown (md)}
				Für Notizen und kleine Dokumente reichen RST, TXT oder MD vollständig aus. Sie
				sind schlank, bieten nur das notwendigste, können versioniert und gemergt
				werden, weil es nur Textfiles sind, und werden von "`GitHub Document aPreview"'
				unterstützt.

			\subsubsection{Für Diagramme, Skizzen: LibreOffice Draw (OpenDocument)}
				Wo es nicht anders geht, wird OpenDocument eingesetzt. Dabei wird
				berücksichtigt, dass es über Git nicht inkrementell versioniert und nicht
				gemerged werden kann.


		\subsection{Modeling}
			Als Modeling-Tool wird Astah gewählt, weil es das beste den Studenten
			bekannte Tool ist.
			Es deckt den geforderten Funktionsumfang grosszügig ab und bietet Image- sowie
			PDF-Export.

			
		%\subsection{UI Drafting}
		


		\subsection{Frameworks}
			\subsubsection{Angular.js}
				Angular.js ist ein bekanntes MVW- und Templating Framework, das eine saubere Trennung von Logik und Darstellung ermöglicht. Angular.js bindet darüber hinaus ViewModel Properties und Functions ans Template, wodurch sich Observerkonstrukte sparen lassen.

			\subsubsection{Require.js}
				Require.js soll zur Strukturierung und Autolading der Klassen und komponenten eingesetzt werden.

			\subsubsection{LESS}
				Less soll als clientseitiger CSS Parser eingesetzt werden, da es den CSS Code stark verschlankt und Vorteile wie Variablen und Mixins bietet.


		\subsection{Testing}
			Testing Framework Anforderungen:
			\begin{itemize}
				\setlength{\itemsep}{-\parsep}
				\item Testing mit realem Browser, Browsersimulationen unterstützen vermutlich WebRTC noch nicht
				\item Einfach einzubinden
				\item Einfach zu erweitern
				\item Bekannte Benutzung mit Tests und Asserts
				\item Möglichkeit zur Anbindung eines Build Tools
			\end{itemize}

			\subsubsection{JsUnit / QUnit}
				JsUnit wie QUnit arbeiten mit einem realen Browser, sind einfach handzuhaben und bieten typische Assert-Syntax.


		\subsection{Building}
			Ein Build-Server wie Ant ist nicht nötig für dieses Projekt. JsUnit bietet zwar
			eine Anbindungsmöglichkeit. Für unsern Anwendungsfall und die nicht sehr
			komplexe Tool-Umgebung lohnt sich der Aufwand eines Build-Servers jedoch nicht.


		\subsection{Entwicklungsumgebung}
			Jeder Entwickler verwendet seine eigene bevorzugte Entwicklungsumgebung. 


		\subsection{RunTime Environment}
			\begin{description}
				\item[Ausgewähltes Environment] Virtuellem Linux Server (Ubuntu Server 14.04, Virtualbox, Vagrant), installiert auf den von der IT zur Verfügung gestellten Workstations.
				\item[Begründung]	 Die Workstations sind über öffentliche Adressen erreichbar und unterliegen weniger Einschränkungen als die von der IT zur Verfügung gestellte Virtuelle Server. So können von der IT extern geblockte Ports auf dem eigenen Server vom lokal Netz aus trotzdem genutzt werden. Zudem müssen auch die Virtuellen Server der IT selbst verwaltet und gebackupt werden wodurch sich keine wirklichen Vorteile bei deren Nutzung ergeben.
			\end{description}


% TODO: Meeting reports: Convert md to pdf and include


% poster
%\begin{landscape}
%	\includepdf[pages=-]{media/documents/poster.pdf}
%\end{landscape}


\end{document}
